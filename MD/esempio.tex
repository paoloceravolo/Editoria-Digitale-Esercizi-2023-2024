% Options for packages loaded elsewhere
\PassOptionsToPackage{unicode}{hyperref}
\PassOptionsToPackage{hyphens}{url}
%
\documentclass[
]{article}
\usepackage{amsmath,amssymb}
\usepackage{iftex}
\ifPDFTeX
  \usepackage[T1]{fontenc}
  \usepackage[utf8]{inputenc}
  \usepackage{textcomp} % provide euro and other symbols
\else % if luatex or xetex
  \usepackage{unicode-math} % this also loads fontspec
  \defaultfontfeatures{Scale=MatchLowercase}
  \defaultfontfeatures[\rmfamily]{Ligatures=TeX,Scale=1}
\fi
\usepackage{lmodern}
\ifPDFTeX\else
  % xetex/luatex font selection
\fi
% Use upquote if available, for straight quotes in verbatim environments
\IfFileExists{upquote.sty}{\usepackage{upquote}}{}
\IfFileExists{microtype.sty}{% use microtype if available
  \usepackage[]{microtype}
  \UseMicrotypeSet[protrusion]{basicmath} % disable protrusion for tt fonts
}{}
\makeatletter
\@ifundefined{KOMAClassName}{% if non-KOMA class
  \IfFileExists{parskip.sty}{%
    \usepackage{parskip}
  }{% else
    \setlength{\parindent}{0pt}
    \setlength{\parskip}{6pt plus 2pt minus 1pt}}
}{% if KOMA class
  \KOMAoptions{parskip=half}}
\makeatother
\usepackage{fancyvrb}
\usepackage{xcolor}
\usepackage{color}
\usepackage{fancyvrb}
\newcommand{\VerbBar}{|}
\newcommand{\VERB}{\Verb[commandchars=\\\{\}]}
\DefineVerbatimEnvironment{Highlighting}{Verbatim}{commandchars=\\\{\}}
% Add ',fontsize=\small' for more characters per line
\newenvironment{Shaded}{}{}
\newcommand{\AlertTok}[1]{\textcolor[rgb]{1.00,0.00,0.00}{\textbf{#1}}}
\newcommand{\AnnotationTok}[1]{\textcolor[rgb]{0.38,0.63,0.69}{\textbf{\textit{#1}}}}
\newcommand{\AttributeTok}[1]{\textcolor[rgb]{0.49,0.56,0.16}{#1}}
\newcommand{\BaseNTok}[1]{\textcolor[rgb]{0.25,0.63,0.44}{#1}}
\newcommand{\BuiltInTok}[1]{\textcolor[rgb]{0.00,0.50,0.00}{#1}}
\newcommand{\CharTok}[1]{\textcolor[rgb]{0.25,0.44,0.63}{#1}}
\newcommand{\CommentTok}[1]{\textcolor[rgb]{0.38,0.63,0.69}{\textit{#1}}}
\newcommand{\CommentVarTok}[1]{\textcolor[rgb]{0.38,0.63,0.69}{\textbf{\textit{#1}}}}
\newcommand{\ConstantTok}[1]{\textcolor[rgb]{0.53,0.00,0.00}{#1}}
\newcommand{\ControlFlowTok}[1]{\textcolor[rgb]{0.00,0.44,0.13}{\textbf{#1}}}
\newcommand{\DataTypeTok}[1]{\textcolor[rgb]{0.56,0.13,0.00}{#1}}
\newcommand{\DecValTok}[1]{\textcolor[rgb]{0.25,0.63,0.44}{#1}}
\newcommand{\DocumentationTok}[1]{\textcolor[rgb]{0.73,0.13,0.13}{\textit{#1}}}
\newcommand{\ErrorTok}[1]{\textcolor[rgb]{1.00,0.00,0.00}{\textbf{#1}}}
\newcommand{\ExtensionTok}[1]{#1}
\newcommand{\FloatTok}[1]{\textcolor[rgb]{0.25,0.63,0.44}{#1}}
\newcommand{\FunctionTok}[1]{\textcolor[rgb]{0.02,0.16,0.49}{#1}}
\newcommand{\ImportTok}[1]{\textcolor[rgb]{0.00,0.50,0.00}{\textbf{#1}}}
\newcommand{\InformationTok}[1]{\textcolor[rgb]{0.38,0.63,0.69}{\textbf{\textit{#1}}}}
\newcommand{\KeywordTok}[1]{\textcolor[rgb]{0.00,0.44,0.13}{\textbf{#1}}}
\newcommand{\NormalTok}[1]{#1}
\newcommand{\OperatorTok}[1]{\textcolor[rgb]{0.40,0.40,0.40}{#1}}
\newcommand{\OtherTok}[1]{\textcolor[rgb]{0.00,0.44,0.13}{#1}}
\newcommand{\PreprocessorTok}[1]{\textcolor[rgb]{0.74,0.48,0.00}{#1}}
\newcommand{\RegionMarkerTok}[1]{#1}
\newcommand{\SpecialCharTok}[1]{\textcolor[rgb]{0.25,0.44,0.63}{#1}}
\newcommand{\SpecialStringTok}[1]{\textcolor[rgb]{0.73,0.40,0.53}{#1}}
\newcommand{\StringTok}[1]{\textcolor[rgb]{0.25,0.44,0.63}{#1}}
\newcommand{\VariableTok}[1]{\textcolor[rgb]{0.10,0.09,0.49}{#1}}
\newcommand{\VerbatimStringTok}[1]{\textcolor[rgb]{0.25,0.44,0.63}{#1}}
\newcommand{\WarningTok}[1]{\textcolor[rgb]{0.38,0.63,0.69}{\textbf{\textit{#1}}}}
\usepackage{longtable,booktabs,array}
\usepackage{calc} % for calculating minipage widths
% Correct order of tables after \paragraph or \subparagraph
\usepackage{etoolbox}
\makeatletter
\patchcmd\longtable{\par}{\if@noskipsec\mbox{}\fi\par}{}{}
\makeatother
% Allow footnotes in longtable head/foot
\IfFileExists{footnotehyper.sty}{\usepackage{footnotehyper}}{\usepackage{footnote}}
\makesavenoteenv{longtable}
\setlength{\emergencystretch}{3em} % prevent overfull lines
\providecommand{\tightlist}{%
  \setlength{\itemsep}{0pt}\setlength{\parskip}{0pt}}
\setcounter{secnumdepth}{-\maxdimen} % remove section numbering
\ifLuaTeX
  \usepackage{selnolig}  % disable illegal ligatures
\fi
\IfFileExists{bookmark.sty}{\usepackage{bookmark}}{\usepackage{hyperref}}
\IfFileExists{xurl.sty}{\usepackage{xurl}}{} % add URL line breaks if available
\urlstyle{same}
\VerbatimFootnotes % allow verbatim text in footnotes
\hypersetup{
  pdftitle={Pandoc User's Guide},
  pdfauthor={John MacFarlane},
  hidelinks,
  pdfcreator={LaTeX via pandoc}}

\title{Pandoc User's Guide}
\author{John MacFarlane}
\date{August 22, 2022}

\begin{document}
\maketitle

\section{Synopsis}\label{synopsis}

\texttt{pandoc} {[}\emph{options}{]} {[}\emph{input-file}{]}\ldots{}

\section{Description}\label{description}

Pandoc is a \href{https://www.haskell.org}{Haskell} library for
converting from one markup format to another, and a command-line tool
that uses this library.

Pandoc can convert between numerous markup and word processing formats,
including, but not limited to, various flavors of
\href{https://daringfireball.net/projects/markdown/}{Markdown},
\href{https://www.w3.org/html/}{HTML},
\href{https://www.latex-project.org/}{LaTeX} and
\href{https://en.wikipedia.org/wiki/Office_Open_XML}{Word docx}. For the
full lists of input and output formats, see the \texttt{-\/-from} and
\texttt{-\/-to} \hyperref[general-options]{options below}. Pandoc can
also produce \href{https://www.adobe.com/pdf/}{PDF} output: see
\hyperref[creating-a-pdf]{creating a PDF}, below.

Pandoc's enhanced version of Markdown includes syntax for
\hyperref[tables]{tables}, \hyperref[definition-lists]{definition
lists}, \hyperref[metadata-blocks]{metadata blocks},
\hyperref[footnotes]{footnotes}, \hyperref[citations]{citations},
\hyperref[math]{math}, and much more. See below under
\hyperref[pandocs-markdown]{Pandoc's Markdown}.

Pandoc has a modular design: it consists of a set of readers, which
parse text in a given format and produce a native representation of the
document (an \emph{abstract syntax tree} or AST), and a set of writers,
which convert this native representation into a target format. Thus,
adding an input or output format requires only adding a reader or
writer. Users can also run custom
\href{https://pandoc.org/filters.html}{pandoc filters} to modify the
intermediate AST.

Because pandoc's intermediate representation of a document is less
expressive than many of the formats it converts between, one should not
expect perfect conversions between every format and every other. Pandoc
attempts to preserve the structural elements of a document, but not
formatting details such as margin size. And some document elements, such
as complex tables, may not fit into pandoc's simple document model.
While conversions from pandoc's Markdown to all formats aspire to be
perfect, conversions from formats more expressive than pandoc's Markdown
can be expected to be lossy.

\subsection{Using pandoc}\label{using-pandoc}

If no \emph{input-files} are specified, input is read from \emph{stdin}.
Output goes to \emph{stdout} by default. For output to a file, use the
\texttt{-o} option:

\begin{verbatim}
pandoc -o output.html input.txt
\end{verbatim}

By default, pandoc produces a document fragment. To produce a standalone
document (e.g.~a valid HTML file including
\texttt{\textless{}head\textgreater{}} and
\texttt{\textless{}body\textgreater{}}), use the \texttt{-s} or
\texttt{-\/-standalone} flag:

\begin{verbatim}
pandoc -s -o output.html input.txt
\end{verbatim}

For more information on how standalone documents are produced, see
\hyperref[templates]{Templates} below.

If multiple input files are given, pandoc will concatenate them all
(with blank lines between them) before parsing. (Use
\texttt{-\/-file-scope} to parse files individually.)

\subsection{Specifying formats}\label{specifying-formats}

The format of the input and output can be specified explicitly using
command-line options. The input format can be specified using the
\texttt{-f/-\/-from} option, the output format using the
\texttt{-t/-\/-to} option. Thus, to convert \texttt{hello.txt} from
Markdown to LaTeX, you could type:

\begin{verbatim}
pandoc -f markdown -t latex hello.txt
\end{verbatim}

To convert \texttt{hello.html} from HTML to Markdown:

\begin{verbatim}
pandoc -f html -t markdown hello.html
\end{verbatim}

Supported input and output formats are listed below under
\hyperref[options]{Options} (see \texttt{-f} for input formats and
\texttt{-t} for output formats). You can also use
\texttt{pandoc\ -\/-list-input-formats} and
\texttt{pandoc\ -\/-list-output-formats} to print lists of supported
formats.

If the input or output format is not specified explicitly, pandoc will
attempt to guess it from the extensions of the filenames. Thus, for
example,

\begin{verbatim}
pandoc -o hello.tex hello.txt
\end{verbatim}

will convert \texttt{hello.txt} from Markdown to LaTeX. If no output
file is specified (so that output goes to \emph{stdout}), or if the
output file's extension is unknown, the output format will default to
HTML. If no input file is specified (so that input comes from
\emph{stdin}), or if the input files' extensions are unknown, the input
format will be assumed to be Markdown.

\subsection{Character encoding}\label{character-encoding}

Pandoc uses the UTF-8 character encoding for both input and output. If
your local character encoding is not UTF-8, you should pipe input and
output through
\href{https://www.gnu.org/software/libiconv/}{\texttt{iconv}}:

\begin{verbatim}
iconv -t utf-8 input.txt | pandoc | iconv -f utf-8
\end{verbatim}

Note that in some output formats (such as HTML, LaTeX, ConTeXt, RTF,
OPML, DocBook, and Texinfo), information about the character encoding is
included in the document header, which will only be included if you use
the \texttt{-s/-\/-standalone} option.

\subsection{Creating a PDF}\label{creating-a-pdf}

To produce a PDF, specify an output file with a \texttt{.pdf} extension:

\begin{verbatim}
pandoc test.txt -o test.pdf
\end{verbatim}

By default, pandoc will use LaTeX to create the PDF, which requires that
a LaTeX engine be installed (see \texttt{-\/-pdf-engine} below).
Alternatively, pandoc can use ConTeXt, roff ms, or HTML as an
intermediate format. To do this, specify an output file with a
\texttt{.pdf} extension, as before, but add the \texttt{-\/-pdf-engine}
option or \texttt{-t\ context}, \texttt{-t\ html}, or \texttt{-t\ ms} to
the command line. The tool used to generate the PDF from the
intermediate format may be specified using \texttt{-\/-pdf-engine}.

You can control the PDF style using variables, depending on the
intermediate format used: see \hyperref[variables-for-latex]{variables
for LaTeX}, \hyperref[variables-for-context]{variables for ConTeXt},
\hyperref[variables-for-wkhtmltopdf]{variables for
\texttt{wkhtmltopdf}}, \hyperref[variables-for-ms]{variables for ms}.
When HTML is used as an intermediate format, the output can be styled
using \texttt{-\/-css}.

To debug the PDF creation, it can be useful to look at the intermediate
representation: instead of \texttt{-o\ test.pdf}, use for example
\texttt{-s\ -o\ test.tex} to output the generated LaTeX. You can then
test it with \texttt{pdflatex\ test.tex}.

When using LaTeX, the following packages need to be available (they are
included with all recent versions of
\href{https://www.tug.org/texlive/}{TeX Live}):
\href{https://ctan.org/pkg/amsfonts}{\texttt{amsfonts}},
\href{https://ctan.org/pkg/amsmath}{\texttt{amsmath}},
\href{https://ctan.org/pkg/lm}{\texttt{lm}},
\href{https://ctan.org/pkg/unicode-math}{\texttt{unicode-math}},
\href{https://ctan.org/pkg/iftex}{\texttt{iftex}},
\href{https://ctan.org/pkg/listings}{\texttt{listings}} (if the
\texttt{-\/-listings} option is used),
\href{https://ctan.org/pkg/fancyvrb}{\texttt{fancyvrb}},
\href{https://ctan.org/pkg/longtable}{\texttt{longtable}},
\href{https://ctan.org/pkg/booktabs}{\texttt{booktabs}},
\href{https://ctan.org/pkg/graphicx}{\texttt{graphicx}} (if the document
contains images),
\href{https://ctan.org/pkg/hyperref}{\texttt{hyperref}},
\href{https://ctan.org/pkg/xcolor}{\texttt{xcolor}},
\href{https://ctan.org/pkg/ulem}{\texttt{ulem}},
\href{https://ctan.org/pkg/geometry}{\texttt{geometry}} (with the
\texttt{geometry} variable set),
\href{https://ctan.org/pkg/setspace}{\texttt{setspace}} (with
\texttt{linestretch}), and
\href{https://ctan.org/pkg/babel}{\texttt{babel}} (with \texttt{lang}).
If \texttt{CJKmainfont} is set,
\href{https://ctan.org/pkg/xecjk}{\texttt{xeCJK}} is needed. The use of
\texttt{xelatex} or \texttt{lualatex} as the PDF engine requires
\href{https://ctan.org/pkg/fontspec}{\texttt{fontspec}}.
\texttt{lualatex} uses
\href{https://ctan.org/pkg/selnolig}{\texttt{selnolig}}.
\texttt{xelatex} uses \href{https://ctan.org/pkg/bidi}{\texttt{bidi}}
(with the \texttt{dir} variable set). If the \texttt{mathspec} variable
is set, \texttt{xelatex} will use
\href{https://ctan.org/pkg/mathspec}{\texttt{mathspec}} instead of
\href{https://ctan.org/pkg/unicode-math}{\texttt{unicode-math}}. The
\href{https://ctan.org/pkg/upquote}{\texttt{upquote}} and
\href{https://ctan.org/pkg/microtype}{\texttt{microtype}} packages are
used if available, and
\href{https://ctan.org/pkg/csquotes}{\texttt{csquotes}} will be used for
\hyperref[typography]{typography} if the \texttt{csquotes} variable or
metadata field is set to a true value. The
\href{https://ctan.org/pkg/natbib}{\texttt{natbib}},
\href{https://ctan.org/pkg/biblatex}{\texttt{biblatex}},
\href{https://ctan.org/pkg/bibtex}{\texttt{bibtex}}, and
\href{https://ctan.org/pkg/biber}{\texttt{biber}} packages can
optionally be used for \hyperref[citation-rendering]{citation
rendering}. The following packages will be used to improve output
quality if present, but pandoc does not require them to be present:
\href{https://ctan.org/pkg/upquote}{\texttt{upquote}} (for straight
quotes in verbatim environments),
\href{https://ctan.org/pkg/microtype}{\texttt{microtype}} (for better
spacing adjustments),
\href{https://ctan.org/pkg/parskip}{\texttt{parskip}} (for better
inter-paragraph spaces), \href{https://ctan.org/pkg/xurl}{\texttt{xurl}}
(for better line breaks in URLs),
\href{https://ctan.org/pkg/bookmark}{\texttt{bookmark}} (for better PDF
bookmarks), and
\href{https://ctan.org/pkg/footnotehyper}{\texttt{footnotehyper}} or
\href{https://ctan.org/pkg/footnote}{\texttt{footnote}} (to allow
footnotes in tables).

\subsection{Reading from the Web}\label{reading-from-the-web}

Instead of an input file, an absolute URI may be given. In this case
pandoc will fetch the content using HTTP:

\begin{verbatim}
pandoc -f html -t markdown https://www.fsf.org
\end{verbatim}

It is possible to supply a custom User-Agent string or other header when
requesting a document from a URL:

\begin{verbatim}
pandoc -f html -t markdown --request-header User-Agent:"Mozilla/5.0" \
  https://www.fsf.org
\end{verbatim}

\section{Options}\label{options}

\subsection{General options}\label{general-options}

\begin{description}
\item[\texttt{-f} \emph{FORMAT}, \texttt{-r} \emph{FORMAT},
\texttt{-\/-from=}\emph{FORMAT}, \texttt{-\/-read=}\emph{FORMAT}]
Specify input format. \emph{FORMAT} can be:

\phantomsection\label{input-formats}
\begin{itemize}
\tightlist
\item
  \texttt{bibtex} (\href{https://ctan.org/pkg/bibtex}{BibTeX}
  bibliography)
\item
  \texttt{biblatex} (\href{https://ctan.org/pkg/biblatex}{BibLaTeX}
  bibliography)
\item
  \texttt{commonmark} (\href{https://commonmark.org}{CommonMark}
  Markdown)
\item
  \texttt{commonmark\_x} (\href{https://commonmark.org}{CommonMark}
  Markdown with extensions)
\item
  \texttt{creole}
  (\href{http://www.wikicreole.org/wiki/Creole1.0}{Creole 1.0})
\item
  \texttt{csljson}
  (\href{https://citeproc-js.readthedocs.io/en/latest/csl-json/markup.html}{CSL
  JSON} bibliography)
\item
  \texttt{csv} (\href{https://tools.ietf.org/html/rfc4180}{CSV} table)
\item
  \texttt{tsv}
  (\href{https://www.iana.org/assignments/media-types/text/tab-separated-values}{TSV}
  table)
\item
  \texttt{docbook} (\href{https://docbook.org}{DocBook})
\item
  \texttt{docx}
  (\href{https://en.wikipedia.org/wiki/Office_Open_XML}{Word docx})
\item
  \texttt{dokuwiki} (\href{https://www.dokuwiki.org/dokuwiki}{DokuWiki
  markup})
\item
  \texttt{endnotexml}
  (\href{https://support.clarivate.com/Endnote/s/article/EndNote-XML-Document-Type-Definition}{EndNote
  XML bibliography})
\item
  \texttt{epub} (\href{http://idpf.org/epub}{EPUB})
\item
  \texttt{fb2}
  (\href{http://www.fictionbook.org/index.php/Eng:XML_Schema_Fictionbook_2.1}{FictionBook2}
  e-book)
\item
  \texttt{gfm}
  (\href{https://help.github.com/articles/github-flavored-markdown/}{GitHub-Flavored
  Markdown}), or the deprecated and less accurate
  \texttt{markdown\_github}; use
  \hyperref[markdown-variants]{\texttt{markdown\_github}} only if you
  need extensions not supported in
  \hyperref[markdown-variants]{\texttt{gfm}}.
\item
  \texttt{haddock}
  (\href{https://www.haskell.org/haddock/doc/html/ch03s08.html}{Haddock
  markup})
\item
  \texttt{html} (\href{https://www.w3.org/html/}{HTML})
\item
  \texttt{ipynb}
  (\href{https://nbformat.readthedocs.io/en/latest/}{Jupyter notebook})
\item
  \texttt{jats} (\href{https://jats.nlm.nih.gov}{JATS} XML)
\item
  \texttt{jira}
  (\href{https://jira.atlassian.com/secure/WikiRendererHelpAction.jspa?section=all}{Jira}/Confluence
  wiki markup)
\item
  \texttt{json} (JSON version of native AST)
\item
  \texttt{latex} (\href{https://www.latex-project.org/}{LaTeX})
\item
  \texttt{markdown} (\hyperref[pandocs-markdown]{Pandoc's Markdown})
\item
  \texttt{markdown\_mmd}
  (\href{https://fletcherpenney.net/multimarkdown/}{MultiMarkdown})
\item
  \texttt{markdown\_phpextra}
  (\href{https://michelf.ca/projects/php-markdown/extra/}{PHP Markdown
  Extra})
\item
  \texttt{markdown\_strict} (original unextended
  \href{https://daringfireball.net/projects/markdown/}{Markdown})
\item
  \texttt{mediawiki}
  (\href{https://www.mediawiki.org/wiki/Help:Formatting}{MediaWiki
  markup})
\item
  \texttt{man} (\href{https://man.cx/groff_man(7)}{roff man})
\item
  \texttt{muse} (\href{https://amusewiki.org/library/manual}{Muse})
\item
  \texttt{native} (native Haskell)
\item
  \texttt{odt} (\href{https://en.wikipedia.org/wiki/OpenDocument}{ODT})
\item
  \texttt{opml} (\href{http://dev.opml.org/spec2.html}{OPML})
\item
  \texttt{org} (\href{https://orgmode.org}{Emacs Org mode})
\item
  \texttt{ris}
  (\href{https://en.wikipedia.org/wiki/RIS_(file_format)}{RIS}
  bibliography)
\item
  \texttt{rtf}
  (\href{https://en.wikipedia.org/wiki/Rich_Text_Format}{Rich Text
  Format})
\item
  \texttt{rst}
  (\href{https://docutils.sourceforge.io/docs/ref/rst/introduction.html}{reStructuredText})
\item
  \texttt{t2t} (\href{https://txt2tags.org}{txt2tags})
\item
  \texttt{textile} (\href{https://www.promptworks.com/textile}{Textile})
\item
  \texttt{tikiwiki}
  (\href{https://doc.tiki.org/Wiki-Syntax-Text\#The_Markup_Language_Wiki-Syntax}{TikiWiki
  markup})
\item
  \texttt{twiki}
  (\href{https://twiki.org/cgi-bin/view/TWiki/TextFormattingRules}{TWiki
  markup})
\item
  \texttt{vimwiki} (\href{https://vimwiki.github.io}{Vimwiki})
\item
  the path of a custom Lua reader, see
  \hyperref[custom-readers-and-writers]{Custom readers and writers}
  below
\end{itemize}

Extensions can be individually enabled or disabled by appending
\texttt{+EXTENSION} or \texttt{-EXTENSION} to the format name. See
\hyperref[extensions]{Extensions} below, for a list of extensions and
their names. See \texttt{-\/-list-input-formats} and
\texttt{-\/-list-extensions}, below.
\item[\texttt{-t} \emph{FORMAT}, \texttt{-w} \emph{FORMAT},
\texttt{-\/-to=}\emph{FORMAT}, \texttt{-\/-write=}\emph{FORMAT}]
Specify output format. \emph{FORMAT} can be:

\phantomsection\label{output-formats}
\begin{itemize}
\tightlist
\item
  \texttt{asciidoc}
  (\href{https://www.methods.co.nz/asciidoc/}{AsciiDoc}) or
  \texttt{asciidoctor} (\href{https://asciidoctor.org/}{AsciiDoctor})
\item
  \texttt{beamer} (\href{https://ctan.org/pkg/beamer}{LaTeX beamer}
  slide show)
\item
  \texttt{bibtex} (\href{https://ctan.org/pkg/bibtex}{BibTeX}
  bibliography)
\item
  \texttt{biblatex} (\href{https://ctan.org/pkg/biblatex}{BibLaTeX}
  bibliography)
\item
  \texttt{commonmark} (\href{https://commonmark.org}{CommonMark}
  Markdown)
\item
  \texttt{commonmark\_x} (\href{https://commonmark.org}{CommonMark}
  Markdown with extensions)
\item
  \texttt{context} (\href{https://www.contextgarden.net/}{ConTeXt})
\item
  \texttt{csljson}
  (\href{https://citeproc-js.readthedocs.io/en/latest/csl-json/markup.html}{CSL
  JSON} bibliography)
\item
  \texttt{docbook} or \texttt{docbook4}
  (\href{https://docbook.org}{DocBook} 4)
\item
  \texttt{docbook5} (DocBook 5)
\item
  \texttt{docx}
  (\href{https://en.wikipedia.org/wiki/Office_Open_XML}{Word docx})
\item
  \texttt{dokuwiki} (\href{https://www.dokuwiki.org/dokuwiki}{DokuWiki
  markup})
\item
  \texttt{epub} or \texttt{epub3} (\href{http://idpf.org/epub}{EPUB} v3
  book)
\item
  \texttt{epub2} (EPUB v2)
\item
  \texttt{fb2}
  (\href{http://www.fictionbook.org/index.php/Eng:XML_Schema_Fictionbook_2.1}{FictionBook2}
  e-book)
\item
  \texttt{gfm}
  (\href{https://help.github.com/articles/github-flavored-markdown/}{GitHub-Flavored
  Markdown}), or the deprecated and less accurate
  \texttt{markdown\_github}; use
  \hyperref[markdown-variants]{\texttt{markdown\_github}} only if you
  need extensions not supported in
  \hyperref[markdown-variants]{\texttt{gfm}}.
\item
  \texttt{haddock}
  (\href{https://www.haskell.org/haddock/doc/html/ch03s08.html}{Haddock
  markup})
\item
  \texttt{html} or \texttt{html5}
  (\href{https://www.w3.org/html/}{HTML},
  i.e.~\href{https://html.spec.whatwg.org/}{HTML5}/XHTML
  \href{https://www.w3.org/TR/html-polyglot/}{polyglot markup})
\item
  \texttt{html4} (\href{https://www.w3.org/TR/xhtml1/}{XHTML} 1.0
  Transitional)
\item
  \texttt{icml}
  (\href{https://wwwimages.adobe.com/www.adobe.com/content/dam/acom/en/devnet/indesign/sdk/cs6/idml/idml-cookbook.pdf}{InDesign
  ICML})
\item
  \texttt{ipynb}
  (\href{https://nbformat.readthedocs.io/en/latest/}{Jupyter notebook})
\item
  \texttt{jats\_archiving} (\href{https://jats.nlm.nih.gov}{JATS} XML,
  Archiving and Interchange Tag Set)
\item
  \texttt{jats\_articleauthoring} (\href{https://jats.nlm.nih.gov}{JATS}
  XML, Article Authoring Tag Set)
\item
  \texttt{jats\_publishing} (\href{https://jats.nlm.nih.gov}{JATS} XML,
  Journal Publishing Tag Set)
\item
  \texttt{jats} (alias for \texttt{jats\_archiving})
\item
  \texttt{jira}
  (\href{https://jira.atlassian.com/secure/WikiRendererHelpAction.jspa?section=all}{Jira}/Confluence
  wiki markup)
\item
  \texttt{json} (JSON version of native AST)
\item
  \texttt{latex} (\href{https://www.latex-project.org/}{LaTeX})
\item
  \texttt{man} (\href{https://man.cx/groff_man(7)}{roff man})
\item
  \texttt{markdown} (\hyperref[pandocs-markdown]{Pandoc's Markdown})
\item
  \texttt{markdown\_mmd}
  (\href{https://fletcherpenney.net/multimarkdown/}{MultiMarkdown})
\item
  \texttt{markdown\_phpextra}
  (\href{https://michelf.ca/projects/php-markdown/extra/}{PHP Markdown
  Extra})
\item
  \texttt{markdown\_strict} (original unextended
  \href{https://daringfireball.net/projects/markdown/}{Markdown})
\item
  \texttt{markua} (\href{https://leanpub.com/markua/read}{Markua})
\item
  \texttt{mediawiki}
  (\href{https://www.mediawiki.org/wiki/Help:Formatting}{MediaWiki
  markup})
\item
  \texttt{ms} (\href{https://man.cx/groff_ms(7)}{roff ms})
\item
  \texttt{muse} (\href{https://amusewiki.org/library/manual}{Muse})
\item
  \texttt{native} (native Haskell)
\item
  \texttt{odt}
  (\href{https://en.wikipedia.org/wiki/OpenDocument}{OpenOffice text
  document})
\item
  \texttt{opml} (\href{http://dev.opml.org/spec2.html}{OPML})
\item
  \texttt{opendocument}
  (\href{http://opendocument.xml.org}{OpenDocument})
\item
  \texttt{org} (\href{https://orgmode.org}{Emacs Org mode})
\item
  \texttt{pdf} (\href{https://www.adobe.com/pdf/}{PDF})
\item
  \texttt{plain} (plain text)
\item
  \texttt{pptx}
  (\href{https://en.wikipedia.org/wiki/Microsoft_PowerPoint}{PowerPoint}
  slide show)
\item
  \texttt{rst}
  (\href{https://docutils.sourceforge.io/docs/ref/rst/introduction.html}{reStructuredText})
\item
  \texttt{rtf}
  (\href{https://en.wikipedia.org/wiki/Rich_Text_Format}{Rich Text
  Format})
\item
  \texttt{texinfo} (\href{https://www.gnu.org/software/texinfo/}{GNU
  Texinfo})
\item
  \texttt{textile} (\href{https://www.promptworks.com/textile}{Textile})
\item
  \texttt{slideous}
  (\href{https://goessner.net/articles/slideous/}{Slideous} HTML and
  JavaScript slide show)
\item
  \texttt{slidy} (\href{https://www.w3.org/Talks/Tools/Slidy2/}{Slidy}
  HTML and JavaScript slide show)
\item
  \texttt{dzslides} (\href{https://paulrouget.com/dzslides/}{DZSlides}
  HTML5 + JavaScript slide show)
\item
  \texttt{revealjs} (\href{https://revealjs.com/}{reveal.js} HTML5 +
  JavaScript slide show)
\item
  \texttt{s5} (\href{https://meyerweb.com/eric/tools/s5/}{S5} HTML and
  JavaScript slide show)
\item
  \texttt{tei} (\href{https://github.com/TEIC/TEI-Simple}{TEI Simple})
\item
  \texttt{xwiki}
  (\href{https://www.xwiki.org/xwiki/bin/view/Documentation/UserGuide/Features/XWikiSyntax/}{XWiki
  markup})
\item
  \texttt{zimwiki}
  (\href{https://zim-wiki.org/manual/Help/Wiki_Syntax.html}{ZimWiki
  markup})
\item
  the path of a custom Lua writer, see
  \hyperref[custom-readers-and-writers]{Custom readers and writers}
  below
\end{itemize}

Note that \texttt{odt}, \texttt{docx}, \texttt{epub}, and \texttt{pdf}
output will not be directed to \emph{stdout} unless forced with
\texttt{-o\ -}.

Extensions can be individually enabled or disabled by appending
\texttt{+EXTENSION} or \texttt{-EXTENSION} to the format name. See
\hyperref[extensions]{Extensions} below, for a list of extensions and
their names. See \texttt{-\/-list-output-formats} and
\texttt{-\/-list-extensions}, below.
\item[\texttt{-o} \emph{FILE}, \texttt{-\/-output=}\emph{FILE}]
Write output to \emph{FILE} instead of \emph{stdout}. If \emph{FILE} is
\texttt{-}, output will go to \emph{stdout}, even if a non-textual
format (\texttt{docx}, \texttt{odt}, \texttt{epub2}, \texttt{epub3}) is
specified.
\item[\texttt{-\/-data-dir=}\emph{DIRECTORY}]
Specify the user data directory to search for pandoc data files. If this
option is not specified, the default user data directory will be used.
On *nix and macOS systems this will be the \texttt{pandoc} subdirectory
of the XDG data directory (by default, \texttt{\$HOME/.local/share},
overridable by setting the \texttt{XDG\_DATA\_HOME} environment
variable). If that directory does not exist and \texttt{\$HOME/.pandoc}
exists, it will be used (for backwards compatibility). On Windows the
default user data directory is
\texttt{C:\textbackslash{}Users\textbackslash{}USERNAME\textbackslash{}AppData\textbackslash{}Roaming\textbackslash{}pandoc}.
You can find the default user data directory on your system by looking
at the output of \texttt{pandoc\ -\/-version}. Data files placed in this
directory (for example, \texttt{reference.odt}, \texttt{reference.docx},
\texttt{epub.css}, \texttt{templates}) will override pandoc's normal
defaults.
\item[\texttt{-d} \emph{FILE}, \texttt{-\/-defaults=}\emph{FILE}]
Specify a set of default option settings. \emph{FILE} is a YAML file
whose fields correspond to command-line option settings. All options for
document conversion, including input and output files, can be set using
a defaults file. The file will be searched for first in the working
directory, and then in the \texttt{defaults} subdirectory of the user
data directory (see \texttt{-\/-data-dir}). The \texttt{.yaml} extension
may be omitted. See the section \hyperref[defaults-files]{Defaults
files} for more information on the file format. Settings from the
defaults file may be overridden or extended by subsequent options on the
command line.
\item[\texttt{-\/-bash-completion}]
Generate a bash completion script. To enable bash completion with
pandoc, add this to your \texttt{.bashrc}:

\begin{verbatim}
eval "$(pandoc --bash-completion)"
\end{verbatim}
\item[\texttt{-\/-verbose}]
Give verbose debugging output.
\item[\texttt{-\/-quiet}]
Suppress warning messages.
\item[\texttt{-\/-fail-if-warnings}]
Exit with error status if there are any warnings.
\item[\texttt{-\/-log=}\emph{FILE}]
Write log messages in machine-readable JSON format to \emph{FILE}. All
messages above DEBUG level will be written, regardless of verbosity
settings (\texttt{-\/-verbose}, \texttt{-\/-quiet}).
\item[\texttt{-\/-list-input-formats}]
List supported input formats, one per line.
\item[\texttt{-\/-list-output-formats}]
List supported output formats, one per line.
\item[\texttt{-\/-list-extensions}{[}\texttt{=}\emph{FORMAT}{]}]
List supported extensions for \emph{FORMAT}, one per line, preceded by a
\texttt{+} or \texttt{-} indicating whether it is enabled by default in
\emph{FORMAT}. If \emph{FORMAT} is not specified, defaults for pandoc's
Markdown are given.
\item[\texttt{-\/-list-highlight-languages}]
List supported languages for syntax highlighting, one per line.
\item[\texttt{-\/-list-highlight-styles}]
List supported styles for syntax highlighting, one per line. See
\texttt{-\/-highlight-style}.
\item[\texttt{-v}, \texttt{-\/-version}]
Print version.
\item[\texttt{-h}, \texttt{-\/-help}]
Show usage message.
\end{description}

\subsection{Reader options}\label{reader-options}

\begin{description}
\item[\texttt{-\/-shift-heading-level-by=}\emph{NUMBER}]
Shift heading levels by a positive or negative integer. For example,
with \texttt{-\/-shift-heading-level-by=-1}, level 2 headings become
level 1 headings, and level 3 headings become level 2 headings. Headings
cannot have a level less than 1, so a heading that would be shifted
below level 1 becomes a regular paragraph. Exception: with a shift of
-N, a level-N heading at the beginning of the document replaces the
metadata title. \texttt{-\/-shift-heading-level-by=-1} is a good choice
when converting HTML or Markdown documents that use an initial level-1
heading for the document title and level-2+ headings for sections.
\texttt{-\/-shift-heading-level-by=1} may be a good choice for
converting Markdown documents that use level-1 headings for sections to
HTML, since pandoc uses a level-1 heading to render the document title.
\item[\texttt{-\/-base-header-level=}\emph{NUMBER}]
\emph{Deprecated. Use \texttt{-\/-shift-heading-level-by}=X instead,
where X = NUMBER - 1.} Specify the base level for headings (defaults to
1).
\item[\texttt{-\/-strip-empty-paragraphs}]
\emph{Deprecated. Use the \texttt{+empty\_paragraphs} extension
instead.} Ignore paragraphs with no content. This option is useful for
converting word processing documents where users have used empty
paragraphs to create inter-paragraph space.
\item[\texttt{-\/-indented-code-classes=}\emph{CLASSES}]
Specify classes to use for indented code blocks--for example,
\texttt{perl,numberLines} or \texttt{haskell}. Multiple classes may be
separated by spaces or commas.
\item[\texttt{-\/-default-image-extension=}\emph{EXTENSION}]
Specify a default extension to use when image paths/URLs have no
extension. This allows you to use the same source for formats that
require different kinds of images. Currently this option only affects
the Markdown and LaTeX readers.
\item[\texttt{-\/-file-scope}]
Parse each file individually before combining for multifile documents.
This will allow footnotes in different files with the same identifiers
to work as expected. If this option is set, footnotes and links will not
work across files. Reading binary files (docx, odt, epub) implies
\texttt{-\/-file-scope}.
\item[\texttt{-F} \emph{PROGRAM}, \texttt{-\/-filter=}\emph{PROGRAM}]
Specify an executable to be used as a filter transforming the pandoc AST
after the input is parsed and before the output is written. The
executable should read JSON from stdin and write JSON to stdout. The
JSON must be formatted like pandoc's own JSON input and output. The name
of the output format will be passed to the filter as the first argument.
Hence,

\begin{verbatim}
pandoc --filter ./caps.py -t latex
\end{verbatim}

is equivalent to

\begin{verbatim}
pandoc -t json | ./caps.py latex | pandoc -f json -t latex
\end{verbatim}

The latter form may be useful for debugging filters.

Filters may be written in any language. \texttt{Text.Pandoc.JSON}
exports \texttt{toJSONFilter} to facilitate writing filters in Haskell.
Those who would prefer to write filters in python can use the module
\href{https://github.com/jgm/pandocfilters}{\texttt{pandocfilters}},
installable from PyPI. There are also pandoc filter libraries in
\href{https://github.com/vinai/pandocfilters-php}{PHP},
\href{https://metacpan.org/pod/Pandoc::Filter}{perl}, and
\href{https://github.com/mvhenderson/pandoc-filter-node}{JavaScript/node.js}.

In order of preference, pandoc will look for filters in

\begin{enumerate}
\def\labelenumi{\arabic{enumi}.}
\item
  a specified full or relative path (executable or non-executable),
\item
  \texttt{\$DATADIR/filters} (executable or non-executable) where
  \texttt{\$DATADIR} is the user data directory (see
  \texttt{-\/-data-dir}, above),
\item
  \texttt{\$PATH} (executable only).
\end{enumerate}

Filters, Lua-filters, and citeproc processing are applied in the order
specified on the command line.
\item[\texttt{-L} \emph{SCRIPT}, \texttt{-\/-lua-filter=}\emph{SCRIPT}]
Transform the document in a similar fashion as JSON filters (see
\texttt{-\/-filter}), but use pandoc's built-in Lua filtering system.
The given Lua script is expected to return a list of Lua filters which
will be applied in order. Each Lua filter must contain
element-transforming functions indexed by the name of the AST element on
which the filter function should be applied.

The \texttt{pandoc} Lua module provides helper functions for element
creation. It is always loaded into the script's Lua environment.

See the \href{https://pandoc.org/lua-filters.html}{Lua filters
documentation} for further details.

In order of preference, pandoc will look for Lua filters in

\begin{enumerate}
\def\labelenumi{\arabic{enumi}.}
\item
  a specified full or relative path,
\item
  \texttt{\$DATADIR/filters} where \texttt{\$DATADIR} is the user data
  directory (see \texttt{-\/-data-dir}, above).
\end{enumerate}

Filters, Lua filters, and citeproc processing are applied in the order
specified on the command line.
\item[\texttt{-M} \emph{KEY}{[}\texttt{=}\emph{VAL}{]},
\texttt{-\/-metadata=}\emph{KEY}{[}\texttt{:}\emph{VAL}{]}]
Set the metadata field \emph{KEY} to the value \emph{VAL}. A value
specified on the command line overrides a value specified in the
document using \hyperref[extension-yaml_metadata_block]{YAML metadata
blocks}. Values will be parsed as YAML boolean or string values. If no
value is specified, the value will be treated as Boolean true. Like
\texttt{-\/-variable}, \texttt{-\/-metadata} causes template variables
to be set. But unlike \texttt{-\/-variable}, \texttt{-\/-metadata}
affects the metadata of the underlying document (which is accessible
from filters and may be printed in some output formats) and metadata
values will be escaped when inserted into the template.
\item[\texttt{-\/-metadata-file=}\emph{FILE}]
Read metadata from the supplied YAML (or JSON) file. This option can be
used with every input format, but string scalars in the YAML file will
always be parsed as Markdown. (If the input format is Markdown or a
Markdown variant, then the same variant will be used to parse the
metadata file; if it is a non-Markdown format, pandoc's default Markdown
extensions will be used.) This option can be used repeatedly to include
multiple metadata files; values in files specified later on the command
line will be preferred over those specified in earlier files. Metadata
values specified inside the document, or by using \texttt{-M}, overwrite
values specified with this option. The file will be searched for first
in the working directory, and then in the \texttt{metadata} subdirectory
of the user data directory (see \texttt{-\/-data-dir}).
\item[\texttt{-p}, \texttt{-\/-preserve-tabs}]
Preserve tabs instead of converting them to spaces. (By default, pandoc
converts tabs to spaces before parsing its input.) Note that this will
only affect tabs in literal code spans and code blocks. Tabs in regular
text are always treated as spaces.
\item[\texttt{-\/-tab-stop=}\emph{NUMBER}]
Specify the number of spaces per tab (default is 4).
\item[\texttt{-\/-track-changes=accept}\textbar{}\texttt{reject}\textbar{}\texttt{all}]
Specifies what to do with insertions, deletions, and comments produced
by the MS Word ``Track Changes'' feature. \texttt{accept} (the default)
processes all the insertions and deletions. \texttt{reject} ignores
them. Both \texttt{accept} and \texttt{reject} ignore comments.
\texttt{all} includes all insertions, deletions, and comments, wrapped
in spans with \texttt{insertion}, \texttt{deletion},
\texttt{comment-start}, and \texttt{comment-end} classes, respectively.
The author and time of change is included. \texttt{all} is useful for
scripting: only accepting changes from a certain reviewer, say, or
before a certain date. If a paragraph is inserted or deleted,
\texttt{track-changes=all} produces a span with the class
\texttt{paragraph-insertion}/\texttt{paragraph-deletion} before the
affected paragraph break. This option only affects the docx reader.
\item[\texttt{-\/-extract-media=}\emph{DIR}]
Extract images and other media contained in or linked from the source
document to the path \emph{DIR}, creating it if necessary, and adjust
the images references in the document so they point to the extracted
files. Media are downloaded, read from the file system, or extracted
from a binary container (e.g.~docx), as needed. The original file paths
are used if they are relative paths not containing \texttt{..}.
Otherwise filenames are constructed from the SHA1 hash of the contents.
\item[\texttt{-\/-abbreviations=}\emph{FILE}]
Specifies a custom abbreviations file, with abbreviations one to a line.
If this option is not specified, pandoc will read the data file
\texttt{abbreviations} from the user data directory or fall back on a
system default. To see the system default, use
\texttt{pandoc\ -\/-print-default-data-file=abbreviations}. The only use
pandoc makes of this list is in the Markdown reader. Strings found in
this list will be followed by a nonbreaking space, and the period will
not produce sentence-ending space in formats like LaTeX. The strings may
not contain spaces.
\item[\texttt{-\/-trace}]
Print diagnostic output tracing parser progress to stderr. This option
is intended for use by developers in diagnosing performance issues.
\end{description}

\subsection{General writer options}\label{general-writer-options}

\begin{description}
\item[\texttt{-s}, \texttt{-\/-standalone}]
Produce output with an appropriate header and footer (e.g.~a standalone
HTML, LaTeX, TEI, or RTF file, not a fragment). This option is set
automatically for \texttt{pdf}, \texttt{epub}, \texttt{epub3},
\texttt{fb2}, \texttt{docx}, and \texttt{odt} output. For
\texttt{native} output, this option causes metadata to be included;
otherwise, metadata is suppressed.
\item[\texttt{-\/-template=}\emph{FILE}\textbar{}\emph{URL}]
Use the specified file as a custom template for the generated document.
Implies \texttt{-\/-standalone}. See \hyperref[templates]{Templates},
below, for a description of template syntax. If no extension is
specified, an extension corresponding to the writer will be added, so
that \texttt{-\/-template=special} looks for \texttt{special.html} for
HTML output. If the template is not found, pandoc will search for it in
the \texttt{templates} subdirectory of the user data directory (see
\texttt{-\/-data-dir}). If this option is not used, a default template
appropriate for the output format will be used (see
\texttt{-D/-\/-print-default-template}).
\item[\texttt{-V} \emph{KEY}{[}\texttt{=}\emph{VAL}{]},
\texttt{-\/-variable=}\emph{KEY}{[}\texttt{:}\emph{VAL}{]}]
Set the template variable \emph{KEY} to the value \emph{VAL} when
rendering the document in standalone mode. If no \emph{VAL} is
specified, the key will be given the value \texttt{true}.
\item[\texttt{-\/-sandbox}]
Run pandoc in a sandbox, limiting IO operations in readers and writers
to reading the files specified on the command line. Note that this
option does not limit IO operations by filters or in the production of
PDF documents. But it does offer security against, for example,
disclosure of files through the use of \texttt{include} directives.
Anyone using pandoc on untrusted user input should use this option.

Note: some readers and writers (e.g., \texttt{docx}) need access to data
files. If these are stored on the file system, then pandoc will not be
able to find them when run in \texttt{-\/-sandbox} mode and will raise
an error. For these applications, we recommend using a pandoc binary
compiled with the \texttt{embed\_data\_files} option, which causes the
data files to be baked into the binary instead of being stored on the
file system.
\item[\texttt{-D} \emph{FORMAT},
\texttt{-\/-print-default-template=}\emph{FORMAT}]
Print the system default template for an output \emph{FORMAT}. (See
\texttt{-t} for a list of possible \emph{FORMAT}s.) Templates in the
user data directory are ignored. This option may be used with
\texttt{-o}/\texttt{-\/-output} to redirect output to a file, but
\texttt{-o}/\texttt{-\/-output} must come before
\texttt{-\/-print-default-template} on the command line.

Note that some of the default templates use partials, for example
\texttt{styles.html}. To print the partials, use
\texttt{-\/-print-default-data-file}: for example,
\texttt{-\/-print-default-data-file=templates/styles.html}.
\item[\texttt{-\/-print-default-data-file=}\emph{FILE}]
Print a system default data file. Files in the user data directory are
ignored. This option may be used with \texttt{-o}/\texttt{-\/-output} to
redirect output to a file, but \texttt{-o}/\texttt{-\/-output} must come
before \texttt{-\/-print-default-data-file} on the command line.
\item[\texttt{-\/-eol=crlf}\textbar{}\texttt{lf}\textbar{}\texttt{native}]
Manually specify line endings: \texttt{crlf} (Windows), \texttt{lf}
(macOS/Linux/UNIX), or \texttt{native} (line endings appropriate to the
OS on which pandoc is being run). The default is \texttt{native}.
\item[\texttt{-\/-dpi}=\emph{NUMBER}]
Specify the default dpi (dots per inch) value for conversion from pixels
to inch/centimeters and vice versa. (Technically, the correct term would
be ppi: pixels per inch.) The default is 96dpi. When images contain
information about dpi internally, the encoded value is used instead of
the default specified by this option.
\item[\texttt{-\/-wrap=auto}\textbar{}\texttt{none}\textbar{}\texttt{preserve}]
Determine how text is wrapped in the output (the source code, not the
rendered version). With \texttt{auto} (the default), pandoc will attempt
to wrap lines to the column width specified by \texttt{-\/-columns}
(default 72). With \texttt{none}, pandoc will not wrap lines at all.
With \texttt{preserve}, pandoc will attempt to preserve the wrapping
from the source document (that is, where there are nonsemantic newlines
in the source, there will be nonsemantic newlines in the output as
well). In \texttt{ipynb} output, this option affects wrapping of the
contents of markdown cells.
\item[\texttt{-\/-columns=}\emph{NUMBER}]
Specify length of lines in characters. This affects text wrapping in the
generated source code (see \texttt{-\/-wrap}). It also affects
calculation of column widths for plain text tables (see
\hyperref[tables]{Tables} below).
\item[\texttt{-\/-toc}, \texttt{-\/-table-of-contents}]
Include an automatically generated table of contents (or, in the case of
\texttt{latex}, \texttt{context}, \texttt{docx}, \texttt{odt},
\texttt{opendocument}, \texttt{rst}, or \texttt{ms}, an instruction to
create one) in the output document. This option has no effect unless
\texttt{-s/-\/-standalone} is used, and it has no effect on
\texttt{man}, \texttt{docbook4}, \texttt{docbook5}, or \texttt{jats}
output.

Note that if you are producing a PDF via \texttt{ms}, the table of
contents will appear at the beginning of the document, before the title.
If you would prefer it to be at the end of the document, use the option
\texttt{-\/-pdf-engine-opt=-\/-no-toc-relocation}.
\item[\texttt{-\/-toc-depth=}\emph{NUMBER}]
Specify the number of section levels to include in the table of
contents. The default is 3 (which means that level-1, 2, and 3 headings
will be listed in the contents).
\item[\texttt{-\/-strip-comments}]
Strip out HTML comments in the Markdown or Textile source, rather than
passing them on to Markdown, Textile or HTML output as raw HTML. This
does not apply to HTML comments inside raw HTML blocks when the
\texttt{markdown\_in\_html\_blocks} extension is not set.
\item[\texttt{-\/-no-highlight}]
Disables syntax highlighting for code blocks and inlines, even when a
language attribute is given.
\item[\texttt{-\/-highlight-style=}\emph{STYLE}\textbar{}\emph{FILE}]
Specifies the coloring style to be used in highlighted source code.
Options are \texttt{pygments} (the default), \texttt{kate},
\texttt{monochrome}, \texttt{breezeDark}, \texttt{espresso},
\texttt{zenburn}, \texttt{haddock}, and \texttt{tango}. For more
information on syntax highlighting in pandoc, see
\hyperref[syntax-highlighting]{Syntax highlighting}, below. See also
\texttt{-\/-list-highlight-styles}.

Instead of a \emph{STYLE} name, a JSON file with extension
\texttt{.theme} may be supplied. This will be parsed as a KDE syntax
highlighting theme and (if valid) used as the highlighting style.

To generate the JSON version of an existing style, use
\texttt{-\/-print-highlight-style}.
\item[\texttt{-\/-print-highlight-style=}\emph{STYLE}\textbar{}\emph{FILE}]
Prints a JSON version of a highlighting style, which can be modified,
saved with a \texttt{.theme} extension, and used with
\texttt{-\/-highlight-style}. This option may be used with
\texttt{-o}/\texttt{-\/-output} to redirect output to a file, but
\texttt{-o}/\texttt{-\/-output} must come before
\texttt{-\/-print-highlight-style} on the command line.
\item[\texttt{-\/-syntax-definition=}\emph{FILE}]
Instructs pandoc to load a KDE XML syntax definition file, which will be
used for syntax highlighting of appropriately marked code blocks. This
can be used to add support for new languages or to use altered syntax
definitions for existing languages. This option may be repeated to add
multiple syntax definitions.
\item[\texttt{-H} \emph{FILE},
\texttt{-\/-include-in-header=}\emph{FILE}\textbar{}\emph{URL}]
Include contents of \emph{FILE}, verbatim, at the end of the header.
This can be used, for example, to include special CSS or JavaScript in
HTML documents. This option can be used repeatedly to include multiple
files in the header. They will be included in the order specified.
Implies \texttt{-\/-standalone}.
\item[\texttt{-B} \emph{FILE},
\texttt{-\/-include-before-body=}\emph{FILE}\textbar{}\emph{URL}]
Include contents of \emph{FILE}, verbatim, at the beginning of the
document body (e.g.~after the \texttt{\textless{}body\textgreater{}} tag
in HTML, or the \texttt{\textbackslash{}begin\{document\}} command in
LaTeX). This can be used to include navigation bars or banners in HTML
documents. This option can be used repeatedly to include multiple files.
They will be included in the order specified. Implies
\texttt{-\/-standalone}.
\item[\texttt{-A} \emph{FILE},
\texttt{-\/-include-after-body=}\emph{FILE}\textbar{}\emph{URL}]
Include contents of \emph{FILE}, verbatim, at the end of the document
body (before the \texttt{\textless{}/body\textgreater{}} tag in HTML, or
the \texttt{\textbackslash{}end\{document\}} command in LaTeX). This
option can be used repeatedly to include multiple files. They will be
included in the order specified. Implies \texttt{-\/-standalone}.
\item[\texttt{-\/-resource-path=}\emph{SEARCHPATH}]
List of paths to search for images and other resources. The paths should
be separated by \texttt{:} on Linux, UNIX, and macOS systems, and by
\texttt{;} on Windows. If \texttt{-\/-resource-path} is not specified,
the default resource path is the working directory. Note that, if
\texttt{-\/-resource-path} is specified, the working directory must be
explicitly listed or it will not be searched. For example:
\texttt{-\/-resource-path=.:test} will search the working directory and
the \texttt{test} subdirectory, in that order. This option can be used
repeatedly. Search path components that come later on the command line
will be searched before those that come earlier, so
\texttt{-\/-resource-path\ foo:bar\ -\/-resource-path\ baz:bim} is
equivalent to \texttt{-\/-resource-path\ baz:bim:foo:bar}.
\item[\texttt{-\/-request-header=}\emph{NAME}\texttt{:}\emph{VAL}]
Set the request header \emph{NAME} to the value \emph{VAL} when making
HTTP requests (for example, when a URL is given on the command line, or
when resources used in a document must be downloaded). If you're behind
a proxy, you also need to set the environment variable
\texttt{http\_proxy} to \texttt{http://...}.
\item[\texttt{-\/-no-check-certificate}]
Disable the certificate verification to allow access to unsecure HTTP
resources (for example when the certificate is no longer valid or self
signed).
\end{description}

\subsection{Options affecting specific
writers}\label{options-affecting-specific-writers}

\begin{description}
\item[\texttt{-\/-self-contained}]
\emph{Deprecated synonym for
\texttt{-\/-embed-resources\ -\/-standalone}.}
\item[\texttt{-\/-embed-resources}]
Produce a standalone HTML file with no external dependencies, using
\texttt{data:} URIs to incorporate the contents of linked scripts,
stylesheets, images, and videos. The resulting file should be
``self-contained,'' in the sense that it needs no external files and no
net access to be displayed properly by a browser. This option works only
with HTML output formats, including \texttt{html4}, \texttt{html5},
\texttt{html+lhs}, \texttt{html5+lhs}, \texttt{s5}, \texttt{slidy},
\texttt{slideous}, \texttt{dzslides}, and \texttt{revealjs}. Scripts,
images, and stylesheets at absolute URLs will be downloaded; those at
relative URLs will be sought relative to the working directory (if the
first source file is local) or relative to the base URL (if the first
source file is remote). Elements with the attribute
\texttt{data-external="1"} will be left alone; the documents they link
to will not be incorporated in the document. Limitation: resources that
are loaded dynamically through JavaScript cannot be incorporated; as a
result, some advanced features (e.g.~zoom or speaker notes) may not work
in an offline ``self-contained'' \texttt{reveal.js} slide show.
\item[\texttt{-\/-html-q-tags}]
Use \texttt{\textless{}q\textgreater{}} tags for quotes in HTML. (This
option only has an effect if the \texttt{smart} extension is enabled for
the input format used.)
\item[\texttt{-\/-ascii}]
Use only ASCII characters in output. Currently supported for XML and
HTML formats (which use entities instead of UTF-8 when this option is
selected), CommonMark, gfm, and Markdown (which use entities), roff ms
(which use hexadecimal escapes), and to a limited degree LaTeX (which
uses standard commands for accented characters when possible). roff man
output uses ASCII by default.
\item[\texttt{-\/-reference-links}]
Use reference-style links, rather than inline links, in writing Markdown
or reStructuredText. By default inline links are used. The placement of
link references is affected by the \texttt{-\/-reference-location}
option.
\item[\texttt{-\/-reference-location=block}\textbar{}\texttt{section}\textbar{}\texttt{document}]
Specify whether footnotes (and references, if \texttt{reference-links}
is set) are placed at the end of the current (top-level) block, the
current section, or the document. The default is \texttt{document}.
Currently this option only affects the \texttt{markdown}, \texttt{muse},
\texttt{html}, \texttt{epub}, \texttt{slidy}, \texttt{s5},
\texttt{slideous}, \texttt{dzslides}, and \texttt{revealjs} writers.
\item[\texttt{-\/-markdown-headings=setext}\textbar{}\texttt{atx}]
Specify whether to use ATX-style (\texttt{\#}-prefixed) or Setext-style
(underlined) headings for level 1 and 2 headings in Markdown output.
(The default is \texttt{atx}.) ATX-style headings are always used for
levels 3+. This option also affects Markdown cells in \texttt{ipynb}
output.
\item[\texttt{-\/-atx-headers}]
\emph{Deprecated synonym for \texttt{-\/-markdown-headings=atx}.}
\item[\texttt{-\/-top-level-division=default}\textbar{}\texttt{section}\textbar{}\texttt{chapter}\textbar{}\texttt{part}]
Treat top-level headings as the given division type in LaTeX, ConTeXt,
DocBook, and TEI output. The hierarchy order is part, chapter, then
section; all headings are shifted such that the top-level heading
becomes the specified type. The default behavior is to determine the
best division type via heuristics: unless other conditions apply,
\texttt{section} is chosen. When the \texttt{documentclass} variable is
set to \texttt{report}, \texttt{book}, or \texttt{memoir} (unless the
\texttt{article} option is specified), \texttt{chapter} is implied as
the setting for this option. If \texttt{beamer} is the output format,
specifying either \texttt{chapter} or \texttt{part} will cause top-level
headings to become \texttt{\textbackslash{}part\{..\}}, while
second-level headings remain as their default type.
\item[\texttt{-N}, \texttt{-\/-number-sections}]
Number section headings in LaTeX, ConTeXt, HTML, Docx, ms, or EPUB
output. By default, sections are not numbered. Sections with class
\texttt{unnumbered} will never be numbered, even if
\texttt{-\/-number-sections} is specified.
\item[\texttt{-\/-number-offset=}\emph{NUMBER}{[}\texttt{,}\emph{NUMBER}\texttt{,}\emph{\ldots{}}{]}]
Offset for section headings in HTML output (ignored in other output
formats). The first number is added to the section number for top-level
headings, the second for second-level headings, and so on. So, for
example, if you want the first top-level heading in your document to be
numbered ``6'', specify \texttt{-\/-number-offset=5}. If your document
starts with a level-2 heading which you want to be numbered ``1.5'',
specify \texttt{-\/-number-offset=1,4}. Offsets are 0 by default.
Implies \texttt{-\/-number-sections}.
\item[\texttt{-\/-listings}]
Use the \href{https://ctan.org/pkg/listings}{\texttt{listings}} package
for LaTeX code blocks. The package does not support multi-byte encoding
for source code. To handle UTF-8 you would need to use a custom
template. This issue is fully documented here:
\href{https://en.wikibooks.org/wiki/LaTeX/Source_Code_Listings\#Encoding_issue}{Encoding
issue with the listings package}.
\item[\texttt{-i}, \texttt{-\/-incremental}]
Make list items in slide shows display incrementally (one by one). The
default is for lists to be displayed all at once.
\item[\texttt{-\/-slide-level=}\emph{NUMBER}]
Specifies that headings with the specified level create slides (for
\texttt{beamer}, \texttt{s5}, \texttt{slidy}, \texttt{slideous},
\texttt{dzslides}). Headings above this level in the hierarchy are used
to divide the slide show into sections; headings below this level create
subheads within a slide. Valid values are 0-6. If a slide level of 0 is
specified, slides will not be split automatically on headings, and
horizontal rules must be used to indicate slide boundaries. If a slide
level is not specified explicitly, the slide level will be set
automatically based on the contents of the document; see
\hyperref[structuring-the-slide-show]{Structuring the slide show}.
\item[\texttt{-\/-section-divs}]
Wrap sections in \texttt{\textless{}section\textgreater{}} tags (or
\texttt{\textless{}div\textgreater{}} tags for \texttt{html4}), and
attach identifiers to the enclosing
\texttt{\textless{}section\textgreater{}} (or
\texttt{\textless{}div\textgreater{}}) rather than the heading itself.
See \hyperref[heading-identifiers]{Heading identifiers}, below.
\item[\texttt{-\/-email-obfuscation=none}\textbar{}\texttt{javascript}\textbar{}\texttt{references}]
Specify a method for obfuscating \texttt{mailto:} links in HTML
documents. \texttt{none} leaves \texttt{mailto:} links as they are.
\texttt{javascript} obfuscates them using JavaScript.
\texttt{references} obfuscates them by printing their letters as decimal
or hexadecimal character references. The default is \texttt{none}.
\item[\texttt{-\/-id-prefix=}\emph{STRING}]
Specify a prefix to be added to all identifiers and internal links in
HTML and DocBook output, and to footnote numbers in Markdown and Haddock
output. This is useful for preventing duplicate identifiers when
generating fragments to be included in other pages.
\item[\texttt{-T} \emph{STRING},
\texttt{-\/-title-prefix=}\emph{STRING}]
Specify \emph{STRING} as a prefix at the beginning of the title that
appears in the HTML header (but not in the title as it appears at the
beginning of the HTML body). Implies \texttt{-\/-standalone}.
\item[\texttt{-c} \emph{URL}, \texttt{-\/-css=}\emph{URL}]
Link to a CSS style sheet. This option can be used repeatedly to include
multiple files. They will be included in the order specified.

A stylesheet is required for generating EPUB. If none is provided using
this option (or the \texttt{css} or \texttt{stylesheet} metadata
fields), pandoc will look for a file \texttt{epub.css} in the user data
directory (see \texttt{-\/-data-dir}). If it is not found there,
sensible defaults will be used.
\item[\texttt{-\/-reference-doc=}\emph{FILE}]
Use the specified file as a style reference in producing a docx or ODT
file.

\begin{description}
\item[Docx]
For best results, the reference docx should be a modified version of a
docx file produced using pandoc. The contents of the reference docx are
ignored, but its stylesheets and document properties (including margins,
page size, header, and footer) are used in the new docx. If no reference
docx is specified on the command line, pandoc will look for a file
\texttt{reference.docx} in the user data directory (see
\texttt{-\/-data-dir}). If this is not found either, sensible defaults
will be used.

To produce a custom \texttt{reference.docx}, first get a copy of the
default \texttt{reference.docx}:
\texttt{pandoc\ -o\ custom-reference.docx\ -\/-print-default-data-file\ reference.docx}.
Then open \texttt{custom-reference.docx} in Word, modify the styles as
you wish, and save the file. For best results, do not make changes to
this file other than modifying the styles used by pandoc:

Paragraph styles:

\begin{itemize}
\tightlist
\item
  Normal
\item
  Body Text
\item
  First Paragraph
\item
  Compact
\item
  Title
\item
  Subtitle
\item
  Author
\item
  Date
\item
  Abstract
\item
  Bibliography
\item
  Heading 1
\item
  Heading 2
\item
  Heading 3
\item
  Heading 4
\item
  Heading 5
\item
  Heading 6
\item
  Heading 7
\item
  Heading 8
\item
  Heading 9
\item
  Block Text
\item
  Source Code
\item
  Footnote Text
\item
  Definition Term
\item
  Definition
\item
  Caption
\item
  Table Caption
\item
  Image Caption
\item
  Figure
\item
  Captioned Figure
\item
  TOC Heading
\end{itemize}

Character styles:

\begin{itemize}
\tightlist
\item
  Default Paragraph Font
\item
  Body Text Char
\item
  Verbatim Char
\item
  Footnote Reference
\item
  Hyperlink
\item
  Section Number
\end{itemize}

Table style:

\begin{itemize}
\tightlist
\item
  Table
\end{itemize}
\item[ODT]
For best results, the reference ODT should be a modified version of an
ODT produced using pandoc. The contents of the reference ODT are
ignored, but its stylesheets are used in the new ODT. If no reference
ODT is specified on the command line, pandoc will look for a file
\texttt{reference.odt} in the user data directory (see
\texttt{-\/-data-dir}). If this is not found either, sensible defaults
will be used.

To produce a custom \texttt{reference.odt}, first get a copy of the
default \texttt{reference.odt}:
\texttt{pandoc\ -o\ custom-reference.odt\ -\/-print-default-data-file\ reference.odt}.
Then open \texttt{custom-reference.odt} in LibreOffice, modify the
styles as you wish, and save the file.
\item[PowerPoint]
Templates included with Microsoft PowerPoint 2013 (either with
\texttt{.pptx} or \texttt{.potx} extension) are known to work, as are
most templates derived from these.

The specific requirement is that the template should contain layouts
with the following names (as seen within PowerPoint):

\begin{itemize}
\tightlist
\item
  Title Slide
\item
  Title and Content
\item
  Section Header
\item
  Two Content
\item
  Comparison
\item
  Content with Caption
\item
  Blank
\end{itemize}

For each name, the first layout found with that name will be used. If no
layout is found with one of the names, pandoc will output a warning and
use the layout with that name from the default reference doc instead.
(How these layouts are used is described in
\hyperref[powerpoint-layout-choice]{PowerPoint layout choice}.)

All templates included with a recent version of MS PowerPoint will fit
these criteria. (You can click on \texttt{Layout} under the
\texttt{Home} menu to check.)

You can also modify the default \texttt{reference.pptx}: first run
\texttt{pandoc\ -o\ custom-reference.pptx\ -\/-print-default-data-file\ reference.pptx},
and then modify \texttt{custom-reference.pptx} in MS PowerPoint (pandoc
will use the layouts with the names listed above).
\end{description}
\item[\texttt{-\/-epub-cover-image=}\emph{FILE}]
Use the specified image as the EPUB cover. It is recommended that the
image be less than 1000px in width and height. Note that in a Markdown
source document you can also specify \texttt{cover-image} in a YAML
metadata block (see \hyperref[epub-metadata]{EPUB Metadata}, below).
\item[\texttt{-\/-epub-metadata=}\emph{FILE}]
Look in the specified XML file for metadata for the EPUB. The file
should contain a series of
\href{https://www.dublincore.org/specifications/dublin-core/dces/}{Dublin
Core elements}. For example:

\begin{verbatim}
 <dc:rights>Creative Commons</dc:rights>
 <dc:language>es-AR</dc:language>
\end{verbatim}

By default, pandoc will include the following metadata elements:
\texttt{\textless{}dc:title\textgreater{}} (from the document title),
\texttt{\textless{}dc:creator\textgreater{}} (from the document
authors), \texttt{\textless{}dc:date\textgreater{}} (from the document
date, which should be in \href{https://www.w3.org/TR/NOTE-datetime}{ISO
8601 format}), \texttt{\textless{}dc:language\textgreater{}} (from the
\texttt{lang} variable, or, if is not set, the locale), and
\texttt{\textless{}dc:identifier\ id="BookId"\textgreater{}} (a randomly
generated UUID). Any of these may be overridden by elements in the
metadata file.

Note: if the source document is Markdown, a YAML metadata block in the
document can be used instead. See below under
\hyperref[epub-metadata]{EPUB Metadata}.
\item[\texttt{-\/-epub-embed-font=}\emph{FILE}]
Embed the specified font in the EPUB. This option can be repeated to
embed multiple fonts. Wildcards can also be used: for example,
\texttt{DejaVuSans-*.ttf}. However, if you use wildcards on the command
line, be sure to escape them or put the whole filename in single quotes,
to prevent them from being interpreted by the shell. To use the embedded
fonts, you will need to add declarations like the following to your CSS
(see \texttt{-\/-css}):

\begin{verbatim}
@font-face {
font-family: DejaVuSans;
font-style: normal;
font-weight: normal;
src:url("DejaVuSans-Regular.ttf");
}
@font-face {
font-family: DejaVuSans;
font-style: normal;
font-weight: bold;
src:url("DejaVuSans-Bold.ttf");
}
@font-face {
font-family: DejaVuSans;
font-style: italic;
font-weight: normal;
src:url("DejaVuSans-Oblique.ttf");
}
@font-face {
font-family: DejaVuSans;
font-style: italic;
font-weight: bold;
src:url("DejaVuSans-BoldOblique.ttf");
}
body { font-family: "DejaVuSans"; }
\end{verbatim}
\item[\texttt{-\/-epub-chapter-level=}\emph{NUMBER}]
Specify the heading level at which to split the EPUB into separate
``chapter'' files. The default is to split into chapters at level-1
headings. This option only affects the internal composition of the EPUB,
not the way chapters and sections are displayed to users. Some readers
may be slow if the chapter files are too large, so for large documents
with few level-1 headings, one might want to use a chapter level of 2 or
3.
\item[\texttt{-\/-epub-subdirectory=}\emph{DIRNAME}]
Specify the subdirectory in the OCF container that is to hold the
EPUB-specific contents. The default is \texttt{EPUB}. To put the EPUB
contents in the top level, use an empty string.
\item[\texttt{-\/-ipynb-output=all\textbar{}none\textbar{}best}]
Determines how ipynb output cells are treated. \texttt{all} means that
all of the data formats included in the original are preserved.
\texttt{none} means that the contents of data cells are omitted.
\texttt{best} causes pandoc to try to pick the richest data block in
each output cell that is compatible with the output format. The default
is \texttt{best}.
\item[\texttt{-\/-pdf-engine=}\emph{PROGRAM}]
Use the specified engine when producing PDF output. Valid values are
\texttt{pdflatex}, \texttt{lualatex}, \texttt{xelatex},
\texttt{latexmk}, \texttt{tectonic}, \texttt{wkhtmltopdf},
\texttt{weasyprint}, \texttt{pagedjs-cli}, \texttt{prince},
\texttt{context}, and \texttt{pdfroff}. If the engine is not in your
PATH, the full path of the engine may be specified here. If this option
is not specified, pandoc uses the following defaults depending on the
output format specified using \texttt{-t/-\/-to}:

\begin{itemize}
\tightlist
\item
  \texttt{-t\ latex} or none: \texttt{pdflatex} (other options:
  \texttt{xelatex}, \texttt{lualatex}, \texttt{tectonic},
  \texttt{latexmk})
\item
  \texttt{-t\ context}: \texttt{context}
\item
  \texttt{-t\ html}: \texttt{wkhtmltopdf} (other options:
  \texttt{prince}, \texttt{weasyprint}, \texttt{pagedjs-cli}; see
  \href{https://print-css.rocks}{print-css.rocks} for a good
  introduction to PDF generation from HTML/CSS)
\item
  \texttt{-t\ ms}: \texttt{pdfroff}
\end{itemize}
\item[\texttt{-\/-pdf-engine-opt=}\emph{STRING}]
Use the given string as a command-line argument to the
\texttt{pdf-engine}. For example, to use a persistent directory
\texttt{foo} for \texttt{latexmk}'s auxiliary files, use
\texttt{-\/-pdf-engine-opt=-outdir=foo}. Note that no check for
duplicate options is done.
\end{description}

\subsection{Citation rendering}\label{citation-rendering}

\begin{description}
\item[\texttt{-C}, \texttt{-\/-citeproc}]
Process the citations in the file, replacing them with rendered
citations and adding a bibliography. Citation processing will not take
place unless bibliographic data is supplied, either through an external
file specified using the \texttt{-\/-bibliography} option or the
\texttt{bibliography} field in metadata, or via a \texttt{references}
section in metadata containing a list of citations in CSL YAML format
with Markdown formatting. The style is controlled by a
\href{https://docs.citationstyles.org/en/stable/specification.html}{CSL}
stylesheet specified using the \texttt{-\/-csl} option or the
\texttt{csl} field in metadata. (If no stylesheet is specified, the
\texttt{chicago-author-date} style will be used by default.) The
citation processing transformation may be applied before or after
filters or Lua filters (see \texttt{-\/-filter},
\texttt{-\/-lua-filter}): these transformations are applied in the order
they appear on the command line. For more information, see the section
on \hyperref[citations]{Citations}.
\item[\texttt{-\/-bibliography=}\emph{FILE}]
Set the \texttt{bibliography} field in the document's metadata to
\emph{FILE}, overriding any value set in the metadata. If you supply
this argument multiple times, each \emph{FILE} will be added to
bibliography. If \emph{FILE} is a URL, it will be fetched via HTTP. If
\emph{FILE} is not found relative to the working directory, it will be
sought in the resource path (see \texttt{-\/-resource-path}).
\item[\texttt{-\/-csl=}\emph{FILE}]
Set the \texttt{csl} field in the document's metadata to \emph{FILE},
overriding any value set in the metadata. (This is equivalent to
\texttt{-\/-metadata\ csl=FILE}.) If \emph{FILE} is a URL, it will be
fetched via HTTP. If \emph{FILE} is not found relative to the working
directory, it will be sought in the resource path (see
\texttt{-\/-resource-path}) and finally in the \texttt{csl} subdirectory
of the pandoc user data directory.
\item[\texttt{-\/-citation-abbreviations=}\emph{FILE}]
Set the \texttt{citation-abbreviations} field in the document's metadata
to \emph{FILE}, overriding any value set in the metadata. (This is
equivalent to \texttt{-\/-metadata\ citation-abbreviations=FILE}.) If
\emph{FILE} is a URL, it will be fetched via HTTP. If \emph{FILE} is not
found relative to the working directory, it will be sought in the
resource path (see \texttt{-\/-resource-path}) and finally in the
\texttt{csl} subdirectory of the pandoc user data directory.
\item[\texttt{-\/-natbib}]
Use \href{https://ctan.org/pkg/natbib}{\texttt{natbib}} for citations in
LaTeX output. This option is not for use with the \texttt{-\/-citeproc}
option or with PDF output. It is intended for use in producing a LaTeX
file that can be processed with
\href{https://ctan.org/pkg/bibtex}{\texttt{bibtex}}.
\item[\texttt{-\/-biblatex}]
Use \href{https://ctan.org/pkg/biblatex}{\texttt{biblatex}} for
citations in LaTeX output. This option is not for use with the
\texttt{-\/-citeproc} option or with PDF output. It is intended for use
in producing a LaTeX file that can be processed with
\href{https://ctan.org/pkg/bibtex}{\texttt{bibtex}} or
\href{https://ctan.org/pkg/biber}{\texttt{biber}}.
\end{description}

\subsection{Math rendering in HTML}\label{math-rendering-in-html}

The default is to render TeX math as far as possible using Unicode
characters. Formulas are put inside a \texttt{span} with
\texttt{class="math"}, so that they may be styled differently from the
surrounding text if needed. However, this gives acceptable results only
for basic math, usually you will want to use \texttt{-\/-mathjax} or
another of the following options.

\begin{description}
\item[\texttt{-\/-mathjax}{[}\texttt{=}\emph{URL}{]}]
Use \href{https://www.mathjax.org}{MathJax} to display embedded TeX math
in HTML output. TeX math will be put between
\texttt{\textbackslash{}(...\textbackslash{})} (for inline math) or
\texttt{\textbackslash{}{[}...\textbackslash{}{]}} (for display math)
and wrapped in \texttt{\textless{}span\textgreater{}} tags with class
\texttt{math}. Then the MathJax JavaScript will render it. The
\emph{URL} should point to the \texttt{MathJax.js} load script. If a
\emph{URL} is not provided, a link to the Cloudflare CDN will be
inserted.
\item[\texttt{-\/-mathml}]
Convert TeX math to \href{https://www.w3.org/Math/}{MathML} (in
\texttt{epub3}, \texttt{docbook4}, \texttt{docbook5}, \texttt{jats},
\texttt{html4} and \texttt{html5}). This is the default in \texttt{odt}
output. Note that currently only Firefox and Safari (and select e-book
readers) natively support MathML.
\item[\texttt{-\/-webtex}{[}\texttt{=}\emph{URL}{]}]
Convert TeX formulas to \texttt{\textless{}img\textgreater{}} tags that
link to an external script that converts formulas to images. The formula
will be URL-encoded and concatenated with the URL provided. For SVG
images you can for example use
\texttt{-\/-webtex\ https://latex.codecogs.com/svg.latex?}. If no URL is
specified, the CodeCogs URL generating PNGs will be used
(\texttt{https://latex.codecogs.com/png.latex?}). Note: the
\texttt{-\/-webtex} option will affect Markdown output as well as HTML,
which is useful if you're targeting a version of Markdown without native
math support.
\item[\texttt{-\/-katex}{[}\texttt{=}\emph{URL}{]}]
Use \href{https://github.com/Khan/KaTeX}{KaTeX} to display embedded TeX
math in HTML output. The \emph{URL} is the base URL for the KaTeX
library. That directory should contain a \texttt{katex.min.js} and a
\texttt{katex.min.css} file. If a \emph{URL} is not provided, a link to
the KaTeX CDN will be inserted.
\item[\texttt{-\/-gladtex}]
Enclose TeX math in \texttt{\textless{}eq\textgreater{}} tags in HTML
output. The resulting HTML can then be processed by
\href{https://humenda.github.io/GladTeX/}{GladTeX} to produce SVG images
of the typeset formulas and an HTML file with these images embedded.

\begin{verbatim}
pandoc -s --gladtex input.md -o myfile.htex
gladtex -d image_dir myfile.htex
# produces myfile.html and images in image_dir
\end{verbatim}
\end{description}

\subsection{Options for wrapper
scripts}\label{options-for-wrapper-scripts}

\begin{description}
\item[\texttt{-\/-dump-args}]
Print information about command-line arguments to \emph{stdout}, then
exit. This option is intended primarily for use in wrapper scripts. The
first line of output contains the name of the output file specified with
the \texttt{-o} option, or \texttt{-} (for \emph{stdout}) if no output
file was specified. The remaining lines contain the command-line
arguments, one per line, in the order they appear. These do not include
regular pandoc options and their arguments, but do include any options
appearing after a \texttt{-\/-} separator at the end of the line.
\item[\texttt{-\/-ignore-args}]
Ignore command-line arguments (for use in wrapper scripts). Regular
pandoc options are not ignored. Thus, for example,

\begin{verbatim}
pandoc --ignore-args -o foo.html -s foo.txt -- -e latin1
\end{verbatim}

is equivalent to

\begin{verbatim}
pandoc -o foo.html -s
\end{verbatim}
\end{description}

\section{Exit codes}\label{exit-codes}

If pandoc completes successfully, it will return exit code 0. Nonzero
exit codes have the following meanings:

\begin{longtable}[]{@{}rl@{}}
\toprule\noalign{}
Code & Error \\
\midrule\noalign{}
\endhead
\bottomrule\noalign{}
\endlastfoot
1 & PandocIOError \\
3 & PandocFailOnWarningError \\
4 & PandocAppError \\
5 & PandocTemplateError \\
6 & PandocOptionError \\
21 & PandocUnknownReaderError \\
22 & PandocUnknownWriterError \\
23 & PandocUnsupportedExtensionError \\
24 & PandocCiteprocError \\
25 & PandocBibliographyError \\
31 & PandocEpubSubdirectoryError \\
43 & PandocPDFError \\
44 & PandocXMLError \\
47 & PandocPDFProgramNotFoundError \\
61 & PandocHttpError \\
62 & PandocShouldNeverHappenError \\
63 & PandocSomeError \\
64 & PandocParseError \\
65 & PandocParsecError \\
66 & PandocMakePDFError \\
67 & PandocSyntaxMapError \\
83 & PandocFilterError \\
84 & PandocLuaError \\
91 & PandocMacroLoop \\
92 & PandocUTF8DecodingError \\
93 & PandocIpynbDecodingError \\
94 & PandocUnsupportedCharsetError \\
97 & PandocCouldNotFindDataFileError \\
98 & PandocCouldNotFindMetadataFileError \\
99 & PandocResourceNotFound \\
\end{longtable}

\section{Defaults files}\label{defaults-files}

The \texttt{-\/-defaults} option may be used to specify a package of
options, in the form of a YAML file.

Fields that are omitted will just have their regular default values. So
a defaults file can be as simple as one line:

\begin{Shaded}
\begin{Highlighting}[]
\FunctionTok{verbosity}\KeywordTok{:}\AttributeTok{ INFO}
\end{Highlighting}
\end{Shaded}

In fields that expect a file path (or list of file paths), the following
syntax may be used to interpolate environment variables:

\begin{Shaded}
\begin{Highlighting}[]
\FunctionTok{csl}\KeywordTok{:}\AttributeTok{  $\{HOME\}/mycsldir/special.csl}
\end{Highlighting}
\end{Shaded}

\texttt{\$\{USERDATA\}} may also be used; this will always resolve to
the user data directory that is current when the defaults file is
parsed, regardless of the setting of the environment variable
\texttt{USERDATA}.

\texttt{\$\{.\}} will resolve to the directory containing the defaults
file itself. This allows you to refer to resources contained in that
directory:

\begin{Shaded}
\begin{Highlighting}[]
\FunctionTok{epub{-}cover{-}image}\KeywordTok{:}\AttributeTok{ $\{.\}/cover.jpg}
\FunctionTok{epub{-}metadata}\KeywordTok{:}\AttributeTok{ $\{.\}/meta.xml}
\FunctionTok{resource{-}path}\KeywordTok{:}
\KeywordTok{{-}}\AttributeTok{ .}\CommentTok{             \# the working directory from which pandoc is run}
\KeywordTok{{-}}\AttributeTok{ $\{.\}/images}\CommentTok{   \# the images subdirectory of the directory}
\CommentTok{                \# containing this defaults file}
\end{Highlighting}
\end{Shaded}

This environment variable interpolation syntax \emph{only} works in
fields that expect file paths.

Defaults files can be placed in the \texttt{defaults} subdirectory of
the user data directory and used from any directory. For example, one
could create a file specifying defaults for writing letters, save it as
\texttt{letter.yaml} in the \texttt{defaults} subdirectory of the user
data directory, and then invoke these defaults from any directory using
\texttt{pandoc\ -\/-defaults\ letter} or \texttt{pandoc\ -dletter}.

When multiple defaults are used, their contents will be combined.

Note that, where command-line arguments may be repeated
(\texttt{-\/-metadata-file}, \texttt{-\/-css},
\texttt{-\/-include-in-header}, \texttt{-\/-include-before-body},
\texttt{-\/-include-after-body}, \texttt{-\/-variable},
\texttt{-\/-metadata}, \texttt{-\/-syntax-definition}), the values
specified on the command line will combine with values specified in the
defaults file, rather than replacing them.

The following tables show the mapping between the command line and
defaults file entries.

\begin{longtable}[]{@{}
  >{\raggedright\arraybackslash}p{(\columnwidth - 2\tabcolsep) * \real{0.4861}}
  >{\raggedright\arraybackslash}p{(\columnwidth - 2\tabcolsep) * \real{0.5000}}@{}}
\toprule\noalign{}
\begin{minipage}[b]{\linewidth}\raggedright
command line
\end{minipage} & \begin{minipage}[b]{\linewidth}\raggedright
defaults file
\end{minipage} \\
\midrule\noalign{}
\endhead
\bottomrule\noalign{}
\endlastfoot
\begin{minipage}[t]{\linewidth}\raggedright
\begin{verbatim}
foo.md
\end{verbatim}
\end{minipage} & \begin{minipage}[t]{\linewidth}\raggedright
\begin{Shaded}
\begin{Highlighting}[]
\FunctionTok{input{-}file}\KeywordTok{:}\AttributeTok{ foo.md}
\end{Highlighting}
\end{Shaded}
\end{minipage} \\
\begin{minipage}[t]{\linewidth}\raggedright
\begin{verbatim}
foo.md bar.md

\end{verbatim}
\end{minipage} & \begin{minipage}[t]{\linewidth}\raggedright
\begin{Shaded}
\begin{Highlighting}[]
\FunctionTok{input{-}files}\KeywordTok{:}
\AttributeTok{  }\KeywordTok{{-}}\AttributeTok{ foo.md}
\AttributeTok{  }\KeywordTok{{-}}\AttributeTok{ bar.md}
\end{Highlighting}
\end{Shaded}
\end{minipage} \\
\end{longtable}

The value of \texttt{input-files} may be left empty to indicate input
from stdin, and it can be an empty sequence \texttt{{[}{]}} for no
input.

\subsection{General options}\label{general-options-1}

\begin{longtable}[]{@{}
  >{\raggedright\arraybackslash}p{(\columnwidth - 2\tabcolsep) * \real{0.4861}}
  >{\raggedright\arraybackslash}p{(\columnwidth - 2\tabcolsep) * \real{0.5000}}@{}}
\toprule\noalign{}
\begin{minipage}[b]{\linewidth}\raggedright
command line
\end{minipage} & \begin{minipage}[b]{\linewidth}\raggedright
defaults file
\end{minipage} \\
\midrule\noalign{}
\endhead
\bottomrule\noalign{}
\endlastfoot
\begin{minipage}[t]{\linewidth}\raggedright
\begin{verbatim}
--from markdown+emoji
\end{verbatim}
\end{minipage} & \begin{minipage}[t]{\linewidth}\raggedright
\begin{Shaded}
\begin{Highlighting}[]
\FunctionTok{from}\KeywordTok{:}\AttributeTok{ markdown+emoji}
\end{Highlighting}
\end{Shaded}

\begin{Shaded}
\begin{Highlighting}[]
\FunctionTok{reader}\KeywordTok{:}\AttributeTok{ markdown+emoji}
\end{Highlighting}
\end{Shaded}
\end{minipage} \\
\begin{minipage}[t]{\linewidth}\raggedright
\begin{verbatim}
--to markdown+hard_line_breaks
\end{verbatim}
\end{minipage} & \begin{minipage}[t]{\linewidth}\raggedright
\begin{Shaded}
\begin{Highlighting}[]
\FunctionTok{to}\KeywordTok{:}\AttributeTok{ markdown+hard\_line\_breaks}
\end{Highlighting}
\end{Shaded}

\begin{Shaded}
\begin{Highlighting}[]
\FunctionTok{writer}\KeywordTok{:}\AttributeTok{ markdown+hard\_line\_breaks}
\end{Highlighting}
\end{Shaded}
\end{minipage} \\
\begin{minipage}[t]{\linewidth}\raggedright
\begin{verbatim}
--output foo.pdf
\end{verbatim}
\end{minipage} & \begin{minipage}[t]{\linewidth}\raggedright
\begin{Shaded}
\begin{Highlighting}[]
\FunctionTok{output{-}file}\KeywordTok{:}\AttributeTok{ foo.pdf}
\end{Highlighting}
\end{Shaded}
\end{minipage} \\
\begin{minipage}[t]{\linewidth}\raggedright
\begin{verbatim}
--output -
\end{verbatim}
\end{minipage} & \begin{minipage}[t]{\linewidth}\raggedright
\begin{Shaded}
\begin{Highlighting}[]
\FunctionTok{output{-}file}\KeywordTok{:}
\end{Highlighting}
\end{Shaded}
\end{minipage} \\
\begin{minipage}[t]{\linewidth}\raggedright
\begin{verbatim}
--data-dir dir
\end{verbatim}
\end{minipage} & \begin{minipage}[t]{\linewidth}\raggedright
\begin{Shaded}
\begin{Highlighting}[]
\FunctionTok{data{-}dir}\KeywordTok{:}\AttributeTok{ dir}
\end{Highlighting}
\end{Shaded}
\end{minipage} \\
\begin{minipage}[t]{\linewidth}\raggedright
\begin{verbatim}
--defaults file
\end{verbatim}
\end{minipage} & \begin{minipage}[t]{\linewidth}\raggedright
\begin{Shaded}
\begin{Highlighting}[]
\FunctionTok{defaults}\KeywordTok{:}
\KeywordTok{{-}}\AttributeTok{ file}
\end{Highlighting}
\end{Shaded}
\end{minipage} \\
\begin{minipage}[t]{\linewidth}\raggedright
\begin{verbatim}
--verbose
\end{verbatim}
\end{minipage} & \begin{minipage}[t]{\linewidth}\raggedright
\begin{Shaded}
\begin{Highlighting}[]
\FunctionTok{verbosity}\KeywordTok{:}\AttributeTok{ INFO}
\end{Highlighting}
\end{Shaded}
\end{minipage} \\
\begin{minipage}[t]{\linewidth}\raggedright
\begin{verbatim}
--quiet
\end{verbatim}
\end{minipage} & \begin{minipage}[t]{\linewidth}\raggedright
\begin{Shaded}
\begin{Highlighting}[]
\FunctionTok{verbosity}\KeywordTok{:}\AttributeTok{ ERROR}
\end{Highlighting}
\end{Shaded}
\end{minipage} \\
\begin{minipage}[t]{\linewidth}\raggedright
\begin{verbatim}
--fail-if-warnings
\end{verbatim}
\end{minipage} & \begin{minipage}[t]{\linewidth}\raggedright
\begin{Shaded}
\begin{Highlighting}[]
\FunctionTok{fail{-}if{-}warnings}\KeywordTok{:}\AttributeTok{ }\CharTok{true}
\end{Highlighting}
\end{Shaded}
\end{minipage} \\
\begin{minipage}[t]{\linewidth}\raggedright
\begin{verbatim}
--sandbox
\end{verbatim}
\end{minipage} & \begin{minipage}[t]{\linewidth}\raggedright
\begin{Shaded}
\begin{Highlighting}[]
\FunctionTok{sandbox}\KeywordTok{:}\AttributeTok{ }\CharTok{true}
\end{Highlighting}
\end{Shaded}
\end{minipage} \\
\begin{minipage}[t]{\linewidth}\raggedright
\begin{verbatim}
--log=FILE
\end{verbatim}
\end{minipage} & \begin{minipage}[t]{\linewidth}\raggedright
\begin{Shaded}
\begin{Highlighting}[]
\FunctionTok{log{-}file}\KeywordTok{:}\AttributeTok{ FILE}
\end{Highlighting}
\end{Shaded}
\end{minipage} \\
\end{longtable}

Options specified in a defaults file itself always have priority over
those in another file included with a \texttt{defaults:} entry.

\texttt{verbosity} can have the values \texttt{ERROR}, \texttt{WARNING},
or \texttt{INFO}.

\subsection{Reader options}\label{reader-options-1}

\begin{longtable}[]{@{}
  >{\raggedright\arraybackslash}p{(\columnwidth - 2\tabcolsep) * \real{0.4861}}
  >{\raggedright\arraybackslash}p{(\columnwidth - 2\tabcolsep) * \real{0.5000}}@{}}
\toprule\noalign{}
\begin{minipage}[b]{\linewidth}\raggedright
command line
\end{minipage} & \begin{minipage}[b]{\linewidth}\raggedright
defaults file
\end{minipage} \\
\midrule\noalign{}
\endhead
\bottomrule\noalign{}
\endlastfoot
\begin{minipage}[t]{\linewidth}\raggedright
\begin{verbatim}
--shift-heading-level-by -1
\end{verbatim}
\end{minipage} & \begin{minipage}[t]{\linewidth}\raggedright
\begin{Shaded}
\begin{Highlighting}[]
\FunctionTok{shift{-}heading{-}level{-}by}\KeywordTok{:}\AttributeTok{ }\DecValTok{{-}1}
\end{Highlighting}
\end{Shaded}
\end{minipage} \\
\begin{minipage}[t]{\linewidth}\raggedright
\begin{verbatim}
--indented-code-classes python
\end{verbatim}
\end{minipage} & \begin{minipage}[t]{\linewidth}\raggedright
\begin{Shaded}
\begin{Highlighting}[]
\FunctionTok{indented{-}code{-}classes}\KeywordTok{:}
\AttributeTok{  }\KeywordTok{{-}}\AttributeTok{ python}
\end{Highlighting}
\end{Shaded}
\end{minipage} \\
\begin{minipage}[t]{\linewidth}\raggedright
\begin{verbatim}
--default-image-extension ".jpg"
\end{verbatim}
\end{minipage} & \begin{minipage}[t]{\linewidth}\raggedright
\begin{Shaded}
\begin{Highlighting}[]
\FunctionTok{default{-}image{-}extension}\KeywordTok{:}\AttributeTok{ }\StringTok{\textquotesingle{}.jpg\textquotesingle{}}
\end{Highlighting}
\end{Shaded}
\end{minipage} \\
\begin{minipage}[t]{\linewidth}\raggedright
\begin{verbatim}
--file-scope
\end{verbatim}
\end{minipage} & \begin{minipage}[t]{\linewidth}\raggedright
\begin{Shaded}
\begin{Highlighting}[]
\FunctionTok{file{-}scope}\KeywordTok{:}\AttributeTok{ }\CharTok{true}
\end{Highlighting}
\end{Shaded}
\end{minipage} \\
\begin{minipage}[t]{\linewidth}\raggedright
\begin{verbatim}
--filter pandoc-citeproc \
 --lua-filter count-words.lua \
 --filter special.lua

\end{verbatim}
\end{minipage} & \begin{minipage}[t]{\linewidth}\raggedright
\begin{Shaded}
\begin{Highlighting}[]
\FunctionTok{filters}\KeywordTok{:}
\AttributeTok{  }\KeywordTok{{-}}\AttributeTok{ pandoc{-}citeproc}
\AttributeTok{  }\KeywordTok{{-}}\AttributeTok{ count{-}words.lua}
\AttributeTok{  }\KeywordTok{{-}}\AttributeTok{ }\FunctionTok{type}\KeywordTok{:}\AttributeTok{ json}
\AttributeTok{    }\FunctionTok{path}\KeywordTok{:}\AttributeTok{ special.lua}
\end{Highlighting}
\end{Shaded}
\end{minipage} \\
\begin{minipage}[t]{\linewidth}\raggedright
\begin{verbatim}
--metadata key=value \
 --metadata key2
\end{verbatim}
\end{minipage} & \begin{minipage}[t]{\linewidth}\raggedright
\begin{Shaded}
\begin{Highlighting}[]
\FunctionTok{metadata}\KeywordTok{:}
\AttributeTok{  }\FunctionTok{key}\KeywordTok{:}\AttributeTok{ value}
\AttributeTok{  }\FunctionTok{key2}\KeywordTok{:}\AttributeTok{ }\CharTok{true}
\end{Highlighting}
\end{Shaded}
\end{minipage} \\
\begin{minipage}[t]{\linewidth}\raggedright
\begin{verbatim}
--metadata-file meta.yaml
\end{verbatim}
\end{minipage} & \begin{minipage}[t]{\linewidth}\raggedright
\begin{Shaded}
\begin{Highlighting}[]
\FunctionTok{metadata{-}files}\KeywordTok{:}
\AttributeTok{  }\KeywordTok{{-}}\AttributeTok{ meta.yaml}
\end{Highlighting}
\end{Shaded}

\begin{Shaded}
\begin{Highlighting}[]
\FunctionTok{metadata{-}file}\KeywordTok{:}\AttributeTok{ meta.yaml}
\end{Highlighting}
\end{Shaded}
\end{minipage} \\
\begin{minipage}[t]{\linewidth}\raggedright
\begin{verbatim}
--preserve-tabs
\end{verbatim}
\end{minipage} & \begin{minipage}[t]{\linewidth}\raggedright
\begin{Shaded}
\begin{Highlighting}[]
\FunctionTok{preserve{-}tabs}\KeywordTok{:}\AttributeTok{ }\CharTok{true}
\end{Highlighting}
\end{Shaded}
\end{minipage} \\
\begin{minipage}[t]{\linewidth}\raggedright
\begin{verbatim}
--tab-stop 8
\end{verbatim}
\end{minipage} & \begin{minipage}[t]{\linewidth}\raggedright
\begin{Shaded}
\begin{Highlighting}[]
\FunctionTok{tab{-}stop}\KeywordTok{:}\AttributeTok{ }\DecValTok{8}
\end{Highlighting}
\end{Shaded}
\end{minipage} \\
\begin{minipage}[t]{\linewidth}\raggedright
\begin{verbatim}
--track-changes accept
\end{verbatim}
\end{minipage} & \begin{minipage}[t]{\linewidth}\raggedright
\begin{Shaded}
\begin{Highlighting}[]
\FunctionTok{track{-}changes}\KeywordTok{:}\AttributeTok{ accept}
\end{Highlighting}
\end{Shaded}
\end{minipage} \\
\begin{minipage}[t]{\linewidth}\raggedright
\begin{verbatim}
--extract-media dir
\end{verbatim}
\end{minipage} & \begin{minipage}[t]{\linewidth}\raggedright
\begin{Shaded}
\begin{Highlighting}[]
\FunctionTok{extract{-}media}\KeywordTok{:}\AttributeTok{ dir}
\end{Highlighting}
\end{Shaded}
\end{minipage} \\
\begin{minipage}[t]{\linewidth}\raggedright
\begin{verbatim}
--abbreviations abbrevs.txt
\end{verbatim}
\end{minipage} & \begin{minipage}[t]{\linewidth}\raggedright
\begin{Shaded}
\begin{Highlighting}[]
\FunctionTok{abbreviations}\KeywordTok{:}\AttributeTok{ abbrevs.txt}
\end{Highlighting}
\end{Shaded}
\end{minipage} \\
\begin{minipage}[t]{\linewidth}\raggedright
\begin{verbatim}
--trace
\end{verbatim}
\end{minipage} & \begin{minipage}[t]{\linewidth}\raggedright
\begin{Shaded}
\begin{Highlighting}[]
\FunctionTok{trace}\KeywordTok{:}\AttributeTok{ }\CharTok{true}
\end{Highlighting}
\end{Shaded}
\end{minipage} \\
\end{longtable}

Metadata values specified in a defaults file are parsed as literal
string text, not Markdown.

Filters will be assumed to be Lua filters if they have the \texttt{.lua}
extension, and JSON filters otherwise. But the filter type can also be
specified explicitly, as shown. Filters are run in the order specified.
To include the built-in citeproc filter, use either \texttt{citeproc} or
\texttt{\{type:\ citeproc\}}.

\subsection{General writer options}\label{general-writer-options-1}

\begin{longtable}[]{@{}
  >{\raggedright\arraybackslash}p{(\columnwidth - 2\tabcolsep) * \real{0.4861}}
  >{\raggedright\arraybackslash}p{(\columnwidth - 2\tabcolsep) * \real{0.5000}}@{}}
\toprule\noalign{}
\begin{minipage}[b]{\linewidth}\raggedright
command line
\end{minipage} & \begin{minipage}[b]{\linewidth}\raggedright
defaults file
\end{minipage} \\
\midrule\noalign{}
\endhead
\bottomrule\noalign{}
\endlastfoot
\begin{minipage}[t]{\linewidth}\raggedright
\begin{verbatim}
--standalone
\end{verbatim}
\end{minipage} & \begin{minipage}[t]{\linewidth}\raggedright
\begin{Shaded}
\begin{Highlighting}[]
\FunctionTok{standalone}\KeywordTok{:}\AttributeTok{ }\CharTok{true}
\end{Highlighting}
\end{Shaded}
\end{minipage} \\
\begin{minipage}[t]{\linewidth}\raggedright
\begin{verbatim}
--template letter
\end{verbatim}
\end{minipage} & \begin{minipage}[t]{\linewidth}\raggedright
\begin{Shaded}
\begin{Highlighting}[]
\FunctionTok{template}\KeywordTok{:}\AttributeTok{ letter}
\end{Highlighting}
\end{Shaded}
\end{minipage} \\
\begin{minipage}[t]{\linewidth}\raggedright
\begin{verbatim}
--variable key=val \
  --variable key2
\end{verbatim}
\end{minipage} & \begin{minipage}[t]{\linewidth}\raggedright
\begin{Shaded}
\begin{Highlighting}[]
\FunctionTok{variables}\KeywordTok{:}
\AttributeTok{  }\FunctionTok{key}\KeywordTok{:}\AttributeTok{ val}
\AttributeTok{  }\FunctionTok{key2}\KeywordTok{:}\AttributeTok{ }\CharTok{true}
\end{Highlighting}
\end{Shaded}
\end{minipage} \\
\begin{minipage}[t]{\linewidth}\raggedright
\begin{verbatim}
--eol nl
\end{verbatim}
\end{minipage} & \begin{minipage}[t]{\linewidth}\raggedright
\begin{Shaded}
\begin{Highlighting}[]
\FunctionTok{eol}\KeywordTok{:}\AttributeTok{ nl}
\end{Highlighting}
\end{Shaded}
\end{minipage} \\
\begin{minipage}[t]{\linewidth}\raggedright
\begin{verbatim}
--dpi 300
\end{verbatim}
\end{minipage} & \begin{minipage}[t]{\linewidth}\raggedright
\begin{Shaded}
\begin{Highlighting}[]
\FunctionTok{dpi}\KeywordTok{:}\AttributeTok{ }\DecValTok{300}
\end{Highlighting}
\end{Shaded}
\end{minipage} \\
\begin{minipage}[t]{\linewidth}\raggedright
\begin{verbatim}
--wrap 60
\end{verbatim}
\end{minipage} & \begin{minipage}[t]{\linewidth}\raggedright
\begin{Shaded}
\begin{Highlighting}[]
\FunctionTok{wrap}\KeywordTok{:}\AttributeTok{ }\DecValTok{60}
\end{Highlighting}
\end{Shaded}
\end{minipage} \\
\begin{minipage}[t]{\linewidth}\raggedright
\begin{verbatim}
--columns 72
\end{verbatim}
\end{minipage} & \begin{minipage}[t]{\linewidth}\raggedright
\begin{Shaded}
\begin{Highlighting}[]
\FunctionTok{columns}\KeywordTok{:}\AttributeTok{ }\DecValTok{72}
\end{Highlighting}
\end{Shaded}
\end{minipage} \\
\begin{minipage}[t]{\linewidth}\raggedright
\begin{verbatim}
--table-of-contents
\end{verbatim}
\end{minipage} & \begin{minipage}[t]{\linewidth}\raggedright
\begin{Shaded}
\begin{Highlighting}[]
\FunctionTok{table{-}of{-}contents}\KeywordTok{:}\AttributeTok{ }\CharTok{true}
\end{Highlighting}
\end{Shaded}
\end{minipage} \\
\begin{minipage}[t]{\linewidth}\raggedright
\begin{verbatim}
--toc
\end{verbatim}
\end{minipage} & \begin{minipage}[t]{\linewidth}\raggedright
\begin{Shaded}
\begin{Highlighting}[]
\FunctionTok{toc}\KeywordTok{:}\AttributeTok{ }\CharTok{true}
\end{Highlighting}
\end{Shaded}
\end{minipage} \\
\begin{minipage}[t]{\linewidth}\raggedright
\begin{verbatim}
--toc-depth 3
\end{verbatim}
\end{minipage} & \begin{minipage}[t]{\linewidth}\raggedright
\begin{Shaded}
\begin{Highlighting}[]
\FunctionTok{toc{-}depth}\KeywordTok{:}\AttributeTok{ }\DecValTok{3}
\end{Highlighting}
\end{Shaded}
\end{minipage} \\
\begin{minipage}[t]{\linewidth}\raggedright
\begin{verbatim}
--strip-comments
\end{verbatim}
\end{minipage} & \begin{minipage}[t]{\linewidth}\raggedright
\begin{Shaded}
\begin{Highlighting}[]
\FunctionTok{strip{-}comments}\KeywordTok{:}\AttributeTok{ }\CharTok{true}
\end{Highlighting}
\end{Shaded}
\end{minipage} \\
\begin{minipage}[t]{\linewidth}\raggedright
\begin{verbatim}
--no-highlight
\end{verbatim}
\end{minipage} & \begin{minipage}[t]{\linewidth}\raggedright
\begin{Shaded}
\begin{Highlighting}[]
\FunctionTok{highlight{-}style}\KeywordTok{:}\AttributeTok{ }\CharTok{null}
\end{Highlighting}
\end{Shaded}
\end{minipage} \\
\begin{minipage}[t]{\linewidth}\raggedright
\begin{verbatim}
--highlight-style kate
\end{verbatim}
\end{minipage} & \begin{minipage}[t]{\linewidth}\raggedright
\begin{Shaded}
\begin{Highlighting}[]
\FunctionTok{highlight{-}style}\KeywordTok{:}\AttributeTok{ kate}
\end{Highlighting}
\end{Shaded}
\end{minipage} \\
\begin{minipage}[t]{\linewidth}\raggedright
\begin{verbatim}
--syntax-definition mylang.xml
\end{verbatim}
\end{minipage} & \begin{minipage}[t]{\linewidth}\raggedright
\begin{Shaded}
\begin{Highlighting}[]
\FunctionTok{syntax{-}definitions}\KeywordTok{:}
\AttributeTok{  }\KeywordTok{{-}}\AttributeTok{ mylang.xml}
\end{Highlighting}
\end{Shaded}

\begin{Shaded}
\begin{Highlighting}[]
\FunctionTok{syntax{-}definition}\KeywordTok{:}\AttributeTok{ mylang.xml}
\end{Highlighting}
\end{Shaded}
\end{minipage} \\
\begin{minipage}[t]{\linewidth}\raggedright
\begin{verbatim}
--include-in-header inc.tex
\end{verbatim}
\end{minipage} & \begin{minipage}[t]{\linewidth}\raggedright
\begin{Shaded}
\begin{Highlighting}[]
\FunctionTok{include{-}in{-}header}\KeywordTok{:}
\AttributeTok{  }\KeywordTok{{-}}\AttributeTok{ inc.tex}
\end{Highlighting}
\end{Shaded}
\end{minipage} \\
\begin{minipage}[t]{\linewidth}\raggedright
\begin{verbatim}
--include-before-body inc.tex
\end{verbatim}
\end{minipage} & \begin{minipage}[t]{\linewidth}\raggedright
\begin{Shaded}
\begin{Highlighting}[]
\FunctionTok{include{-}before{-}body}\KeywordTok{:}
\AttributeTok{  }\KeywordTok{{-}}\AttributeTok{ inc.tex}
\end{Highlighting}
\end{Shaded}
\end{minipage} \\
\begin{minipage}[t]{\linewidth}\raggedright
\begin{verbatim}
--include-after-body inc.tex
\end{verbatim}
\end{minipage} & \begin{minipage}[t]{\linewidth}\raggedright
\begin{Shaded}
\begin{Highlighting}[]
\FunctionTok{include{-}after{-}body}\KeywordTok{:}
\AttributeTok{  }\KeywordTok{{-}}\AttributeTok{ inc.tex}
\end{Highlighting}
\end{Shaded}
\end{minipage} \\
\begin{minipage}[t]{\linewidth}\raggedright
\begin{verbatim}
--resource-path .:foo
\end{verbatim}
\end{minipage} & \begin{minipage}[t]{\linewidth}\raggedright
\begin{Shaded}
\begin{Highlighting}[]
\FunctionTok{resource{-}path}\KeywordTok{:}\AttributeTok{ }\KeywordTok{[}\StringTok{\textquotesingle{}.\textquotesingle{}}\KeywordTok{,}\StringTok{\textquotesingle{}foo\textquotesingle{}}\KeywordTok{]}
\end{Highlighting}
\end{Shaded}
\end{minipage} \\
\begin{minipage}[t]{\linewidth}\raggedright
\begin{verbatim}
--request-header foo:bar
\end{verbatim}
\end{minipage} & \begin{minipage}[t]{\linewidth}\raggedright
\begin{Shaded}
\begin{Highlighting}[]
\FunctionTok{request{-}headers}\KeywordTok{:}
\AttributeTok{  }\KeywordTok{{-}}\AttributeTok{ }\KeywordTok{[}\StringTok{"User{-}Agent"}\KeywordTok{,}\AttributeTok{ }\StringTok{"Mozilla/5.0"}\KeywordTok{]}
\end{Highlighting}
\end{Shaded}
\end{minipage} \\
\begin{minipage}[t]{\linewidth}\raggedright
\begin{verbatim}
--no-check-certificate
\end{verbatim}
\end{minipage} & \begin{minipage}[t]{\linewidth}\raggedright
\begin{Shaded}
\begin{Highlighting}[]
\FunctionTok{no{-}check{-}certificate}\KeywordTok{:}\AttributeTok{ }\CharTok{true}
\end{Highlighting}
\end{Shaded}
\end{minipage} \\
\end{longtable}

\subsection{Options affecting specific
writers}\label{options-affecting-specific-writers-1}

\begin{longtable}[]{@{}
  >{\raggedright\arraybackslash}p{(\columnwidth - 2\tabcolsep) * \real{0.4861}}
  >{\raggedright\arraybackslash}p{(\columnwidth - 2\tabcolsep) * \real{0.5000}}@{}}
\toprule\noalign{}
\begin{minipage}[b]{\linewidth}\raggedright
command line
\end{minipage} & \begin{minipage}[b]{\linewidth}\raggedright
defaults file
\end{minipage} \\
\midrule\noalign{}
\endhead
\bottomrule\noalign{}
\endlastfoot
\begin{minipage}[t]{\linewidth}\raggedright
\begin{verbatim}
--self-contained
\end{verbatim}
\end{minipage} & \begin{minipage}[t]{\linewidth}\raggedright
\begin{Shaded}
\begin{Highlighting}[]
\FunctionTok{self{-}contained}\KeywordTok{:}\AttributeTok{ }\CharTok{true}
\end{Highlighting}
\end{Shaded}
\end{minipage} \\
\begin{minipage}[t]{\linewidth}\raggedright
\begin{verbatim}
--html-q-tags
\end{verbatim}
\end{minipage} & \begin{minipage}[t]{\linewidth}\raggedright
\begin{Shaded}
\begin{Highlighting}[]
\FunctionTok{html{-}q{-}tags}\KeywordTok{:}\AttributeTok{ }\CharTok{true}
\end{Highlighting}
\end{Shaded}
\end{minipage} \\
\begin{minipage}[t]{\linewidth}\raggedright
\begin{verbatim}
--ascii
\end{verbatim}
\end{minipage} & \begin{minipage}[t]{\linewidth}\raggedright
\begin{Shaded}
\begin{Highlighting}[]
\FunctionTok{ascii}\KeywordTok{:}\AttributeTok{ }\CharTok{true}
\end{Highlighting}
\end{Shaded}
\end{minipage} \\
\begin{minipage}[t]{\linewidth}\raggedright
\begin{verbatim}
--reference-links
\end{verbatim}
\end{minipage} & \begin{minipage}[t]{\linewidth}\raggedright
\begin{Shaded}
\begin{Highlighting}[]
\FunctionTok{reference{-}links}\KeywordTok{:}\AttributeTok{ }\CharTok{true}
\end{Highlighting}
\end{Shaded}
\end{minipage} \\
\begin{minipage}[t]{\linewidth}\raggedright
\begin{verbatim}
--reference-location block
\end{verbatim}
\end{minipage} & \begin{minipage}[t]{\linewidth}\raggedright
\begin{Shaded}
\begin{Highlighting}[]
\FunctionTok{reference{-}location}\KeywordTok{:}\AttributeTok{ block}
\end{Highlighting}
\end{Shaded}
\end{minipage} \\
\begin{minipage}[t]{\linewidth}\raggedright
\begin{verbatim}
--markdown-headings atx
\end{verbatim}
\end{minipage} & \begin{minipage}[t]{\linewidth}\raggedright
\begin{Shaded}
\begin{Highlighting}[]
\FunctionTok{markdown{-}headings}\KeywordTok{:}\AttributeTok{ atx}
\end{Highlighting}
\end{Shaded}
\end{minipage} \\
\begin{minipage}[t]{\linewidth}\raggedright
\begin{verbatim}
--top-level-division chapter
\end{verbatim}
\end{minipage} & \begin{minipage}[t]{\linewidth}\raggedright
\begin{Shaded}
\begin{Highlighting}[]
\FunctionTok{top{-}level{-}division}\KeywordTok{:}\AttributeTok{ chapter}
\end{Highlighting}
\end{Shaded}
\end{minipage} \\
\begin{minipage}[t]{\linewidth}\raggedright
\begin{verbatim}
--number-sections
\end{verbatim}
\end{minipage} & \begin{minipage}[t]{\linewidth}\raggedright
\begin{Shaded}
\begin{Highlighting}[]
\FunctionTok{number{-}sections}\KeywordTok{:}\AttributeTok{ }\CharTok{true}
\end{Highlighting}
\end{Shaded}
\end{minipage} \\
\begin{minipage}[t]{\linewidth}\raggedright
\begin{verbatim}
--number-offset=1,4
\end{verbatim}
\end{minipage} & \begin{minipage}[t]{\linewidth}\raggedright
\begin{Shaded}
\begin{Highlighting}[]
\FunctionTok{number{-}offset}\KeywordTok{:}\AttributeTok{ \textbackslash{}[1,4\textbackslash{}]}
\end{Highlighting}
\end{Shaded}
\end{minipage} \\
\begin{minipage}[t]{\linewidth}\raggedright
\begin{verbatim}
--listings
\end{verbatim}
\end{minipage} & \begin{minipage}[t]{\linewidth}\raggedright
\begin{Shaded}
\begin{Highlighting}[]
\FunctionTok{listings}\KeywordTok{:}\AttributeTok{ }\CharTok{true}
\end{Highlighting}
\end{Shaded}
\end{minipage} \\
\begin{minipage}[t]{\linewidth}\raggedright
\begin{verbatim}
--incremental
\end{verbatim}
\end{minipage} & \begin{minipage}[t]{\linewidth}\raggedright
\begin{Shaded}
\begin{Highlighting}[]
\FunctionTok{incremental}\KeywordTok{:}\AttributeTok{ }\CharTok{true}
\end{Highlighting}
\end{Shaded}
\end{minipage} \\
\begin{minipage}[t]{\linewidth}\raggedright
\begin{verbatim}
--slide-level 2
\end{verbatim}
\end{minipage} & \begin{minipage}[t]{\linewidth}\raggedright
\begin{Shaded}
\begin{Highlighting}[]
\FunctionTok{slide{-}level}\KeywordTok{:}\AttributeTok{ }\DecValTok{2}
\end{Highlighting}
\end{Shaded}
\end{minipage} \\
\begin{minipage}[t]{\linewidth}\raggedright
\begin{verbatim}
--section-divs
\end{verbatim}
\end{minipage} & \begin{minipage}[t]{\linewidth}\raggedright
\begin{Shaded}
\begin{Highlighting}[]
\FunctionTok{section{-}divs}\KeywordTok{:}\AttributeTok{ }\CharTok{true}
\end{Highlighting}
\end{Shaded}
\end{minipage} \\
\begin{minipage}[t]{\linewidth}\raggedright
\begin{verbatim}
--email-obfuscation references
\end{verbatim}
\end{minipage} & \begin{minipage}[t]{\linewidth}\raggedright
\begin{Shaded}
\begin{Highlighting}[]
\FunctionTok{email{-}obfuscation}\KeywordTok{:}\AttributeTok{ references}
\end{Highlighting}
\end{Shaded}
\end{minipage} \\
\begin{minipage}[t]{\linewidth}\raggedright
\begin{verbatim}
--id-prefix ch1
\end{verbatim}
\end{minipage} & \begin{minipage}[t]{\linewidth}\raggedright
\begin{Shaded}
\begin{Highlighting}[]
\FunctionTok{identifier{-}prefix}\KeywordTok{:}\AttributeTok{ ch1}
\end{Highlighting}
\end{Shaded}
\end{minipage} \\
\begin{minipage}[t]{\linewidth}\raggedright
\begin{verbatim}
--title-prefix MySite
\end{verbatim}
\end{minipage} & \begin{minipage}[t]{\linewidth}\raggedright
\begin{Shaded}
\begin{Highlighting}[]
\FunctionTok{title{-}prefix}\KeywordTok{:}\AttributeTok{ MySite}
\end{Highlighting}
\end{Shaded}
\end{minipage} \\
\begin{minipage}[t]{\linewidth}\raggedright
\begin{verbatim}
--css styles/screen.css  \
  --css styles/special.css
\end{verbatim}
\end{minipage} & \begin{minipage}[t]{\linewidth}\raggedright
\begin{Shaded}
\begin{Highlighting}[]
\FunctionTok{css}\KeywordTok{:}
\AttributeTok{  }\KeywordTok{{-}}\AttributeTok{ styles/screen.css}
\AttributeTok{  }\KeywordTok{{-}}\AttributeTok{ styles/special.css}
\end{Highlighting}
\end{Shaded}
\end{minipage} \\
\begin{minipage}[t]{\linewidth}\raggedright
\begin{verbatim}
--reference-doc my.docx
\end{verbatim}
\end{minipage} & \begin{minipage}[t]{\linewidth}\raggedright
\begin{Shaded}
\begin{Highlighting}[]
\FunctionTok{reference{-}doc}\KeywordTok{:}\AttributeTok{ my.docx}
\end{Highlighting}
\end{Shaded}
\end{minipage} \\
\begin{minipage}[t]{\linewidth}\raggedright
\begin{verbatim}
--epub-cover-image cover.jpg
\end{verbatim}
\end{minipage} & \begin{minipage}[t]{\linewidth}\raggedright
\begin{Shaded}
\begin{Highlighting}[]
\FunctionTok{epub{-}cover{-}image}\KeywordTok{:}\AttributeTok{ cover.jpg}
\end{Highlighting}
\end{Shaded}
\end{minipage} \\
\begin{minipage}[t]{\linewidth}\raggedright
\begin{verbatim}
--epub-metadata meta.xml
\end{verbatim}
\end{minipage} & \begin{minipage}[t]{\linewidth}\raggedright
\begin{Shaded}
\begin{Highlighting}[]
\FunctionTok{epub{-}metadata}\KeywordTok{:}\AttributeTok{ meta.xml}
\end{Highlighting}
\end{Shaded}
\end{minipage} \\
\begin{minipage}[t]{\linewidth}\raggedright
\begin{verbatim}
--epub-embed-font special.otf \
  --epub-embed-font headline.otf
\end{verbatim}
\end{minipage} & \begin{minipage}[t]{\linewidth}\raggedright
\begin{Shaded}
\begin{Highlighting}[]
\FunctionTok{epub{-}fonts}\KeywordTok{:}
\AttributeTok{  }\KeywordTok{{-}}\AttributeTok{ special.otf}
\AttributeTok{  }\KeywordTok{{-}}\AttributeTok{ headline.otf}
\end{Highlighting}
\end{Shaded}
\end{minipage} \\
\begin{minipage}[t]{\linewidth}\raggedright
\begin{verbatim}
--epub-chapter-level 2
\end{verbatim}
\end{minipage} & \begin{minipage}[t]{\linewidth}\raggedright
\begin{Shaded}
\begin{Highlighting}[]
\FunctionTok{epub{-}chapter{-}level}\KeywordTok{:}\AttributeTok{ }\DecValTok{2}
\end{Highlighting}
\end{Shaded}
\end{minipage} \\
\begin{minipage}[t]{\linewidth}\raggedright
\begin{verbatim}
--epub-subdirectory=""
\end{verbatim}
\end{minipage} & \begin{minipage}[t]{\linewidth}\raggedright
\begin{Shaded}
\begin{Highlighting}[]
\FunctionTok{epub{-}subdirectory}\KeywordTok{:}\AttributeTok{ }\StringTok{\textquotesingle{}\textquotesingle{}}
\end{Highlighting}
\end{Shaded}
\end{minipage} \\
\begin{minipage}[t]{\linewidth}\raggedright
\begin{verbatim}
--ipynb-output best
\end{verbatim}
\end{minipage} & \begin{minipage}[t]{\linewidth}\raggedright
\begin{Shaded}
\begin{Highlighting}[]
\FunctionTok{ipynb{-}output}\KeywordTok{:}\AttributeTok{ best}
\end{Highlighting}
\end{Shaded}
\end{minipage} \\
\begin{minipage}[t]{\linewidth}\raggedright
\begin{verbatim}
--pdf-engine xelatex
\end{verbatim}
\end{minipage} & \begin{minipage}[t]{\linewidth}\raggedright
\begin{Shaded}
\begin{Highlighting}[]
\FunctionTok{pdf{-}engine}\KeywordTok{:}\AttributeTok{ xelatex}
\end{Highlighting}
\end{Shaded}
\end{minipage} \\
\begin{minipage}[t]{\linewidth}\raggedright
\begin{verbatim}
--pdf-engine-opt=--shell-escape
\end{verbatim}
\end{minipage} & \begin{minipage}[t]{\linewidth}\raggedright
\begin{Shaded}
\begin{Highlighting}[]
\FunctionTok{pdf{-}engine{-}opts}\KeywordTok{:}
\AttributeTok{  }\KeywordTok{{-}}\AttributeTok{ }\StringTok{\textquotesingle{}{-}shell{-}escape\textquotesingle{}}
\end{Highlighting}
\end{Shaded}

\begin{Shaded}
\begin{Highlighting}[]
\FunctionTok{pdf{-}engine{-}opt}\KeywordTok{:}\AttributeTok{ }\StringTok{\textquotesingle{}{-}shell{-}escape\textquotesingle{}}
\end{Highlighting}
\end{Shaded}
\end{minipage} \\
\end{longtable}

\subsection{Citation rendering}\label{citation-rendering-1}

\begin{longtable}[]{@{}
  >{\raggedright\arraybackslash}p{(\columnwidth - 2\tabcolsep) * \real{0.4861}}
  >{\raggedright\arraybackslash}p{(\columnwidth - 2\tabcolsep) * \real{0.5000}}@{}}
\toprule\noalign{}
\begin{minipage}[b]{\linewidth}\raggedright
command line
\end{minipage} & \begin{minipage}[b]{\linewidth}\raggedright
defaults file
\end{minipage} \\
\midrule\noalign{}
\endhead
\bottomrule\noalign{}
\endlastfoot
\begin{minipage}[t]{\linewidth}\raggedright
\begin{verbatim}
--citeproc
\end{verbatim}
\end{minipage} & \begin{minipage}[t]{\linewidth}\raggedright
\begin{Shaded}
\begin{Highlighting}[]
\FunctionTok{citeproc}\KeywordTok{:}\AttributeTok{ }\CharTok{true}
\end{Highlighting}
\end{Shaded}
\end{minipage} \\
\begin{minipage}[t]{\linewidth}\raggedright
\begin{verbatim}
--bibliography logic.bib
\end{verbatim}
\end{minipage} & \begin{minipage}[t]{\linewidth}\raggedright
\begin{Shaded}
\begin{Highlighting}[]
\FunctionTok{metadata}\KeywordTok{:}
\AttributeTok{  }\FunctionTok{bibliography}\KeywordTok{:}\AttributeTok{ logic.bib}
\end{Highlighting}
\end{Shaded}
\end{minipage} \\
\begin{minipage}[t]{\linewidth}\raggedright
\begin{verbatim}
--csl ieee.csl
\end{verbatim}
\end{minipage} & \begin{minipage}[t]{\linewidth}\raggedright
\begin{Shaded}
\begin{Highlighting}[]
\FunctionTok{metadata}\KeywordTok{:}
\AttributeTok{  }\FunctionTok{csl}\KeywordTok{:}\AttributeTok{ ieee.csl}
\end{Highlighting}
\end{Shaded}
\end{minipage} \\
\begin{minipage}[t]{\linewidth}\raggedright
\begin{verbatim}
--citation-abbreviations ab.json
\end{verbatim}
\end{minipage} & \begin{minipage}[t]{\linewidth}\raggedright
\begin{Shaded}
\begin{Highlighting}[]
\FunctionTok{metadata}\KeywordTok{:}
\AttributeTok{  }\FunctionTok{citation{-}abbreviations}\KeywordTok{:}\AttributeTok{ ab.json}
\end{Highlighting}
\end{Shaded}
\end{minipage} \\
\begin{minipage}[t]{\linewidth}\raggedright
\begin{verbatim}
--natbib
\end{verbatim}
\end{minipage} & \begin{minipage}[t]{\linewidth}\raggedright
\begin{Shaded}
\begin{Highlighting}[]
\FunctionTok{cite{-}method}\KeywordTok{:}\AttributeTok{ natbib}
\end{Highlighting}
\end{Shaded}
\end{minipage} \\
\begin{minipage}[t]{\linewidth}\raggedright
\begin{verbatim}
--biblatex
\end{verbatim}
\end{minipage} & \begin{minipage}[t]{\linewidth}\raggedright
\begin{Shaded}
\begin{Highlighting}[]
\FunctionTok{cite{-}method}\KeywordTok{:}\AttributeTok{ biblatex}
\end{Highlighting}
\end{Shaded}
\end{minipage} \\
\end{longtable}

\texttt{cite-method} can be \texttt{citeproc}, \texttt{natbib}, or
\texttt{biblatex}. This only affects LaTeX output. If you want to use
citeproc to format citations, you should also set `citeproc: true'.

If you need control over when the citeproc processing is done relative
to other filters, you should instead use \texttt{citeproc} in the list
of \texttt{filters} (see above).

\subsection{Math rendering in HTML}\label{math-rendering-in-html-1}

\begin{longtable}[]{@{}
  >{\raggedright\arraybackslash}p{(\columnwidth - 2\tabcolsep) * \real{0.4861}}
  >{\raggedright\arraybackslash}p{(\columnwidth - 2\tabcolsep) * \real{0.5000}}@{}}
\toprule\noalign{}
\begin{minipage}[b]{\linewidth}\raggedright
command line
\end{minipage} & \begin{minipage}[b]{\linewidth}\raggedright
defaults file
\end{minipage} \\
\midrule\noalign{}
\endhead
\bottomrule\noalign{}
\endlastfoot
\begin{minipage}[t]{\linewidth}\raggedright
\begin{verbatim}
--mathjax
\end{verbatim}
\end{minipage} & \begin{minipage}[t]{\linewidth}\raggedright
\begin{Shaded}
\begin{Highlighting}[]
\FunctionTok{html{-}math{-}method}\KeywordTok{:}
\AttributeTok{  }\FunctionTok{method}\KeywordTok{:}\AttributeTok{ mathjax}
\end{Highlighting}
\end{Shaded}
\end{minipage} \\
\begin{minipage}[t]{\linewidth}\raggedright
\begin{verbatim}
--mathml
\end{verbatim}
\end{minipage} & \begin{minipage}[t]{\linewidth}\raggedright
\begin{Shaded}
\begin{Highlighting}[]
\FunctionTok{html{-}math{-}method}\KeywordTok{:}
\AttributeTok{  }\FunctionTok{method}\KeywordTok{:}\AttributeTok{ mathml}
\end{Highlighting}
\end{Shaded}
\end{minipage} \\
\begin{minipage}[t]{\linewidth}\raggedright
\begin{verbatim}
--webtex
\end{verbatim}
\end{minipage} & \begin{minipage}[t]{\linewidth}\raggedright
\begin{Shaded}
\begin{Highlighting}[]
\FunctionTok{html{-}math{-}method}\KeywordTok{:}
\AttributeTok{  }\FunctionTok{method}\KeywordTok{:}\AttributeTok{ webtex}
\end{Highlighting}
\end{Shaded}
\end{minipage} \\
\begin{minipage}[t]{\linewidth}\raggedright
\begin{verbatim}
--katex
\end{verbatim}
\end{minipage} & \begin{minipage}[t]{\linewidth}\raggedright
\begin{Shaded}
\begin{Highlighting}[]
\FunctionTok{html{-}math{-}method}\KeywordTok{:}
\AttributeTok{  }\FunctionTok{method}\KeywordTok{:}\AttributeTok{ katex}
\end{Highlighting}
\end{Shaded}
\end{minipage} \\
\begin{minipage}[t]{\linewidth}\raggedright
\begin{verbatim}
--gladtex
\end{verbatim}
\end{minipage} & \begin{minipage}[t]{\linewidth}\raggedright
\begin{Shaded}
\begin{Highlighting}[]
\FunctionTok{html{-}math{-}method}\KeywordTok{:}
\AttributeTok{  }\FunctionTok{method}\KeywordTok{:}\AttributeTok{ gladtex}
\end{Highlighting}
\end{Shaded}
\end{minipage} \\
\end{longtable}

In addition to the values listed above, \texttt{method} can have the
value \texttt{plain}.

If the command line option accepts a URL argument, an \texttt{url:}
field can be added to \texttt{html-math-method:}.

\subsection{Options for wrapper
scripts}\label{options-for-wrapper-scripts-1}

\begin{longtable}[]{@{}
  >{\raggedright\arraybackslash}p{(\columnwidth - 2\tabcolsep) * \real{0.4861}}
  >{\raggedright\arraybackslash}p{(\columnwidth - 2\tabcolsep) * \real{0.5000}}@{}}
\toprule\noalign{}
\begin{minipage}[b]{\linewidth}\raggedright
command line
\end{minipage} & \begin{minipage}[b]{\linewidth}\raggedright
defaults file
\end{minipage} \\
\midrule\noalign{}
\endhead
\bottomrule\noalign{}
\endlastfoot
\begin{minipage}[t]{\linewidth}\raggedright
\begin{verbatim}
--dump-args
\end{verbatim}
\end{minipage} & \begin{minipage}[t]{\linewidth}\raggedright
\begin{Shaded}
\begin{Highlighting}[]
\FunctionTok{dump{-}args}\KeywordTok{:}\AttributeTok{ }\CharTok{true}
\end{Highlighting}
\end{Shaded}
\end{minipage} \\
\begin{minipage}[t]{\linewidth}\raggedright
\begin{verbatim}
--ignore-args
\end{verbatim}
\end{minipage} & \begin{minipage}[t]{\linewidth}\raggedright
\begin{Shaded}
\begin{Highlighting}[]
\FunctionTok{ignore{-}args}\KeywordTok{:}\AttributeTok{ }\CharTok{true}
\end{Highlighting}
\end{Shaded}
\end{minipage} \\
\end{longtable}

\section{Templates}\label{templates}

When the \texttt{-s/-\/-standalone} option is used, pandoc uses a
template to add header and footer material that is needed for a
self-standing document. To see the default template that is used, just
type

\begin{verbatim}
pandoc -D *FORMAT*
\end{verbatim}

where \emph{FORMAT} is the name of the output format. A custom template
can be specified using the \texttt{-\/-template} option. You can also
override the system default templates for a given output format
\emph{FORMAT} by putting a file \texttt{templates/default.*FORMAT*} in
the user data directory (see \texttt{-\/-data-dir}, above).
\emph{Exceptions:}

\begin{itemize}
\tightlist
\item
  For \texttt{odt} output, customize the \texttt{default.opendocument}
  template.
\item
  For \texttt{pdf} output, customize the \texttt{default.latex} template
  (or the \texttt{default.context} template, if you use
  \texttt{-t\ context}, or the \texttt{default.ms} template, if you use
  \texttt{-t\ ms}, or the \texttt{default.html} template, if you use
  \texttt{-t\ html}).
\item
  \texttt{docx} and \texttt{pptx} have no template (however, you can use
  \texttt{-\/-reference-doc} to customize the output).
\end{itemize}

Templates contain \emph{variables}, which allow for the inclusion of
arbitrary information at any point in the file. They may be set at the
command line using the \texttt{-V/-\/-variable} option. If a variable is
not set, pandoc will look for the key in the document's metadata, which
can be set using either \hyperref[extension-yaml_metadata_block]{YAML
metadata blocks} or with the \texttt{-M/-\/-metadata} option. In
addition, some variables are given default values by pandoc. See
\hyperref[variables]{Variables} below for a list of variables used in
pandoc's default templates.

If you use custom templates, you may need to revise them as pandoc
changes. We recommend tracking the changes in the default templates, and
modifying your custom templates accordingly. An easy way to do this is
to fork the
\href{https://github.com/jgm/pandoc-templates}{pandoc-templates}
repository and merge in changes after each pandoc release.

\subsection{Template syntax}\label{template-syntax}

\subsubsection{Comments}\label{comments}

Anything between the sequence \texttt{\$-\/-} and the end of the line
will be treated as a comment and omitted from the output.

\subsubsection{Delimiters}\label{delimiters}

To mark variables and control structures in the template, either
\texttt{\$}\ldots{}\texttt{\$} or \texttt{\$\{}\ldots{}\texttt{\}} may
be used as delimiters. The styles may also be mixed in the same
template, but the opening and closing delimiter must match in each case.
The opening delimiter may be followed by one or more spaces or tabs,
which will be ignored. The closing delimiter may be followed by one or
more spaces or tabs, which will be ignored.

To include a literal \texttt{\$} in the document, use \texttt{\$\$}.

\subsubsection{Interpolated variables}\label{interpolated-variables}

A slot for an interpolated variable is a variable name surrounded by
matched delimiters. Variable names must begin with a letter and can
contain letters, numbers, \texttt{\_}, \texttt{-}, and \texttt{.}. The
keywords \texttt{it}, \texttt{if}, \texttt{else}, \texttt{endif},
\texttt{for}, \texttt{sep}, and \texttt{endfor} may not be used as
variable names. Examples:

\begin{verbatim}
$foo$
$foo.bar.baz$
$foo_bar.baz-bim$
$ foo $
${foo}
${foo.bar.baz}
${foo_bar.baz-bim}
${ foo }
\end{verbatim}

Variable names with periods are used to get at structured variable
values. So, for example, \texttt{employee.salary} will return the value
of the \texttt{salary} field of the object that is the value of the
\texttt{employee} field.

\begin{itemize}
\tightlist
\item
  If the value of the variable is a simple value, it will be rendered
  verbatim. (Note that no escaping is done; the assumption is that the
  calling program will escape the strings appropriately for the output
  format.)
\item
  If the value is a list, the values will be concatenated.
\item
  If the value is a map, the string \texttt{true} will be rendered.
\item
  Every other value will be rendered as the empty string.
\end{itemize}

\subsubsection{Conditionals}\label{conditionals}

A conditional begins with \texttt{if(variable)} (enclosed in matched
delimiters) and ends with \texttt{endif} (enclosed in matched
delimiters). It may optionally contain an \texttt{else} (enclosed in
matched delimiters). The \texttt{if} section is used if
\texttt{variable} has a non-empty value, otherwise the \texttt{else}
section is used (if present). Examples:

\begin{verbatim}
$if(foo)$bar$endif$

$if(foo)$
  $foo$
$endif$

$if(foo)$
part one
$else$
part two
$endif$

${if(foo)}bar${endif}

${if(foo)}
  ${foo}
${endif}

${if(foo)}
${ foo.bar }
${else}
no foo!
${endif}
\end{verbatim}

The keyword \texttt{elseif} may be used to simplify complex nested
conditionals:

\begin{verbatim}
$if(foo)$
XXX
$elseif(bar)$
YYY
$else$
ZZZ
$endif$
\end{verbatim}

\subsubsection{For loops}\label{for-loops}

A for loop begins with \texttt{for(variable)} (enclosed in matched
delimiters) and ends with \texttt{endfor} (enclosed in matched
delimiters).

\begin{itemize}
\tightlist
\item
  If \texttt{variable} is an array, the material inside the loop will be
  evaluated repeatedly, with \texttt{variable} being set to each value
  of the array in turn, and concatenated.
\item
  If \texttt{variable} is a map, the material inside will be set to the
  map.
\item
  If the value of the associated variable is not an array or a map, a
  single iteration will be performed on its value.
\end{itemize}

Examples:

\begin{verbatim}
$for(foo)$$foo$$sep$, $endfor$

$for(foo)$
  - $foo.last$, $foo.first$
$endfor$

${ for(foo.bar) }
  - ${ foo.bar.last }, ${ foo.bar.first }
${ endfor }

$for(mymap)$
$it.name$: $it.office$
$endfor$
\end{verbatim}

You may optionally specify a separator between consecutive values using
\texttt{sep} (enclosed in matched delimiters). The material between
\texttt{sep} and the \texttt{endfor} is the separator.

\begin{verbatim}
${ for(foo) }${ foo }${ sep }, ${ endfor }
\end{verbatim}

Instead of using \texttt{variable} inside the loop, the special
anaphoric keyword \texttt{it} may be used.

\begin{verbatim}
${ for(foo.bar) }
  - ${ it.last }, ${ it.first }
${ endfor }
\end{verbatim}

\subsubsection{Partials}\label{partials}

Partials (subtemplates stored in different files) may be included by
using the name of the partial, followed by \texttt{()}, for example:

\begin{verbatim}
${ styles() }
\end{verbatim}

Partials will be sought in the directory containing the main template.
The file name will be assumed to have the same extension as the main
template if it lacks an extension. When calling the partial, the full
name including file extension can also be used:

\begin{verbatim}
${ styles.html() }
\end{verbatim}

(If a partial is not found in the directory of the template and the
template path is given as a relative path, it will also be sought in the
\texttt{templates} subdirectory of the user data directory.)

Partials may optionally be applied to variables using a colon:

\begin{verbatim}
${ date:fancy() }

${ articles:bibentry() }
\end{verbatim}

If \texttt{articles} is an array, this will iterate over its values,
applying the partial \texttt{bibentry()} to each one. So the second
example above is equivalent to

\begin{verbatim}
${ for(articles) }
${ it:bibentry() }
${ endfor }
\end{verbatim}

Note that the anaphoric keyword \texttt{it} must be used when iterating
over partials. In the above examples, the \texttt{bibentry} partial
should contain \texttt{it.title} (and so on) instead of
\texttt{articles.title}.

Final newlines are omitted from included partials.

Partials may include other partials.

A separator between values of an array may be specified in square
brackets, immediately after the variable name or partial:

\begin{verbatim}
${months[, ]}$

${articles:bibentry()[; ]$
\end{verbatim}

The separator in this case is literal and (unlike with \texttt{sep} in
an explicit \texttt{for} loop) cannot contain interpolated variables or
other template directives.

\subsubsection{Nesting}\label{nesting}

To ensure that content is ``nested,'' that is, subsequent lines
indented, use the \texttt{\^{}} directive:

\begin{verbatim}
$item.number$  $^$$item.description$ ($item.price$)
\end{verbatim}

In this example, if \texttt{item.description} has multiple lines, they
will all be indented to line up with the first line:

\begin{verbatim}
00123  A fine bottle of 18-year old
       Oban whiskey. ($148)
\end{verbatim}

To nest multiple lines to the same level, align them with the
\texttt{\^{}} directive in the template. For example:

\begin{verbatim}
$item.number$  $^$$item.description$ ($item.price$)
               (Available til $item.sellby$.)
\end{verbatim}

will produce

\begin{verbatim}
00123  A fine bottle of 18-year old
       Oban whiskey. ($148)
       (Available til March 30, 2020.)
\end{verbatim}

If a variable occurs by itself on a line, preceded by whitespace and not
followed by further text or directives on the same line, and the
variable's value contains multiple lines, it will be nested
automatically.

\subsubsection{Breakable spaces}\label{breakable-spaces}

Normally, spaces in the template itself (as opposed to values of the
interpolated variables) are not breakable, but they can be made
breakable in part of the template by using the
\texttt{\textasciitilde{}} keyword (ended with another
\texttt{\textasciitilde{}}).

\begin{verbatim}
$~$This long line may break if the document is rendered
with a short line length.$~$
\end{verbatim}

\subsubsection{Pipes}\label{pipes}

A pipe transforms the value of a variable or partial. Pipes are
specified using a slash (\texttt{/}) between the variable name (or
partial) and the pipe name. Example:

\begin{verbatim}
$for(name)$
$name/uppercase$
$endfor$

$for(metadata/pairs)$
- $it.key$: $it.value$
$endfor$

$employee:name()/uppercase$
\end{verbatim}

Pipes may be chained:

\begin{verbatim}
$for(employees/pairs)$
$it.key/alpha/uppercase$. $it.name$
$endfor$
\end{verbatim}

Some pipes take parameters:

\begin{verbatim}
|----------------------|------------|
$for(employee)$
$it.name.first/uppercase/left 20 "| "$$it.name.salary/right 10 " | " " |"$
$endfor$
|----------------------|------------|
\end{verbatim}

Currently the following pipes are predefined:

\begin{itemize}
\item
  \texttt{pairs}: Converts a map or array to an array of maps, each with
  \texttt{key} and \texttt{value} fields. If the original value was an
  array, the \texttt{key} will be the array index, starting with 1.
\item
  \texttt{uppercase}: Converts text to uppercase.
\item
  \texttt{lowercase}: Converts text to lowercase.
\item
  \texttt{length}: Returns the length of the value: number of characters
  for a textual value, number of elements for a map or array.
\item
  \texttt{reverse}: Reverses a textual value or array, and has no effect
  on other values.
\item
  \texttt{first}: Returns the first value of an array, if applied to a
  non-empty array; otherwise returns the original value.
\item
  \texttt{last}: Returns the last value of an array, if applied to a
  non-empty array; otherwise returns the original value.
\item
  \texttt{rest}: Returns all but the first value of an array, if applied
  to a non-empty array; otherwise returns the original value.
\item
  \texttt{allbutlast}: Returns all but the last value of an array, if
  applied to a non-empty array; otherwise returns the original value.
\item
  \texttt{chomp}: Removes trailing newlines (and breakable space).
\item
  \texttt{nowrap}: Disables line wrapping on breakable spaces.
\item
  \texttt{alpha}: Converts textual values that can be read as an integer
  into lowercase alphabetic characters \texttt{a..z} (mod 26). This can
  be used to get lettered enumeration from array indices. To get
  uppercase letters, chain with \texttt{uppercase}.
\item
  \texttt{roman}: Converts textual values that can be read as an integer
  into lowercase roman numerals. This can be used to get lettered
  enumeration from array indices. To get uppercase roman, chain with
  \texttt{uppercase}.
\item
  \texttt{left\ n\ "leftborder"\ "rightborder"}: Renders a textual value
  in a block of width \texttt{n}, aligned to the left, with an optional
  left and right border. Has no effect on other values. This can be used
  to align material in tables. Widths are positive integers indicating
  the number of characters. Borders are strings inside double quotes;
  literal \texttt{"} and \texttt{\textbackslash{}} characters must be
  backslash-escaped.
\item
  \texttt{right\ n\ "leftborder"\ "rightborder"}: Renders a textual
  value in a block of width \texttt{n}, aligned to the right, and has no
  effect on other values.
\item
  \texttt{center\ n\ "leftborder"\ "rightborder"}: Renders a textual
  value in a block of width \texttt{n}, aligned to the center, and has
  no effect on other values.
\end{itemize}

\subsection{Variables}\label{variables}

\subsubsection{Metadata variables}\label{metadata-variables}

\begin{description}
\item[\texttt{title}, \texttt{author}, \texttt{date}]
allow identification of basic aspects of the document. Included in PDF
metadata through LaTeX and ConTeXt. These can be set through a
\hyperref[extension-pandoc_title_block]{pandoc title block}, which
allows for multiple authors, or through a
\hyperref[extension-yaml_metadata_block]{YAML metadata block}:

\begin{verbatim}
---
author:
- Aristotle
- Peter Abelard
...
\end{verbatim}

Note that if you just want to set PDF or HTML metadata, without
including a title block in the document itself, you can set the
\texttt{title-meta}, \texttt{author-meta}, and \texttt{date-meta}
variables. (By default these are set automatically, based on
\texttt{title}, \texttt{author}, and \texttt{date}.) The page title in
HTML is set by \texttt{pagetitle}, which is equal to \texttt{title} by
default.
\item[\texttt{subtitle}]
document subtitle, included in HTML, EPUB, LaTeX, ConTeXt, and docx
documents
\item[\texttt{abstract}]
document summary, included in HTML, LaTeX, ConTeXt, AsciiDoc, and docx
documents
\item[\texttt{abstract-title}]
title of abstract, currently used only in HTML and EPUB. This will be
set automatically to a localized value, depending on \texttt{lang}, but
can be manually overridden.
\item[\texttt{keywords}]
list of keywords to be included in HTML, PDF, ODT, pptx, docx and
AsciiDoc metadata; repeat as for \texttt{author}, above
\item[\texttt{subject}]
document subject, included in ODT, PDF, docx, EPUB, and pptx metadata
\item[\texttt{description}]
document description, included in ODT, docx and pptx metadata. Some
applications show this as \texttt{Comments} metadata.
\item[\texttt{category}]
document category, included in docx and pptx metadata
\end{description}

Additionally, any root-level string metadata, not included in ODT, docx
or pptx metadata is added as a \emph{custom property}. The following
\href{https://yaml.org/spec/1.2/spec.html}{YAML} metadata block for
instance:

\begin{verbatim}
---
title:  'This is the title'
subtitle: "This is the subtitle"
author:
- Author One
- Author Two
description: |
    This is a long
    description.

    It consists of two paragraphs
...
\end{verbatim}

will include \texttt{title}, \texttt{author} and \texttt{description} as
standard document properties and \texttt{subtitle} as a custom property
when converting to docx, ODT or pptx.

\subsubsection{Language variables}\label{language-variables}

\begin{description}
\item[\texttt{lang}]
identifies the main language of the document using IETF language tags
(following the \href{https://tools.ietf.org/html/bcp47}{BCP 47}
standard), such as \texttt{en} or \texttt{en-GB}. The
\href{https://r12a.github.io/app-subtags/}{Language subtag lookup} tool
can look up or verify these tags. This affects most formats, and
controls hyphenation in PDF output when using LaTeX (through
\href{https://ctan.org/pkg/babel}{\texttt{babel}} and
\href{https://ctan.org/pkg/polyglossia}{\texttt{polyglossia}}) or
ConTeXt.

Use native pandoc \hyperref[divs-and-spans]{Divs and Spans} with the
\texttt{lang} attribute to switch the language:

\begin{verbatim}
---
lang: en-GB
...

Text in the main document language (British English).

::: {lang=fr-CA}
> Cette citation est écrite en français canadien.
:::

More text in English. ['Zitat auf Deutsch.']{lang=de}
\end{verbatim}
\item[\texttt{dir}]
the base script direction, either \texttt{rtl} (right-to-left) or
\texttt{ltr} (left-to-right).

For bidirectional documents, native pandoc \texttt{span}s and
\texttt{div}s with the \texttt{dir} attribute (value \texttt{rtl} or
\texttt{ltr}) can be used to override the base direction in some output
formats. This may not always be necessary if the final renderer
(e.g.~the browser, when generating HTML) supports the
\href{https://www.w3.org/International/articles/inline-bidi-markup/uba-basics}{Unicode
Bidirectional Algorithm}.

When using LaTeX for bidirectional documents, only the \texttt{xelatex}
engine is fully supported (use \texttt{-\/-pdf-engine=xelatex}).
\end{description}

\subsubsection{Variables for HTML}\label{variables-for-html}

\begin{description}
\tightlist
\item[\texttt{document-css}]
Enables inclusion of most of the
\href{https://developer.mozilla.org/en-US/docs/Learn/CSS}{CSS} in the
\texttt{styles.html} \hyperref[partials]{partial} (have a look with
\texttt{pandoc\ -\/-print-default-data-file=templates/styles.html}).
Unless you use \hyperref[option--css]{\texttt{-\/-css}}, this variable
is set to \texttt{true} by default. You can disable it with
e.g.~\texttt{pandoc\ -M\ document-css=false}.
\item[\texttt{mainfont}]
sets the CSS \texttt{font-family} property on the \texttt{html} element.
\item[\texttt{fontsize}]
sets the base CSS \texttt{font-size}, which you'd usually set to
e.g.~\texttt{20px}, but it also accepts \texttt{pt} (12pt = 16px in most
browsers).
\item[\texttt{fontcolor}]
sets the CSS \texttt{color} property on the \texttt{html} element.
\item[\texttt{linkcolor}]
sets the CSS \texttt{color} property on all links.
\item[\texttt{monofont}]
sets the CSS \texttt{font-family} property on \texttt{code} elements.
\item[\texttt{monobackgroundcolor}]
sets the CSS \texttt{background-color} property on \texttt{code}
elements and adds extra padding.
\item[\texttt{linestretch}]
sets the CSS \texttt{line-height} property on the \texttt{html} element,
which is preferred to be unitless.
\item[\texttt{backgroundcolor}]
sets the CSS \texttt{background-color} property on the \texttt{html}
element.
\item[\texttt{margin-left}, \texttt{margin-right}, \texttt{margin-top},
\texttt{margin-bottom}]
sets the corresponding CSS \texttt{padding} properties on the
\texttt{body} element.
\end{description}

To override or extend some
\href{https://developer.mozilla.org/en-US/docs/Learn/CSS}{CSS} for just
one document, include for example:

\begin{verbatim}
---
header-includes: |
  <style>
  blockquote {
    font-style: italic;
  }
  tr.even {
    background-color: #f0f0f0;
  }
  td, th {
    padding: 0.5em 2em 0.5em 0.5em;
  }
  tbody {
    border-bottom: none;
  }
  </style>
---
\end{verbatim}

\subsubsection{Variables for HTML math}\label{variables-for-html-math}

\begin{description}
\tightlist
\item[\texttt{classoption}]
when using \hyperref[option--katex]{KaTeX}, you can render display math
equations flush left using \hyperref[layout]{YAML metadata} or with
\texttt{-M\ classoption=fleqn}.
\end{description}

\subsubsection{Variables for HTML
slides}\label{variables-for-html-slides}

These affect HTML output when \hyperref[slide-shows]{producing slide
shows with pandoc}.

\begin{description}
\tightlist
\item[\texttt{institute}]
author affiliations: can be a list when there are multiple authors
\item[\texttt{revealjs-url}]
base URL for reveal.js documents (defaults to
\texttt{https://unpkg.com/reveal.js@\^{}4/})
\item[\texttt{s5-url}]
base URL for S5 documents (defaults to \texttt{s5/default})
\item[\texttt{slidy-url}]
base URL for Slidy documents (defaults to
\texttt{https://www.w3.org/Talks/Tools/Slidy2})
\item[\texttt{slideous-url}]
base URL for Slideous documents (defaults to \texttt{slideous})
\item[\texttt{title-slide-attributes}]
additional attributes for the title slide of reveal.js slide shows. See
\hyperref[background-in-reveal.js-beamer-and-pptx]{background in
reveal.js, beamer, and pptx} for an example.
\end{description}

All \href{https://revealjs.com/config/}{reveal.js configuration options}
are available as variables. To turn off boolean flags that default to
true in reveal.js, use \texttt{0}.

\subsubsection{Variables for Beamer
slides}\label{variables-for-beamer-slides}

These variables change the appearance of PDF slides using
\href{https://ctan.org/pkg/beamer}{\texttt{beamer}}.

\begin{description}
\tightlist
\item[\texttt{aspectratio}]
slide aspect ratio (\texttt{43} for 4:3 {[}default{]}, \texttt{169} for
16:9, \texttt{1610} for 16:10, \texttt{149} for 14:9, \texttt{141} for
1.41:1, \texttt{54} for 5:4, \texttt{32} for 3:2)
\item[\texttt{beameroption}]
add extra beamer option with
\texttt{\textbackslash{}setbeameroption\{\}}
\item[\texttt{institute}]
author affiliations: can be a list when there are multiple authors
\item[\texttt{logo}]
logo image for slides
\item[\texttt{navigation}]
controls navigation symbols (default is \texttt{empty} for no navigation
symbols; other valid values are \texttt{frame}, \texttt{vertical}, and
\texttt{horizontal})
\item[\texttt{section-titles}]
enables ``title pages'' for new sections (default is true)
\item[\texttt{theme}, \texttt{colortheme}, \texttt{fonttheme},
\texttt{innertheme}, \texttt{outertheme}]
beamer themes
\item[\texttt{themeoptions}]
options for LaTeX beamer themes (a list).
\item[\texttt{titlegraphic}]
image for title slide
\end{description}

\subsubsection{Variables for PowerPoint}\label{variables-for-powerpoint}

These variables control the visual aspects of a slide show that are not
easily controlled via templates.

\begin{description}
\tightlist
\item[\texttt{monofont}]
font to use for code.
\end{description}

\subsubsection{Variables for LaTeX}\label{variables-for-latex}

Pandoc uses these variables when \hyperref[creating-a-pdf]{creating a
PDF} with a LaTeX engine.

\paragraph{Layout}\label{layout}

\begin{description}
\item[\texttt{block-headings}]
make \texttt{\textbackslash{}paragraph} and
\texttt{\textbackslash{}subparagraph} (fourth- and fifth-level headings,
or fifth- and sixth-level with book classes) free-standing rather than
run-in; requires further formatting to distinguish from
\texttt{\textbackslash{}subsubsection} (third- or fourth-level
headings). Instead of using this option,
\href{https://ctan.org/pkg/koma-script}{KOMA-Script} can adjust headings
more extensively:

\begin{verbatim}
---
documentclass: scrartcl
header-includes: |
  \RedeclareSectionCommand[
    beforeskip=-10pt plus -2pt minus -1pt,
    afterskip=1sp plus -1sp minus 1sp,
    font=\normalfont\itshape]{paragraph}
  \RedeclareSectionCommand[
    beforeskip=-10pt plus -2pt minus -1pt,
    afterskip=1sp plus -1sp minus 1sp,
    font=\normalfont\scshape,
    indent=0pt]{subparagraph}
...
\end{verbatim}
\item[\texttt{classoption}]
option for document class, e.g.~\texttt{oneside}; repeat for multiple
options:

\begin{verbatim}
---
classoption:
- twocolumn
- landscape
...
\end{verbatim}
\item[\texttt{documentclass}]
document class: usually one of the standard classes,
\href{https://ctan.org/pkg/article}{\texttt{article}},
\href{https://ctan.org/pkg/book}{\texttt{book}}, and
\href{https://ctan.org/pkg/report}{\texttt{report}}; the
\href{https://ctan.org/pkg/koma-script}{KOMA-Script} equivalents,
\texttt{scrartcl}, \texttt{scrbook}, and \texttt{scrreprt}, which
default to smaller margins; or
\href{https://ctan.org/pkg/memoir}{\texttt{memoir}}
\item[\texttt{geometry}]
option for \href{https://ctan.org/pkg/geometry}{\texttt{geometry}}
package, e.g.~\texttt{margin=1in}; repeat for multiple options:

\begin{verbatim}
---
geometry:
- top=30mm
- left=20mm
- heightrounded
...
\end{verbatim}
\item[\texttt{hyperrefoptions}]
option for \href{https://ctan.org/pkg/hyperref}{\texttt{hyperref}}
package, e.g.~\texttt{linktoc=all}; repeat for multiple options:

\begin{verbatim}
---
hyperrefoptions:
- linktoc=all
- pdfwindowui
- pdfpagemode=FullScreen
...
\end{verbatim}
\item[\texttt{indent}]
if true, pandoc will use document class settings for indentation (the
default LaTeX template otherwise removes indentation and adds space
between paragraphs)
\item[\texttt{linestretch}]
adjusts line spacing using the
\href{https://ctan.org/pkg/setspace}{\texttt{setspace}} package,
e.g.~\texttt{1.25}, \texttt{1.5}
\item[\texttt{margin-left}, \texttt{margin-right}, \texttt{margin-top},
\texttt{margin-bottom}]
sets margins if \texttt{geometry} is not used (otherwise
\texttt{geometry} overrides these)
\item[\texttt{pagestyle}]
control \texttt{\textbackslash{}pagestyle\{\}}: the default article
class supports \texttt{plain} (default), \texttt{empty} (no running
heads or page numbers), and \texttt{headings} (section titles in running
heads)
\item[\texttt{papersize}]
paper size, e.g.~\texttt{letter}, \texttt{a4}
\item[\texttt{secnumdepth}]
numbering depth for sections (with \texttt{-\/-number-sections} option
or \texttt{numbersections} variable)
\item[\texttt{beamerarticle}]
produce an article from Beamer slides
\end{description}

\paragraph{Fonts}\label{fonts}

\begin{description}
\item[\texttt{fontenc}]
allows font encoding to be specified through \texttt{fontenc} package
(with \texttt{pdflatex}); default is \texttt{T1} (see
\href{https://ctan.org/pkg/encguide}{LaTeX font encodings guide})
\item[\texttt{fontfamily}]
font package for use with \texttt{pdflatex}:
\href{https://www.tug.org/texlive/}{TeX Live} includes many options,
documented in the \href{https://tug.org/FontCatalogue/}{LaTeX Font
Catalogue}. The default is \href{https://ctan.org/pkg/lm}{Latin Modern}.
\item[\texttt{fontfamilyoptions}]
options for package used as \texttt{fontfamily}; repeat for multiple
options. For example, to use the Libertine font with proportional
lowercase (old-style) figures through the
\href{https://ctan.org/pkg/libertinus}{\texttt{libertinus}} package:

\begin{verbatim}
---
fontfamily: libertinus
fontfamilyoptions:
- osf
- p
...
\end{verbatim}
\item[\texttt{fontsize}]
font size for body text. The standard classes allow 10pt, 11pt, and
12pt. To use another size, set \texttt{documentclass} to one of the
\href{https://ctan.org/pkg/koma-script}{KOMA-Script} classes, such as
\texttt{scrartcl} or \texttt{scrbook}.
\item[\texttt{mainfont}, \texttt{sansfont}, \texttt{monofont},
\texttt{mathfont}, \texttt{CJKmainfont}]
font families for use with \texttt{xelatex} or \texttt{lualatex}: take
the name of any system font, using the
\href{https://ctan.org/pkg/fontspec}{\texttt{fontspec}} package.
\texttt{CJKmainfont} uses the
\href{https://ctan.org/pkg/xecjk}{\texttt{xecjk}} package.
\item[\texttt{mainfontoptions}, \texttt{sansfontoptions},
\texttt{monofontoptions}, \texttt{mathfontoptions}, \texttt{CJKoptions}]
options to use with \texttt{mainfont}, \texttt{sansfont},
\texttt{monofont}, \texttt{mathfont}, \texttt{CJKmainfont} in
\texttt{xelatex} and \texttt{lualatex}. Allow for any choices available
through \href{https://ctan.org/pkg/fontspec}{\texttt{fontspec}}; repeat
for multiple options. For example, to use the
\href{http://www.gust.org.pl/projects/e-foundry/tex-gyre}{TeX Gyre}
version of Palatino with lowercase figures:

\begin{verbatim}
---
mainfont: TeX Gyre Pagella
mainfontoptions:
- Numbers=Lowercase
- Numbers=Proportional
...
\end{verbatim}
\item[\texttt{microtypeoptions}]
options to pass to the microtype package
\end{description}

\paragraph{Links}\label{links}

\begin{description}
\tightlist
\item[\texttt{colorlinks}]
add color to link text; automatically enabled if any of
\texttt{linkcolor}, \texttt{filecolor}, \texttt{citecolor},
\texttt{urlcolor}, or \texttt{toccolor} are set
\item[\texttt{boxlinks}]
add visible box around links (has no effect if \texttt{colorlinks} is
set)
\item[\texttt{linkcolor}, \texttt{filecolor}, \texttt{citecolor},
\texttt{urlcolor}, \texttt{toccolor}]
color for internal links, external links, citation links, linked URLs,
and links in table of contents, respectively: uses options allowed by
\href{https://ctan.org/pkg/xcolor}{\texttt{xcolor}}, including the
\texttt{dvipsnames}, \texttt{svgnames}, and \texttt{x11names} lists
\item[\texttt{links-as-notes}]
causes links to be printed as footnotes
\end{description}

\paragraph{Front matter}\label{front-matter}

\begin{description}
\tightlist
\item[\texttt{lof}, \texttt{lot}]
include list of figures, list of tables
\item[\texttt{thanks}]
contents of acknowledgments footnote after document title
\item[\texttt{toc}]
include table of contents (can also be set using
\texttt{-\/-toc/-\/-table-of-contents})
\item[\texttt{toc-depth}]
level of section to include in table of contents
\end{description}

\paragraph{BibLaTeX Bibliographies}\label{biblatex-bibliographies}

These variables function when using BibLaTeX for
\hyperref[citation-rendering]{citation rendering}.

\begin{description}
\tightlist
\item[\texttt{biblatexoptions}]
list of options for biblatex
\item[\texttt{biblio-style}]
bibliography style, when used with \texttt{-\/-natbib} and
\texttt{-\/-biblatex}
\item[\texttt{biblio-title}]
bibliography title, when used with \texttt{-\/-natbib} and
\texttt{-\/-biblatex}
\item[\texttt{bibliography}]
bibliography to use for resolving references
\item[\texttt{natbiboptions}]
list of options for natbib
\end{description}

\subsubsection{Variables for ConTeXt}\label{variables-for-context}

Pandoc uses these variables when \hyperref[creating-a-pdf]{creating a
PDF} with ConTeXt.

\begin{description}
\tightlist
\item[\texttt{fontsize}]
font size for body text (e.g.~\texttt{10pt}, \texttt{12pt})
\item[\texttt{headertext}, \texttt{footertext}]
text to be placed in running header or footer (see
\href{https://wiki.contextgarden.net/Headers_and_Footers}{ConTeXt
Headers and Footers}); repeat up to four times for different placement
\item[\texttt{indenting}]
controls indentation of paragraphs, e.g.~\texttt{yes,small,next} (see
\href{https://wiki.contextgarden.net/Indentation}{ConTeXt Indentation});
repeat for multiple options
\item[\texttt{interlinespace}]
adjusts line spacing, e.g.~\texttt{4ex} (using
\href{https://wiki.contextgarden.net/Command/setupinterlinespace}{\texttt{setupinterlinespace}});
repeat for multiple options
\item[\texttt{layout}]
options for page margins and text arrangement (see
\href{https://wiki.contextgarden.net/Layout}{ConTeXt Layout}); repeat
for multiple options
\item[\texttt{linkcolor}, \texttt{contrastcolor}]
color for links outside and inside a page, e.g.~\texttt{red},
\texttt{blue} (see \href{https://wiki.contextgarden.net/Color}{ConTeXt
Color})
\item[\texttt{linkstyle}]
typeface style for links, e.g.~\texttt{normal}, \texttt{bold},
\texttt{slanted}, \texttt{boldslanted}, \texttt{type}, \texttt{cap},
\texttt{small}
\item[\texttt{lof}, \texttt{lot}]
include list of figures, list of tables
\item[\texttt{mainfont}, \texttt{sansfont}, \texttt{monofont},
\texttt{mathfont}]
font families: take the name of any system font (see
\href{https://wiki.contextgarden.net/Font_Switching}{ConTeXt Font
Switching})
\item[\texttt{margin-left}, \texttt{margin-right}, \texttt{margin-top},
\texttt{margin-bottom}]
sets margins, if \texttt{layout} is not used (otherwise \texttt{layout}
overrides these)
\item[\texttt{pagenumbering}]
page number style and location (using
\href{https://wiki.contextgarden.net/Command/setuppagenumbering}{\texttt{setuppagenumbering}});
repeat for multiple options
\item[\texttt{papersize}]
paper size, e.g.~\texttt{letter}, \texttt{A4}, \texttt{landscape} (see
\href{https://wiki.contextgarden.net/PaperSetup}{ConTeXt Paper Setup});
repeat for multiple options
\item[\texttt{pdfa}]
adds to the preamble the setup necessary to generate PDF/A of the type
specified, e.g.~\texttt{1a:2005}, \texttt{2a}. If no type is specified
(i.e.~the value is set to True, by e.g. \texttt{-\/-metadata=pdfa} or
\texttt{pdfa:\ true} in a YAML metadata block), \texttt{1b:2005} will be
used as default, for reasons of backwards compatibility. Using
\texttt{-\/-variable=pdfa} without specified value is not supported. To
successfully generate PDF/A the required ICC color profiles have to be
available and the content and all included files (such as images) have
to be standard-conforming. The ICC profiles and output intent may be
specified using the variables \texttt{pdfaiccprofile} and
\texttt{pdfaintent}. See also
\href{https://wiki.contextgarden.net/PDF/A}{ConTeXt PDFA} for more
details.
\item[\texttt{pdfaiccprofile}]
when used in conjunction with \texttt{pdfa}, specifies the ICC profile
to use in the PDF, e.g.~\texttt{default.cmyk}. If left unspecified,
\texttt{sRGB.icc} is used as default. May be repeated to include
multiple profiles. Note that the profiles have to be available on the
system. They can be obtained from
\href{https://wiki.contextgarden.net/PDFX\#ICC_profiles}{ConTeXt ICC
Profiles}.
\item[\texttt{pdfaintent}]
when used in conjunction with \texttt{pdfa}, specifies the output intent
for the colors,
e.g.~\texttt{ISO\ coated\ v2\ 300\textbackslash{}letterpercent\textbackslash{}space\ (ECI)}
If left unspecified, \texttt{sRGB\ IEC61966-2.1} is used as default.
\item[\texttt{toc}]
include table of contents (can also be set using
\texttt{-\/-toc/-\/-table-of-contents})
\item[\texttt{whitespace}]
spacing between paragraphs, e.g.~\texttt{none}, \texttt{small} (using
\href{https://wiki.contextgarden.net/Command/setupwhitespace}{\texttt{setupwhitespace}})
\item[\texttt{includesource}]
include all source documents as file attachments in the PDF file
\end{description}

\subsubsection{\texorpdfstring{Variables for
\texttt{wkhtmltopdf}}{Variables for wkhtmltopdf}}\label{variables-for-wkhtmltopdf}

Pandoc uses these variables when \hyperref[creating-a-pdf]{creating a
PDF} with \href{https://wkhtmltopdf.org}{\texttt{wkhtmltopdf}}. The
\texttt{-\/-css} option also affects the output.

\begin{description}
\tightlist
\item[\texttt{footer-html}, \texttt{header-html}]
add information to the header and footer
\item[\texttt{margin-left}, \texttt{margin-right}, \texttt{margin-top},
\texttt{margin-bottom}]
set the page margins
\item[\texttt{papersize}]
sets the PDF paper size
\end{description}

\subsubsection{Variables for man pages}\label{variables-for-man-pages}

\begin{description}
\tightlist
\item[\texttt{adjusting}]
adjusts text to left (\texttt{l}), right (\texttt{r}), center
(\texttt{c}), or both (\texttt{b}) margins
\item[\texttt{footer}]
footer in man pages
\item[\texttt{header}]
header in man pages
\item[\texttt{hyphenate}]
if \texttt{true} (the default), hyphenation will be used
\item[\texttt{section}]
section number in man pages
\end{description}

\subsubsection{Variables for ms}\label{variables-for-ms}

\begin{description}
\tightlist
\item[\texttt{fontfamily}]
font family (e.g.~\texttt{T} or \texttt{P})
\item[\texttt{indent}]
paragraph indent (e.g.~\texttt{2m})
\item[\texttt{lineheight}]
line height (e.g.~\texttt{12p})
\item[\texttt{pointsize}]
point size (e.g.~\texttt{10p})
\end{description}

\subsubsection{Variables set
automatically}\label{variables-set-automatically}

Pandoc sets these variables automatically in response to
\hyperref[options]{options} or document contents; users can also modify
them. These vary depending on the output format, and include the
following:

\begin{description}
\item[\texttt{body}]
body of document
\item[\texttt{date-meta}]
the \texttt{date} variable converted to ISO 8601 YYYY-MM-DD, included in
all HTML based formats (dzslides, epub, html, html4, html5, revealjs,
s5, slideous, slidy). The recognized formats for \texttt{date} are:
\texttt{mm/dd/yyyy}, \texttt{mm/dd/yy}, \texttt{yyyy-mm-dd} (ISO 8601),
\texttt{dd\ MM\ yyyy} (e.g.~either \texttt{02\ Apr\ 2018} or
\texttt{02\ April\ 2018}), \texttt{MM\ dd,\ yyyy}
(e.g.~\texttt{Apr.\ 02,\ 2018} or
\texttt{April\ 02,\ 2018),}yyyy{[}mm{[}dd{]}{]}\texttt{(e.g.}20180402,
\texttt{201804} or \texttt{2018}).
\item[\texttt{header-includes}]
contents specified by \texttt{-H/-\/-include-in-header} (may have
multiple values)
\item[\texttt{include-before}]
contents specified by \texttt{-B/-\/-include-before-body} (may have
multiple values)
\item[\texttt{include-after}]
contents specified by \texttt{-A/-\/-include-after-body} (may have
multiple values)
\item[\texttt{meta-json}]
JSON representation of all of the document's metadata. Field values are
transformed to the selected output format.
\item[\texttt{numbersections}]
non-null value if \texttt{-N/-\/-number-sections} was specified
\item[\texttt{sourcefile}, \texttt{outputfile}]
source and destination filenames, as given on the command line.
\texttt{sourcefile} can also be a list if input comes from multiple
files, or empty if input is from stdin. You can use the following
snippet in your template to distinguish them:

\begin{verbatim}
$if(sourcefile)$
$for(sourcefile)$
$sourcefile$
$endfor$
$else$
(stdin)
$endif$
\end{verbatim}

Similarly, \texttt{outputfile} can be \texttt{-} if output goes to the
terminal.

If you need absolute paths, use e.g.~\texttt{\$curdir\$/\$sourcefile\$}.
\item[\texttt{curdir}]
working directory from which pandoc is run.
\item[\texttt{toc}]
non-null value if \texttt{-\/-toc/-\/-table-of-contents} was specified
\item[\texttt{toc-title}]
title of table of contents (works only with EPUB, HTML, revealjs,
opendocument, odt, docx, pptx, beamer, LaTeX)
\end{description}

\section{Extensions}\label{extensions}

The behavior of some of the readers and writers can be adjusted by
enabling or disabling various extensions.

An extension can be enabled by adding \texttt{+EXTENSION} to the format
name and disabled by adding \texttt{-EXTENSION}. For example,
\texttt{-\/-from\ markdown\_strict+footnotes} is strict Markdown with
footnotes enabled, while
\texttt{-\/-from\ markdown-footnotes-pipe\_tables} is pandoc's Markdown
without footnotes or pipe tables.

The markdown reader and writer make by far the most use of extensions.
Extensions only used by them are therefore covered in the section
\hyperref[pandocs-markdown]{Pandoc's Markdown} below (see
\hyperref[markdown-variants]{Markdown variants} for \texttt{commonmark}
and \texttt{gfm}). In the following, extensions that also work for other
formats are covered.

Note that markdown extensions added to the \texttt{ipynb} format affect
Markdown cells in Jupyter notebooks (as do command-line options like
\texttt{-\/-atx-headers}).

\subsection{Typography}\label{typography}

\paragraph{\texorpdfstring{Extension:
\texttt{smart}}{Extension: smart}}\label{extension-smart}

Interpret straight quotes as curly quotes, \texttt{-\/-\/-} as
em-dashes, \texttt{-\/-} as en-dashes, and \texttt{...} as ellipses.
Nonbreaking spaces are inserted after certain abbreviations, such as
``Mr.''

This extension can be enabled/disabled for the following formats:

\begin{description}
\tightlist
\item[input formats]
\texttt{markdown}, \texttt{commonmark}, \texttt{latex},
\texttt{mediawiki}, \texttt{org}, \texttt{rst}, \texttt{twiki},
\texttt{html}
\item[output formats]
\texttt{markdown}, \texttt{latex}, \texttt{context}, \texttt{rst}
\item[enabled by default in]
\texttt{markdown}, \texttt{latex}, \texttt{context} (both input and
output)
\end{description}

Note: If you are \emph{writing} Markdown, then the \texttt{smart}
extension has the reverse effect: what would have been curly quotes
comes out straight.

In LaTeX, \texttt{smart} means to use the standard TeX ligatures for
quotation marks (\texttt{\textasciigrave{}\textasciigrave{}} and
\texttt{\textquotesingle{}\textquotesingle{}} for double quotes,
\texttt{\textasciigrave{}} and \texttt{\textquotesingle{}} for single
quotes) and dashes (\texttt{-\/-} for en-dash and \texttt{-\/-\/-} for
em-dash). If \texttt{smart} is disabled, then in reading LaTeX pandoc
will parse these characters literally. In writing LaTeX, enabling
\texttt{smart} tells pandoc to use the ligatures when possible; if
\texttt{smart} is disabled pandoc will use unicode quotation mark and
dash characters.

\subsection{Headings and sections}\label{headings-and-sections}

\paragraph{\texorpdfstring{Extension:
\texttt{auto\_identifiers}}{Extension: auto\_identifiers}}\label{extension-auto_identifiers}

A heading without an explicitly specified identifier will be
automatically assigned a unique identifier based on the heading text.

This extension can be enabled/disabled for the following formats:

\begin{description}
\tightlist
\item[input formats]
\texttt{markdown}, \texttt{latex}, \texttt{rst}, \texttt{mediawiki},
\texttt{textile}
\item[output formats]
\texttt{markdown}, \texttt{muse}
\item[enabled by default in]
\texttt{markdown}, \texttt{muse}
\end{description}

The default algorithm used to derive the identifier from the heading
text is:

\begin{itemize}
\tightlist
\item
  Remove all formatting, links, etc.
\item
  Remove all footnotes.
\item
  Remove all non-alphanumeric characters, except underscores, hyphens,
  and periods.
\item
  Replace all spaces and newlines with hyphens.
\item
  Convert all alphabetic characters to lowercase.
\item
  Remove everything up to the first letter (identifiers may not begin
  with a number or punctuation mark).
\item
  If nothing is left after this, use the identifier \texttt{section}.
\end{itemize}

Thus, for example,

\begin{longtable}[]{@{}ll@{}}
\toprule\noalign{}
Heading & Identifier \\
\midrule\noalign{}
\endhead
\bottomrule\noalign{}
\endlastfoot
\texttt{Heading\ identifiers\ in\ HTML} &
\texttt{heading-identifiers-in-html} \\
\texttt{Maître\ d\textquotesingle{}hôtel} & \texttt{maître-dhôtel} \\
\texttt{*Dogs*?-\/-in\ *my*\ house?} & \texttt{dogs-\/-in-my-house} \\
\texttt{{[}HTML{]},\ {[}S5{]},\ or\ {[}RTF{]}?} &
\texttt{html-s5-or-rtf} \\
\texttt{3.\ Applications} & \texttt{applications} \\
\texttt{33} & \texttt{section} \\
\end{longtable}

These rules should, in most cases, allow one to determine the identifier
from the heading text. The exception is when several headings have the
same text; in this case, the first will get an identifier as described
above; the second will get the same identifier with \texttt{-1}
appended; the third with \texttt{-2}; and so on.

(However, a different algorithm is used if
\texttt{gfm\_auto\_identifiers} is enabled; see below.)

These identifiers are used to provide link targets in the table of
contents generated by the
\texttt{-\/-toc\textbar{}-\/-table-of-contents} option. They also make
it easy to provide links from one section of a document to another. A
link to this section, for example, might look like this:

\begin{verbatim}
See the section on
[heading identifiers](#heading-identifiers-in-html-latex-and-context).
\end{verbatim}

Note, however, that this method of providing links to sections works
only in HTML, LaTeX, and ConTeXt formats.

If the \texttt{-\/-section-divs} option is specified, then each section
will be wrapped in a \texttt{section} (or a \texttt{div}, if
\texttt{html4} was specified), and the identifier will be attached to
the enclosing \texttt{\textless{}section\textgreater{}} (or
\texttt{\textless{}div\textgreater{}}) tag rather than the heading
itself. This allows entire sections to be manipulated using JavaScript
or treated differently in CSS.

\paragraph{\texorpdfstring{Extension:
\texttt{ascii\_identifiers}}{Extension: ascii\_identifiers}}\label{extension-ascii_identifiers}

Causes the identifiers produced by \texttt{auto\_identifiers} to be pure
ASCII. Accents are stripped off of accented Latin letters, and non-Latin
letters are omitted.

\paragraph{\texorpdfstring{Extension:
\texttt{gfm\_auto\_identifiers}}{Extension: gfm\_auto\_identifiers}}\label{extension-gfm_auto_identifiers}

Changes the algorithm used by \texttt{auto\_identifiers} to conform to
GitHub's method. Spaces are converted to dashes (\texttt{-}), uppercase
characters to lowercase characters, and punctuation characters other
than \texttt{-} and \texttt{\_} are removed. Emojis are replaced by
their names.

\subsection{Math Input}\label{math-input}

The extensions
\hyperref[extension-tex_math_dollars]{\texttt{tex\_math\_dollars}},
\hyperref[extension-tex_math_single_backslash]{\texttt{tex\_math\_single\_backslash}},
and
\hyperref[extension-tex_math_double_backslash]{\texttt{tex\_math\_double\_backslash}}
are described in the section about Pandoc's Markdown.

However, they can also be used with HTML input. This is handy for
reading web pages formatted using MathJax, for example.

\subsection{Raw HTML/TeX}\label{raw-htmltex}

The following extensions are described in more detail in their
respective sections of \hyperref[pandocs-markdown]{Pandoc's Markdown}:

\begin{itemize}
\item
  \hyperref[extension-raw_html]{\texttt{raw\_html}} allows HTML elements
  which are not representable in pandoc's AST to be parsed as raw HTML.
  By default, this is disabled for HTML input.
\item
  \hyperref[extension-raw_tex]{\texttt{raw\_tex}} allows raw LaTeX, TeX,
  and ConTeXt to be included in a document. This extension can be
  enabled/disabled for the following formats (in addition to
  \texttt{markdown}):

  \begin{description}
  \tightlist
  \item[input formats]
  \texttt{latex}, \texttt{textile}, \texttt{html} (environments,
  \texttt{\textbackslash{}ref}, and \texttt{\textbackslash{}eqref}
  only), \texttt{ipynb}
  \item[output formats]
  \texttt{textile}, \texttt{commonmark}
  \end{description}

  Note: as applied to \texttt{ipynb}, \texttt{raw\_html} and
  \texttt{raw\_tex} affect not only raw TeX in markdown cells, but data
  with mime type \texttt{text/html} in output cells. Since the
  \texttt{ipynb} reader attempts to preserve the richest possible
  outputs when several options are given, you will get best results if
  you disable \texttt{raw\_html} and \texttt{raw\_tex} when converting
  to formats like \texttt{docx} which don't allow raw \texttt{html} or
  \texttt{tex}.
\item
  \hyperref[extension-native_divs]{\texttt{native\_divs}} causes HTML
  \texttt{div} elements to be parsed as native pandoc Div blocks. If you
  want them to be parsed as raw HTML, use
  \texttt{-f\ html-native\_divs+raw\_html}.
\item
  \hyperref[extension-native_spans]{\texttt{native\_spans}} causes HTML
  \texttt{span} elements to be parsed as native pandoc Span inlines. If
  you want them to be parsed as raw HTML, use
  \texttt{-f\ html-native\_spans+raw\_html}. If you want to drop all
  \texttt{div}s and \texttt{span}s when converting HTML to Markdown, you
  can use
  \texttt{pandoc\ -f\ html-native\_divs-native\_spans\ -t\ markdown}.
\end{itemize}

\subsection{Literate Haskell support}\label{literate-haskell-support}

\paragraph{\texorpdfstring{Extension:
\texttt{literate\_haskell}}{Extension: literate\_haskell}}\label{extension-literate_haskell}

Treat the document as literate Haskell source.

This extension can be enabled/disabled for the following formats:

\begin{description}
\tightlist
\item[input formats]
\texttt{markdown}, \texttt{rst}, \texttt{latex}
\item[output formats]
\texttt{markdown}, \texttt{rst}, \texttt{latex}, \texttt{html}
\end{description}

If you append \texttt{+lhs} (or \texttt{+literate\_haskell}) to one of
the formats above, pandoc will treat the document as literate Haskell
source. This means that

\begin{itemize}
\item
  In Markdown input, ``bird track'' sections will be parsed as Haskell
  code rather than block quotations. Text between
  \texttt{\textbackslash{}begin\{code\}} and
  \texttt{\textbackslash{}end\{code\}} will also be treated as Haskell
  code. For ATX-style headings the character `=' will be used instead of
  `\#'.
\item
  In Markdown output, code blocks with classes \texttt{haskell} and
  \texttt{literate} will be rendered using bird tracks, and block
  quotations will be indented one space, so they will not be treated as
  Haskell code. In addition, headings will be rendered setext-style
  (with underlines) rather than ATX-style (with `\#' characters). (This
  is because ghc treats `\#' characters in column 1 as introducing line
  numbers.)
\item
  In restructured text input, ``bird track'' sections will be parsed as
  Haskell code.
\item
  In restructured text output, code blocks with class \texttt{haskell}
  will be rendered using bird tracks.
\item
  In LaTeX input, text in \texttt{code} environments will be parsed as
  Haskell code.
\item
  In LaTeX output, code blocks with class \texttt{haskell} will be
  rendered inside \texttt{code} environments.
\item
  In HTML output, code blocks with class \texttt{haskell} will be
  rendered with class \texttt{literatehaskell} and bird tracks.
\end{itemize}

Examples:

\begin{verbatim}
pandoc -f markdown+lhs -t html
\end{verbatim}

reads literate Haskell source formatted with Markdown conventions and
writes ordinary HTML (without bird tracks).

\begin{verbatim}
pandoc -f markdown+lhs -t html+lhs
\end{verbatim}

writes HTML with the Haskell code in bird tracks, so it can be copied
and pasted as literate Haskell source.

Note that GHC expects the bird tracks in the first column, so indented
literate code blocks (e.g.~inside an itemized environment) will not be
picked up by the Haskell compiler.

\subsection{Other extensions}\label{other-extensions}

\paragraph{\texorpdfstring{Extension:
\texttt{empty\_paragraphs}}{Extension: empty\_paragraphs}}\label{extension-empty_paragraphs}

Allows empty paragraphs. By default empty paragraphs are omitted.

This extension can be enabled/disabled for the following formats:

\begin{description}
\tightlist
\item[input formats]
\texttt{docx}, \texttt{html}
\item[output formats]
\texttt{docx}, \texttt{odt}, \texttt{opendocument}, \texttt{html}
\end{description}

\paragraph{\texorpdfstring{Extension:
\texttt{native\_numbering}}{Extension: native\_numbering}}\label{extension-native_numbering}

Enables native numbering of figures and tables. Enumeration starts at 1.

This extension can be enabled/disabled for the following formats:

\begin{description}
\tightlist
\item[output formats]
\texttt{odt}, \texttt{opendocument}, \texttt{docx}
\end{description}

\paragraph{\texorpdfstring{Extension:
\texttt{xrefs\_name}}{Extension: xrefs\_name}}\label{extension-xrefs_name}

Links to headings, figures and tables inside the document are
substituted with cross-references that will use the name or caption of
the referenced item. The original link text is replaced once the
generated document is refreshed. This extension can be combined with
\texttt{xrefs\_number} in which case numbers will appear before the
name.

Text in cross-references is only made consistent with the referenced
item once the document has been refreshed.

This extension can be enabled/disabled for the following formats:

\begin{description}
\tightlist
\item[output formats]
\texttt{odt}, \texttt{opendocument}
\end{description}

\paragraph{\texorpdfstring{Extension:
\texttt{xrefs\_number}}{Extension: xrefs\_number}}\label{extension-xrefs_number}

Links to headings, figures and tables inside the document are
substituted with cross-references that will use the number of the
referenced item. The original link text is discarded. This extension can
be combined with \texttt{xrefs\_name} in which case the name or caption
numbers will appear after the number.

For the \texttt{xrefs\_number} to be useful heading numbers must be
enabled in the generated document, also table and figure captions must
be enabled using for example the \texttt{native\_numbering} extension.

Numbers in cross-references are only visible in the final document once
it has been refreshed.

This extension can be enabled/disabled for the following formats:

\begin{description}
\tightlist
\item[output formats]
\texttt{odt}, \texttt{opendocument}
\end{description}

\paragraph{\texorpdfstring{Extension:
\texttt{styles}}{Extension: styles}}\label{ext-styles}

When converting from docx, read all docx styles as divs (for paragraph
styles) and spans (for character styles) regardless of whether pandoc
understands the meaning of these styles. This can be used with
\hyperref[custom-styles]{docx custom styles}. Disabled by default.

\begin{description}
\tightlist
\item[input formats]
\texttt{docx}
\end{description}

\paragraph{\texorpdfstring{Extension:
\texttt{amuse}}{Extension: amuse}}\label{extension-amuse}

In the \texttt{muse} input format, this enables Text::Amuse extensions
to Emacs Muse markup.

\paragraph{\texorpdfstring{Extension:
\texttt{raw\_markdown}}{Extension: raw\_markdown}}\label{extension-raw_markdown}

In the \texttt{ipynb} input format, this causes Markdown cells to be
included as raw Markdown blocks (allowing lossless round-tripping)
rather than being parsed. Use this only when you are targeting
\texttt{ipynb} or a markdown-based output format.

\paragraph{\texorpdfstring{Extension:
\texttt{citations}}{Extension: citations}}\label{org-citations}

When the \texttt{citations} extension is enabled in \texttt{org},
org-cite and org-ref style citations will be parsed as native pandoc
citations.

When \texttt{citations} is enabled in \texttt{docx}, citations inserted
by Zotero or Mendeley or EndNote plugins will be parsed as native pandoc
citations. (Otherwise, the formatted citations generated by the
bibliographic software will be parsed as regular text.)

\paragraph{\texorpdfstring{Extension:
\texttt{fancy\_lists}}{Extension: fancy\_lists}}\label{org-fancy-lists}

Some aspects of \hyperref[extension-fancy_lists]{Pandoc's Markdown fancy
lists} are also accepted in \texttt{org} input, mimicking the option
\texttt{org-list-allow-alphabetical} in Emacs. As in Org Mode, enabling
this extension allows lowercase and uppercase alphabetical markers for
ordered lists to be parsed in addition to arabic ones. Note that for
Org, this does not include roman numerals or the \texttt{\#} placeholder
that are enabled by the extension in Pandoc's Markdown.

\paragraph{\texorpdfstring{Extension:
\texttt{element\_citations}}{Extension: element\_citations}}\label{extension-element_citations}

In the \texttt{jats} output formats, this causes reference items to be
replaced with \texttt{\textless{}element-citation\textgreater{}}
elements. These elements are not influenced by CSL styles, but all
information on the item is included in tags.

\paragraph{\texorpdfstring{Extension:
\texttt{ntb}}{Extension: ntb}}\label{extension-ntb}

In the \texttt{context} output format this enables the use of
\href{https://wiki.contextgarden.net/TABLE}{Natural Tables (TABLE)}
instead of the default
\href{https://wiki.contextgarden.net/xtables}{Extreme Tables (xtables)}.
Natural tables allow more fine-grained global customization but come at
a performance penalty compared to extreme tables.

\section{Pandoc's Markdown}\label{pandocs-markdown}

Pandoc understands an extended and slightly revised version of John
Gruber's \href{https://daringfireball.net/projects/markdown/}{Markdown}
syntax. This document explains the syntax, noting differences from
original Markdown. Except where noted, these differences can be
suppressed by using the \texttt{markdown\_strict} format instead of
\texttt{markdown}. Extensions can be enabled or disabled to specify the
behavior more granularly. They are described in the following. See also
\hyperref[extensions]{Extensions} above, for extensions that work also
on other formats.

\subsection{Philosophy}\label{philosophy}

Markdown is designed to be easy to write, and, even more importantly,
easy to read:

\begin{quote}
A Markdown-formatted document should be publishable as-is, as plain
text, without looking like it's been marked up with tags or formatting
instructions. --
\href{https://daringfireball.net/projects/markdown/syntax\#philosophy}{John
Gruber}
\end{quote}

This principle has guided pandoc's decisions in finding syntax for
tables, footnotes, and other extensions.

There is, however, one respect in which pandoc's aims are different from
the original aims of Markdown. Whereas Markdown was originally designed
with HTML generation in mind, pandoc is designed for multiple output
formats. Thus, while pandoc allows the embedding of raw HTML, it
discourages it, and provides other, non-HTMLish ways of representing
important document elements like definition lists, tables, mathematics,
and footnotes.

\subsection{Paragraphs}\label{paragraphs}

A paragraph is one or more lines of text followed by one or more blank
lines. Newlines are treated as spaces, so you can reflow your paragraphs
as you like. If you need a hard line break, put two or more spaces at
the end of a line.

\paragraph{\texorpdfstring{Extension:
\texttt{escaped\_line\_breaks}}{Extension: escaped\_line\_breaks}}\label{extension-escaped_line_breaks}

A backslash followed by a newline is also a hard line break. Note: in
multiline and grid table cells, this is the only way to create a hard
line break, since trailing spaces in the cells are ignored.

\subsection{Headings}\label{headings}

There are two kinds of headings: Setext and ATX.

\subsubsection{Setext-style headings}\label{setext-style-headings}

A setext-style heading is a line of text ``underlined'' with a row of
\texttt{=} signs (for a level-one heading) or \texttt{-} signs (for a
level-two heading):

\begin{verbatim}
A level-one heading
===================

A level-two heading
-------------------
\end{verbatim}

The heading text can contain inline formatting, such as emphasis (see
\hyperref[inline-formatting]{Inline formatting}, below).

\subsubsection{ATX-style headings}\label{atx-style-headings}

An ATX-style heading consists of one to six \texttt{\#} signs and a line
of text, optionally followed by any number of \texttt{\#} signs. The
number of \texttt{\#} signs at the beginning of the line is the heading
level:

\begin{verbatim}
## A level-two heading

### A level-three heading ###
\end{verbatim}

As with setext-style headings, the heading text can contain formatting:

\begin{verbatim}
# A level-one heading with a [link](/url) and *emphasis*
\end{verbatim}

\paragraph{\texorpdfstring{Extension:
\texttt{blank\_before\_header}}{Extension: blank\_before\_header}}\label{extension-blank_before_header}

Original Markdown syntax does not require a blank line before a heading.
Pandoc does require this (except, of course, at the beginning of the
document). The reason for the requirement is that it is all too easy for
a \texttt{\#} to end up at the beginning of a line by accident (perhaps
through line wrapping). Consider, for example:

\begin{verbatim}
I like several of their flavors of ice cream:
#22, for example, and #5.
\end{verbatim}

\paragraph{\texorpdfstring{Extension:
\texttt{space\_in\_atx\_header}}{Extension: space\_in\_atx\_header}}\label{extension-space_in_atx_header}

Many Markdown implementations do not require a space between the opening
\texttt{\#}s of an ATX heading and the heading text, so that
\texttt{\#5\ bolt} and \texttt{\#hashtag} count as headings. With this
extension, pandoc does require the space.

\subsubsection{Heading identifiers}\label{heading-identifiers}

See also the
\hyperref[extension-auto_identifiers]{\texttt{auto\_identifiers}
extension} above.

\paragraph{\texorpdfstring{Extension:
\texttt{header\_attributes}}{Extension: header\_attributes}}\label{extension-header_attributes}

Headings can be assigned attributes using this syntax at the end of the
line containing the heading text:

\begin{verbatim}
{#identifier .class .class key=value key=value}
\end{verbatim}

Thus, for example, the following headings will all be assigned the
identifier \texttt{foo}:

\begin{verbatim}
# My heading {#foo}

## My heading ##    {#foo}

My other heading   {#foo}
---------------
\end{verbatim}

(This syntax is compatible with
\href{https://michelf.ca/projects/php-markdown/extra/}{PHP Markdown
Extra}.)

Note that although this syntax allows assignment of classes and
key/value attributes, writers generally don't use all of this
information. Identifiers, classes, and key/value attributes are used in
HTML and HTML-based formats such as EPUB and slidy. Identifiers are used
for labels and link anchors in the LaTeX, ConTeXt, Textile, Jira markup,
and AsciiDoc writers.

Headings with the class \texttt{unnumbered} will not be numbered, even
if \texttt{-\/-number-sections} is specified. A single hyphen
(\texttt{-}) in an attribute context is equivalent to
\texttt{.unnumbered}, and preferable in non-English documents. So,

\begin{verbatim}
# My heading {-}
\end{verbatim}

is just the same as

\begin{verbatim}
# My heading {.unnumbered}
\end{verbatim}

If the \texttt{unlisted} class is present in addition to
\texttt{unnumbered}, the heading will not be included in a table of
contents. (Currently this feature is only implemented for certain
formats: those based on LaTeX and HTML, PowerPoint, and RTF.)

\paragraph{\texorpdfstring{Extension:
\texttt{implicit\_header\_references}}{Extension: implicit\_header\_references}}\label{extension-implicit_header_references}

Pandoc behaves as if reference links have been defined for each heading.
So, to link to a heading

\begin{verbatim}
# Heading identifiers in HTML
\end{verbatim}

you can simply write

\begin{verbatim}
[Heading identifiers in HTML]
\end{verbatim}

or

\begin{verbatim}
[Heading identifiers in HTML][]
\end{verbatim}

or

\begin{verbatim}
[the section on heading identifiers][heading identifiers in
HTML]
\end{verbatim}

instead of giving the identifier explicitly:

\begin{verbatim}
[Heading identifiers in HTML](#heading-identifiers-in-html)
\end{verbatim}

If there are multiple headings with identical text, the corresponding
reference will link to the first one only, and you will need to use
explicit links to link to the others, as described above.

Like regular reference links, these references are case-insensitive.

Explicit link reference definitions always take priority over implicit
heading references. So, in the following example, the link will point to
\texttt{bar}, not to \texttt{\#foo}:

\begin{verbatim}
# Foo

[foo]: bar

See [foo]
\end{verbatim}

\subsection{Block quotations}\label{block-quotations}

Markdown uses email conventions for quoting blocks of text. A block
quotation is one or more paragraphs or other block elements (such as
lists or headings), with each line preceded by a \texttt{\textgreater{}}
character and an optional space. (The \texttt{\textgreater{}} need not
start at the left margin, but it should not be indented more than three
spaces.)

\begin{verbatim}
> This is a block quote. This
> paragraph has two lines.
>
> 1. This is a list inside a block quote.
> 2. Second item.
\end{verbatim}

A ``lazy'' form, which requires the \texttt{\textgreater{}} character
only on the first line of each block, is also allowed:

\begin{verbatim}
> This is a block quote. This
paragraph has two lines.

> 1. This is a list inside a block quote.
2. Second item.
\end{verbatim}

Among the block elements that can be contained in a block quote are
other block quotes. That is, block quotes can be nested:

\begin{verbatim}
> This is a block quote.
>
> > A block quote within a block quote.
\end{verbatim}

If the \texttt{\textgreater{}} character is followed by an optional
space, that space will be considered part of the block quote marker and
not part of the indentation of the contents. Thus, to put an indented
code block in a block quote, you need five spaces after the
\texttt{\textgreater{}}:

\begin{verbatim}
>     code
\end{verbatim}

\paragraph{\texorpdfstring{Extension:
\texttt{blank\_before\_blockquote}}{Extension: blank\_before\_blockquote}}\label{extension-blank_before_blockquote}

Original Markdown syntax does not require a blank line before a block
quote. Pandoc does require this (except, of course, at the beginning of
the document). The reason for the requirement is that it is all too easy
for a \texttt{\textgreater{}} to end up at the beginning of a line by
accident (perhaps through line wrapping). So, unless the
\texttt{markdown\_strict} format is used, the following does not produce
a nested block quote in pandoc:

\begin{verbatim}
> This is a block quote.
>> Nested.
\end{verbatim}

\subsection{Verbatim (code) blocks}\label{verbatim-code-blocks}

\subsubsection{Indented code blocks}\label{indented-code-blocks}

A block of text indented four spaces (or one tab) is treated as verbatim
text: that is, special characters do not trigger special formatting, and
all spaces and line breaks are preserved. For example,

\begin{verbatim}
    if (a > 3) {
      moveShip(5 * gravity, DOWN);
    }
\end{verbatim}

The initial (four space or one tab) indentation is not considered part
of the verbatim text, and is removed in the output.

Note: blank lines in the verbatim text need not begin with four spaces.

\subsubsection{Fenced code blocks}\label{fenced-code-blocks}

\paragraph{\texorpdfstring{Extension:
\texttt{fenced\_code\_blocks}}{Extension: fenced\_code\_blocks}}\label{extension-fenced_code_blocks}

In addition to standard indented code blocks, pandoc supports
\emph{fenced} code blocks. These begin with a row of three or more
tildes (\texttt{\textasciitilde{}}) and end with a row of tildes that
must be at least as long as the starting row. Everything between these
lines is treated as code. No indentation is necessary:

\begin{verbatim}
~~~~~~~
if (a > 3) {
  moveShip(5 * gravity, DOWN);
}
~~~~~~~
\end{verbatim}

Like regular code blocks, fenced code blocks must be separated from
surrounding text by blank lines.

If the code itself contains a row of tildes or backticks, just use a
longer row of tildes or backticks at the start and end:

\begin{verbatim}
~~~~~~~~~~~~~~~~
~~~~~~~~~~
code including tildes
~~~~~~~~~~
~~~~~~~~~~~~~~~~
\end{verbatim}

\paragraph{\texorpdfstring{Extension:
\texttt{backtick\_code\_blocks}}{Extension: backtick\_code\_blocks}}\label{extension-backtick_code_blocks}

Same as \texttt{fenced\_code\_blocks}, but uses backticks
(\texttt{\textasciigrave{}}) instead of tildes
(\texttt{\textasciitilde{}}).

\paragraph{\texorpdfstring{Extension:
\texttt{fenced\_code\_attributes}}{Extension: fenced\_code\_attributes}}\label{extension-fenced_code_attributes}

Optionally, you may attach attributes to fenced or backtick code block
using this syntax:

\begin{verbatim}
~~~~ {#mycode .haskell .numberLines startFrom="100"}
qsort []     = []
qsort (x:xs) = qsort (filter (< x) xs) ++ [x] ++
               qsort (filter (>= x) xs)
~~~~~~~~~~~~~~~~~~~~~~~~~~~~~~~~~~~~~~~~~~~~~~~~~
\end{verbatim}

Here \texttt{mycode} is an identifier, \texttt{haskell} and
\texttt{numberLines} are classes, and \texttt{startFrom} is an attribute
with value \texttt{100}. Some output formats can use this information to
do syntax highlighting. Currently, the only output formats that use this
information are HTML, LaTeX, Docx, Ms, and PowerPoint. If highlighting
is supported for your output format and language, then the code block
above will appear highlighted, with numbered lines. (To see which
languages are supported, type
\texttt{pandoc\ -\/-list-highlight-languages}.) Otherwise, the code
block above will appear as follows:

\begin{verbatim}
<pre id="mycode" class="haskell numberLines" startFrom="100">
  <code>
  ...
  </code>
</pre>
\end{verbatim}

The \texttt{numberLines} (or \texttt{number-lines}) class will cause the
lines of the code block to be numbered, starting with \texttt{1} or the
value of the \texttt{startFrom} attribute. The \texttt{lineAnchors} (or
\texttt{line-anchors}) class will cause the lines to be clickable
anchors in HTML output.

A shortcut form can also be used for specifying the language of the code
block:

\begin{verbatim}
```haskell
qsort [] = []
```
\end{verbatim}

This is equivalent to:

\begin{verbatim}
``` {.haskell}
qsort [] = []
```
\end{verbatim}

If the \texttt{fenced\_code\_attributes} extension is disabled, but
input contains class attribute(s) for the code block, the first class
attribute will be printed after the opening fence as a bare word.

To prevent all highlighting, use the \texttt{-\/-no-highlight} flag. To
set the highlighting style, use \texttt{-\/-highlight-style}. For more
information on highlighting, see \hyperref[syntax-highlighting]{Syntax
highlighting}, below.

\subsection{Line blocks}\label{line-blocks}

\paragraph{\texorpdfstring{Extension:
\texttt{line\_blocks}}{Extension: line\_blocks}}\label{extension-line_blocks}

A line block is a sequence of lines beginning with a vertical bar
(\texttt{\textbar{}}) followed by a space. The division into lines will
be preserved in the output, as will any leading spaces; otherwise, the
lines will be formatted as Markdown. This is useful for verse and
addresses:

\begin{verbatim}
| The limerick packs laughs anatomical
| In space that is quite economical.
|    But the good ones I've seen
|    So seldom are clean
| And the clean ones so seldom are comical

| 200 Main St.
| Berkeley, CA 94718
\end{verbatim}

The lines can be hard-wrapped if needed, but the continuation line must
begin with a space.

\begin{verbatim}
| The Right Honorable Most Venerable and Righteous Samuel L.
  Constable, Jr.
| 200 Main St.
| Berkeley, CA 94718
\end{verbatim}

Inline formatting (such as emphasis) is allowed in the content, but not
block-level formatting (such as block quotes or lists).

This syntax is borrowed from
\href{https://docutils.sourceforge.io/docs/ref/rst/introduction.html}{reStructuredText}.

\subsection{Lists}\label{lists}

\subsubsection{Bullet lists}\label{bullet-lists}

A bullet list is a list of bulleted list items. A bulleted list item
begins with a bullet (\texttt{*}, \texttt{+}, or \texttt{-}). Here is a
simple example:

\begin{verbatim}
* one
* two
* three
\end{verbatim}

This will produce a ``compact'' list. If you want a ``loose'' list, in
which each item is formatted as a paragraph, put spaces between the
items:

\begin{verbatim}
* one

* two

* three
\end{verbatim}

The bullets need not be flush with the left margin; they may be indented
one, two, or three spaces. The bullet must be followed by whitespace.

List items look best if subsequent lines are flush with the first line
(after the bullet):

\begin{verbatim}
* here is my first
  list item.
* and my second.
\end{verbatim}

But Markdown also allows a ``lazy'' format:

\begin{verbatim}
* here is my first
list item.
* and my second.
\end{verbatim}

\subsubsection{Block content in list
items}\label{block-content-in-list-items}

A list item may contain multiple paragraphs and other block-level
content. However, subsequent paragraphs must be preceded by a blank line
and indented to line up with the first non-space content after the list
marker.

\begin{verbatim}
  * First paragraph.

    Continued.

  * Second paragraph. With a code block, which must be indented
    eight spaces:

        { code }
\end{verbatim}

Exception: if the list marker is followed by an indented code block,
which must begin 5 spaces after the list marker, then subsequent
paragraphs must begin two columns after the last character of the list
marker:

\begin{verbatim}
*     code

  continuation paragraph
\end{verbatim}

List items may include other lists. In this case the preceding blank
line is optional. The nested list must be indented to line up with the
first non-space character after the list marker of the containing list
item.

\begin{verbatim}
* fruits
  + apples
    - macintosh
    - red delicious
  + pears
  + peaches
* vegetables
  + broccoli
  + chard
\end{verbatim}

As noted above, Markdown allows you to write list items ``lazily,''
instead of indenting continuation lines. However, if there are multiple
paragraphs or other blocks in a list item, the first line of each must
be indented.

\begin{verbatim}
+ A lazy, lazy, list
item.

+ Another one; this looks
bad but is legal.

    Second paragraph of second
list item.
\end{verbatim}

\subsubsection{Ordered lists}\label{ordered-lists}

Ordered lists work just like bulleted lists, except that the items begin
with enumerators rather than bullets.

In original Markdown, enumerators are decimal numbers followed by a
period and a space. The numbers themselves are ignored, so there is no
difference between this list:

\begin{verbatim}
1.  one
2.  two
3.  three
\end{verbatim}

and this one:

\begin{verbatim}
5.  one
7.  two
1.  three
\end{verbatim}

\paragraph{\texorpdfstring{Extension:
\texttt{fancy\_lists}}{Extension: fancy\_lists}}\label{extension-fancy_lists}

Unlike original Markdown, pandoc allows ordered list items to be marked
with uppercase and lowercase letters and roman numerals, in addition to
Arabic numerals. List markers may be enclosed in parentheses or followed
by a single right-parenthesis or period. They must be separated from the
text that follows by at least one space, and, if the list marker is a
capital letter with a period, by at least two spaces.\footnote{The point
  of this rule is to ensure that normal paragraphs starting with
  people's initials, like

\begin{Verbatim}
B. Russell was an English philosopher.
\end{Verbatim}

  do not get treated as list items.

  This rule will not prevent

\begin{Verbatim}
(C) 2007 Joe Smith
\end{Verbatim}

  from being interpreted as a list item. In this case, a backslash
  escape can be used:

\begin{Verbatim}
(C\) 2007 Joe Smith
\end{Verbatim}
}

The \texttt{fancy\_lists} extension also allows `\texttt{\#}' to be used
as an ordered list marker in place of a numeral:

\begin{verbatim}
#. one
#. two
\end{verbatim}

\paragraph{\texorpdfstring{Extension:
\texttt{startnum}}{Extension: startnum}}\label{extension-startnum}

Pandoc also pays attention to the type of list marker used, and to the
starting number, and both of these are preserved where possible in the
output format. Thus, the following yields a list with numbers followed
by a single parenthesis, starting with 9, and a sublist with lowercase
roman numerals:

\begin{verbatim}
 9)  Ninth
10)  Tenth
11)  Eleventh
       i. subone
      ii. subtwo
     iii. subthree
\end{verbatim}

Pandoc will start a new list each time a different type of list marker
is used. So, the following will create three lists:

\begin{verbatim}
(2) Two
(5) Three
1.  Four
*   Five
\end{verbatim}

If default list markers are desired, use \texttt{\#.}:

\begin{verbatim}
#.  one
#.  two
#.  three
\end{verbatim}

\paragraph{\texorpdfstring{Extension:
\texttt{task\_lists}}{Extension: task\_lists}}\label{extension-task_lists}

Pandoc supports task lists, using the syntax of GitHub-Flavored
Markdown.

\begin{verbatim}
- [ ] an unchecked task list item
- [x] checked item
\end{verbatim}

\subsubsection{Definition lists}\label{definition-lists}

\paragraph{\texorpdfstring{Extension:
\texttt{definition\_lists}}{Extension: definition\_lists}}\label{extension-definition_lists}

Pandoc supports definition lists, using the syntax of
\href{https://michelf.ca/projects/php-markdown/extra/}{PHP Markdown
Extra} with some extensions.\footnote{I have been influenced by the
  suggestions of
  \href{https://justatheory.com/2009/02/modest-markdown-proposal/}{David
  Wheeler}.}

\begin{verbatim}
Term 1

:   Definition 1

Term 2 with *inline markup*

:   Definition 2

        { some code, part of Definition 2 }

    Third paragraph of definition 2.
\end{verbatim}

Each term must fit on one line, which may optionally be followed by a
blank line, and must be followed by one or more definitions. A
definition begins with a colon or tilde, which may be indented one or
two spaces.

A term may have multiple definitions, and each definition may consist of
one or more block elements (paragraph, code block, list, etc.), each
indented four spaces or one tab stop. The body of the definition (not
including the first line) should be indented four spaces. However, as
with other Markdown lists, you can ``lazily'' omit indentation except at
the beginning of a paragraph or other block element:

\begin{verbatim}
Term 1

:   Definition
with lazy continuation.

    Second paragraph of the definition.
\end{verbatim}

If you leave space before the definition (as in the example above), the
text of the definition will be treated as a paragraph. In some output
formats, this will mean greater spacing between term/definition pairs.
For a more compact definition list, omit the space before the
definition:

\begin{verbatim}
Term 1
  ~ Definition 1

Term 2
  ~ Definition 2a
  ~ Definition 2b
\end{verbatim}

Note that space between items in a definition list is required. (A
variant that loosens this requirement, but disallows ``lazy'' hard
wrapping, can be activated with \texttt{compact\_definition\_lists}: see
\hyperref[non-default-extensions]{Non-default extensions}, below.)

\subsubsection{Numbered example lists}\label{numbered-example-lists}

\paragraph{\texorpdfstring{Extension:
\texttt{example\_lists}}{Extension: example\_lists}}\label{extension-example_lists}

The special list marker \texttt{@} can be used for sequentially numbered
examples. The first list item with a \texttt{@} marker will be numbered
`1', the next `2', and so on, throughout the document. The numbered
examples need not occur in a single list; each new list using \texttt{@}
will take up where the last stopped. So, for example:

\begin{verbatim}
(@)  My first example will be numbered (1).
(@)  My second example will be numbered (2).

Explanation of examples.

(@)  My third example will be numbered (3).
\end{verbatim}

Numbered examples can be labeled and referred to elsewhere in the
document:

\begin{verbatim}
(@good)  This is a good example.

As (@good) illustrates, ...
\end{verbatim}

The label can be any string of alphanumeric characters, underscores, or
hyphens.

Note: continuation paragraphs in example lists must always be indented
four spaces, regardless of the length of the list marker. That is,
example lists always behave as if the \texttt{four\_space\_rule}
extension is set. This is because example labels tend to be long, and
indenting content to the first non-space character after the label would
be awkward.

\subsubsection{Ending a list}\label{ending-a-list}

What if you want to put an indented code block after a list?

\begin{verbatim}
-   item one
-   item two

    { my code block }
\end{verbatim}

Trouble! Here pandoc (like other Markdown implementations) will treat
\texttt{\{\ my\ code\ block\ \}} as the second paragraph of item two,
and not as a code block.

To ``cut off'' the list after item two, you can insert some non-indented
content, like an HTML comment, which won't produce visible output in any
format:

\begin{verbatim}
-   item one
-   item two

<!-- end of list -->

    { my code block }
\end{verbatim}

You can use the same trick if you want two consecutive lists instead of
one big list:

\begin{verbatim}
1.  one
2.  two
3.  three

<!-- -->

1.  uno
2.  dos
3.  tres
\end{verbatim}

\subsection{Horizontal rules}\label{horizontal-rules}

A line containing a row of three or more \texttt{*}, \texttt{-}, or
\texttt{\_} characters (optionally separated by spaces) produces a
horizontal rule:

\begin{verbatim}
*  *  *  *

---------------
\end{verbatim}

We strongly recommend that horizontal rules be separated from
surrounding text by blank lines. If a horizontal rule is not followed by
a blank line, pandoc may try to interpret the lines that follow as a
YAML metadata block or a table.

\subsection{Tables}\label{tables}

Four kinds of tables may be used. The first three kinds presuppose the
use of a fixed-width font, such as Courier. The fourth kind can be used
with proportionally spaced fonts, as it does not require lining up
columns.

\paragraph{\texorpdfstring{Extension:
\texttt{table\_captions}}{Extension: table\_captions}}\label{extension-table_captions}

A caption may optionally be provided with all 4 kinds of tables (as
illustrated in the examples below). A caption is a paragraph beginning
with the string \texttt{Table:} (or just \texttt{:}), which will be
stripped off. It may appear either before or after the table.

\paragraph{\texorpdfstring{Extension:
\texttt{simple\_tables}}{Extension: simple\_tables}}\label{extension-simple_tables}

Simple tables look like this:

\begin{verbatim}
  Right     Left     Center     Default
-------     ------ ----------   -------
     12     12        12            12
    123     123       123          123
      1     1          1             1

Table:  Demonstration of simple table syntax.
\end{verbatim}

The header and table rows must each fit on one line. Column alignments
are determined by the position of the header text relative to the dashed
line below it:\footnote{This scheme is due to Michel Fortin, who
  proposed it on the
  \href{http://six.pairlist.net/pipermail/markdown-discuss/2005-March/001097.html}{Markdown
  discussion list}.}

\begin{itemize}
\tightlist
\item
  If the dashed line is flush with the header text on the right side but
  extends beyond it on the left, the column is right-aligned.
\item
  If the dashed line is flush with the header text on the left side but
  extends beyond it on the right, the column is left-aligned.
\item
  If the dashed line extends beyond the header text on both sides, the
  column is centered.
\item
  If the dashed line is flush with the header text on both sides, the
  default alignment is used (in most cases, this will be left).
\end{itemize}

The table must end with a blank line, or a line of dashes followed by a
blank line.

The column header row may be omitted, provided a dashed line is used to
end the table. For example:

\begin{verbatim}
-------     ------ ----------   -------
     12     12        12             12
    123     123       123           123
      1     1          1              1
-------     ------ ----------   -------
\end{verbatim}

When the header row is omitted, column alignments are determined on the
basis of the first line of the table body. So, in the tables above, the
columns would be right, left, center, and right aligned, respectively.

\paragraph{\texorpdfstring{Extension:
\texttt{multiline\_tables}}{Extension: multiline\_tables}}\label{extension-multiline_tables}

Multiline tables allow header and table rows to span multiple lines of
text (but cells that span multiple columns or rows of the table are not
supported). Here is an example:

\begin{verbatim}
-------------------------------------------------------------
 Centered   Default           Right Left
  Header    Aligned         Aligned Aligned
----------- ------- --------------- -------------------------
   First    row                12.0 Example of a row that
                                    spans multiple lines.

  Second    row                 5.0 Here's another one. Note
                                    the blank line between
                                    rows.
-------------------------------------------------------------

Table: Here's the caption. It, too, may span
multiple lines.
\end{verbatim}

These work like simple tables, but with the following differences:

\begin{itemize}
\tightlist
\item
  They must begin with a row of dashes, before the header text (unless
  the header row is omitted).
\item
  They must end with a row of dashes, then a blank line.
\item
  The rows must be separated by blank lines.
\end{itemize}

In multiline tables, the table parser pays attention to the widths of
the columns, and the writers try to reproduce these relative widths in
the output. So, if you find that one of the columns is too narrow in the
output, try widening it in the Markdown source.

The header may be omitted in multiline tables as well as simple tables:

\begin{verbatim}
----------- ------- --------------- -------------------------
   First    row                12.0 Example of a row that
                                    spans multiple lines.

  Second    row                 5.0 Here's another one. Note
                                    the blank line between
                                    rows.
----------- ------- --------------- -------------------------

: Here's a multiline table without a header.
\end{verbatim}

It is possible for a multiline table to have just one row, but the row
should be followed by a blank line (and then the row of dashes that ends
the table), or the table may be interpreted as a simple table.

\paragraph{\texorpdfstring{Extension:
\texttt{grid\_tables}}{Extension: grid\_tables}}\label{extension-grid_tables}

Grid tables look like this:

\begin{verbatim}
: Sample grid table.

+---------------+---------------+--------------------+
| Fruit         | Price         | Advantages         |
+===============+===============+====================+
| Bananas       | $1.34         | - built-in wrapper |
|               |               | - bright color     |
+---------------+---------------+--------------------+
| Oranges       | $2.10         | - cures scurvy     |
|               |               | - tasty            |
+---------------+---------------+--------------------+
\end{verbatim}

The row of \texttt{=}s separates the header from the table body, and can
be omitted for a headerless table. The cells of grid tables may contain
arbitrary block elements (multiple paragraphs, code blocks, lists,
etc.). Cells that span multiple columns or rows are not supported. Grid
tables can be created easily using Emacs' table-mode
(\texttt{M-x\ table-insert}).

Alignments can be specified as with pipe tables, by putting colons at
the boundaries of the separator line after the header:

\begin{verbatim}
+---------------+---------------+--------------------+
| Right         | Left          | Centered           |
+==============:+:==============+:==================:+
| Bananas       | $1.34         | built-in wrapper   |
+---------------+---------------+--------------------+
\end{verbatim}

For headerless tables, the colons go on the top line instead:

\begin{verbatim}
+--------------:+:--------------+:------------------:+
| Right         | Left          | Centered           |
+---------------+---------------+--------------------+
\end{verbatim}

\subparagraph{Grid Table Limitations}\label{grid-table-limitations}

Pandoc does not support grid tables with row spans or column spans. This
means that neither variable numbers of columns across rows nor variable
numbers of rows across columns are supported by Pandoc. All grid tables
must have the same number of columns in each row, and the same number of
rows in each column. For example, the Docutils
\href{https://docutils.sourceforge.io/docs/ref/rst/restructuredtext.html\#grid-tables}{sample
grid tables} will not render as expected with Pandoc.

\paragraph{\texorpdfstring{Extension:
\texttt{pipe\_tables}}{Extension: pipe\_tables}}\label{extension-pipe_tables}

Pipe tables look like this:

\begin{verbatim}
| Right | Left | Default | Center |
|------:|:-----|---------|:------:|
|   12  |  12  |    12   |    12  |
|  123  |  123 |   123   |   123  |
|    1  |    1 |     1   |     1  |

  : Demonstration of pipe table syntax.
\end{verbatim}

The syntax is identical to
\href{https://michelf.ca/projects/php-markdown/extra/\#table}{PHP
Markdown Extra tables}. The beginning and ending pipe characters are
optional, but pipes are required between all columns. The colons
indicate column alignment as shown. The header cannot be omitted. To
simulate a headerless table, include a header with blank cells.

Since the pipes indicate column boundaries, columns need not be
vertically aligned, as they are in the above example. So, this is a
perfectly legal (though ugly) pipe table:

\begin{verbatim}
fruit| price
-----|-----:
apple|2.05
pear|1.37
orange|3.09
\end{verbatim}

The cells of pipe tables cannot contain block elements like paragraphs
and lists, and cannot span multiple lines. If any line of the markdown
source is longer than the column width (see \texttt{-\/-columns}), then
the table will take up the full text width and the cell contents will
wrap, with the relative cell widths determined by the number of dashes
in the line separating the table header from the table body. (For
example \texttt{-\/-\/-\textbar{}-} would make the first column 3/4 and
the second column 1/4 of the full text width.) On the other hand, if no
lines are wider than column width, then cell contents will not be
wrapped, and the cells will be sized to their contents.

Note: pandoc also recognizes pipe tables of the following form, as can
be produced by Emacs' orgtbl-mode:

\begin{verbatim}
| One | Two   |
|-----+-------|
| my  | table |
| is  | nice  |
\end{verbatim}

The difference is that \texttt{+} is used instead of
\texttt{\textbar{}}. Other orgtbl features are not supported. In
particular, to get non-default column alignment, you'll need to add
colons as above.

\subsection{Metadata blocks}\label{metadata-blocks}

\paragraph{\texorpdfstring{Extension:
\texttt{pandoc\_title\_block}}{Extension: pandoc\_title\_block}}\label{extension-pandoc_title_block}

If the file begins with a title block

\begin{verbatim}
% title
% author(s) (separated by semicolons)
% date
\end{verbatim}

it will be parsed as bibliographic information, not regular text. (It
will be used, for example, in the title of standalone LaTeX or HTML
output.) The block may contain just a title, a title and an author, or
all three elements. If you want to include an author but no title, or a
title and a date but no author, you need a blank line:

\begin{verbatim}
%
% Author
\end{verbatim}

\begin{verbatim}
% My title
%
% June 15, 2006
\end{verbatim}

The title may occupy multiple lines, but continuation lines must begin
with leading space, thus:

\begin{verbatim}
% My title
  on multiple lines
\end{verbatim}

If a document has multiple authors, the authors may be put on separate
lines with leading space, or separated by semicolons, or both. So, all
of the following are equivalent:

\begin{verbatim}
% Author One
  Author Two
\end{verbatim}

\begin{verbatim}
% Author One; Author Two
\end{verbatim}

\begin{verbatim}
% Author One;
  Author Two
\end{verbatim}

The date must fit on one line.

All three metadata fields may contain standard inline formatting
(italics, links, footnotes, etc.).

Title blocks will always be parsed, but they will affect the output only
when the \texttt{-\/-standalone} (\texttt{-s}) option is chosen. In HTML
output, titles will appear twice: once in the document head -- this is
the title that will appear at the top of the window in a browser -- and
once at the beginning of the document body. The title in the document
head can have an optional prefix attached (\texttt{-\/-title-prefix} or
\texttt{-T} option). The title in the body appears as an H1 element with
class ``title'', so it can be suppressed or reformatted with CSS. If a
title prefix is specified with \texttt{-T} and no title block appears in
the document, the title prefix will be used by itself as the HTML title.

The man page writer extracts a title, man page section number, and other
header and footer information from the title line. The title is assumed
to be the first word on the title line, which may optionally end with a
(single-digit) section number in parentheses. (There should be no space
between the title and the parentheses.) Anything after this is assumed
to be additional footer and header text. A single pipe character
(\texttt{\textbar{}}) should be used to separate the footer text from
the header text. Thus,

\begin{verbatim}
% PANDOC(1)
\end{verbatim}

will yield a man page with the title \texttt{PANDOC} and section 1.

\begin{verbatim}
% PANDOC(1) Pandoc User Manuals
\end{verbatim}

will also have ``Pandoc User Manuals'' in the footer.

\begin{verbatim}
% PANDOC(1) Pandoc User Manuals | Version 4.0
\end{verbatim}

will also have ``Version 4.0'' in the header.

\paragraph{\texorpdfstring{Extension:
\texttt{yaml\_metadata\_block}}{Extension: yaml\_metadata\_block}}\label{extension-yaml_metadata_block}

A \href{https://yaml.org/spec/1.2/spec.html}{YAML} metadata block is a
valid YAML object, delimited by a line of three hyphens
(\texttt{-\/-\/-}) at the top and a line of three hyphens
(\texttt{-\/-\/-}) or three dots (\texttt{...}) at the bottom. The
initial line \texttt{-\/-\/-} must not be followed by a blank line. A
YAML metadata block may occur anywhere in the document, but if it is not
at the beginning, it must be preceded by a blank line.

Note that, because of the way pandoc concatenates input files when
several are provided, you may also keep the metadata in a separate YAML
file and pass it to pandoc as an argument, along with your Markdown
files:

\begin{verbatim}
pandoc chap1.md chap2.md chap3.md metadata.yaml -s -o book.html
\end{verbatim}

Just be sure that the YAML file begins with \texttt{-\/-\/-} and ends
with \texttt{-\/-\/-} or \texttt{...}. Alternatively, you can use the
\texttt{-\/-metadata-file} option. Using that approach however, you
cannot reference content (like footnotes) from the main markdown input
document.

Metadata will be taken from the fields of the YAML object and added to
any existing document metadata. Metadata can contain lists and objects
(nested arbitrarily), but all string scalars will be interpreted as
Markdown. Fields with names ending in an underscore will be ignored by
pandoc. (They may be given a role by external processors.) Field names
must not be interpretable as YAML numbers or boolean values (so, for
example, \texttt{yes}, \texttt{True}, and \texttt{15} cannot be used as
field names).

A document may contain multiple metadata blocks. If two metadata blocks
attempt to set the same field, the value from the second block will be
taken.

Each metadata block is handled internally as an independent YAML
document. This means, for example, that any YAML anchors defined in a
block cannot be referenced in another block.

When pandoc is used with \texttt{-t\ markdown} to create a Markdown
document, a YAML metadata block will be produced only if the
\texttt{-s/-\/-standalone} option is used. All of the metadata will
appear in a single block at the beginning of the document.

Note that \href{https://yaml.org/spec/1.2/spec.html}{YAML} escaping
rules must be followed. Thus, for example, if a title contains a colon,
it must be quoted, and if it contains a backslash escape, then it must
be ensured that it is not treated as a
\href{https://yaml.org/spec/1.2/spec.html\#id2776092}{YAML escape
sequence}. The pipe character (\texttt{\textbar{}}) can be used to begin
an indented block that will be interpreted literally, without need for
escaping. This form is necessary when the field contains blank lines or
block-level formatting:

\begin{verbatim}
---
title:  'This is the title: it contains a colon'
author:
- Author One
- Author Two
keywords: [nothing, nothingness]
abstract: |
  This is the abstract.

  It consists of two paragraphs.
...
\end{verbatim}

The literal block after the \texttt{\textbar{}} must be indented
relative to the line containing the \texttt{\textbar{}}. If it is not,
the YAML will be invalid and pandoc will not interpret it as metadata.
For an overview of the complex rules governing YAML, see the
\href{https://en.wikipedia.org/wiki/YAML\#Syntax}{Wikipedia entry on
YAML syntax}.

Template variables will be set automatically from the metadata. Thus,
for example, in writing HTML, the variable \texttt{abstract} will be set
to the HTML equivalent of the Markdown in the \texttt{abstract} field:

\begin{verbatim}
<p>This is the abstract.</p>
<p>It consists of two paragraphs.</p>
\end{verbatim}

Variables can contain arbitrary YAML structures, but the template must
match this structure. The \texttt{author} variable in the default
templates expects a simple list or string, but can be changed to support
more complicated structures. The following combination, for example,
would add an affiliation to the author if one is given:

\begin{verbatim}
---
title: The document title
author:
- name: Author One
  affiliation: University of Somewhere
- name: Author Two
  affiliation: University of Nowhere
...
\end{verbatim}

To use the structured authors in the example above, you would need a
custom template:

\begin{verbatim}
$for(author)$
$if(author.name)$
$author.name$$if(author.affiliation)$ ($author.affiliation$)$endif$
$else$
$author$
$endif$
$endfor$
\end{verbatim}

Raw content to include in the document's header may be specified using
\texttt{header-includes}; however, it is important to mark up this
content as raw code for a particular output format, using the
\hyperref[extension-raw_attribute]{\texttt{raw\_attribute} extension},
or it will be interpreted as markdown. For example:

\begin{verbatim}
header-includes:
- |
  ```{=latex}
  \let\oldsection\section
  \renewcommand{\section}[1]{\clearpage\oldsection{#1}}
  ```
\end{verbatim}

Note: the \texttt{yaml\_metadata\_block} extension works with
\texttt{commonmark} as well as \texttt{markdown} (and it is enabled by
default in \texttt{gfm} and \texttt{commonmark\_x}). However, in these
formats the following restrictions apply:

\begin{itemize}
\item
  The YAML metadata block must occur at the beginning of the document
  (and there can be only one). If multiple files are given as arguments
  to pandoc, only the first can be a YAML metadata block.
\item
  The leaf nodes of the YAML structure are parsed in isolation from each
  other and from the rest of the document. So, for example, you can't
  use a reference link in these contexts if the link definition is
  somewhere else in the document.
\end{itemize}

\subsection{Backslash escapes}\label{backslash-escapes}

\paragraph{\texorpdfstring{Extension:
\texttt{all\_symbols\_escapable}}{Extension: all\_symbols\_escapable}}\label{extension-all_symbols_escapable}

Except inside a code block or inline code, any punctuation or space
character preceded by a backslash will be treated literally, even if it
would normally indicate formatting. Thus, for example, if one writes

\begin{verbatim}
*\*hello\**
\end{verbatim}

one will get

\begin{verbatim}
<em>*hello*</em>
\end{verbatim}

instead of

\begin{verbatim}
<strong>hello</strong>
\end{verbatim}

This rule is easier to remember than original Markdown's rule, which
allows only the following characters to be backslash-escaped:

\begin{verbatim}
\`*_{}[]()>#+-.!
\end{verbatim}

(However, if the \texttt{markdown\_strict} format is used, the original
Markdown rule will be used.)

A backslash-escaped space is parsed as a nonbreaking space. In TeX
output, it will appear as \texttt{\textasciitilde{}}. In HTML and XML
output, it will appear as a literal unicode nonbreaking space character
(note that it will thus actually look ``invisible'' in the generated
HTML source; you can still use the \texttt{-\/-ascii} command-line
option to make it appear as an explicit entity).

A backslash-escaped newline (i.e.~a backslash occurring at the end of a
line) is parsed as a hard line break. It will appear in TeX output as
\texttt{\textbackslash{}\textbackslash{}} and in HTML as
\texttt{\textless{}br\ /\textgreater{}}. This is a nice alternative to
Markdown's ``invisible'' way of indicating hard line breaks using two
trailing spaces on a line.

Backslash escapes do not work in verbatim contexts.

\subsection{Inline formatting}\label{inline-formatting}

\subsubsection{Emphasis}\label{emphasis}

To \emph{emphasize} some text, surround it with \texttt{*}s or
\texttt{\_}, like this:

\begin{verbatim}
This text is _emphasized with underscores_, and this
is *emphasized with asterisks*.
\end{verbatim}

Double \texttt{*} or \texttt{\_} produces \textbf{strong emphasis}:

\begin{verbatim}
This is **strong emphasis** and __with underscores__.
\end{verbatim}

A \texttt{*} or \texttt{\_} character surrounded by spaces, or
backslash-escaped, will not trigger emphasis:

\begin{verbatim}
This is * not emphasized *, and \*neither is this\*.
\end{verbatim}

\paragraph{\texorpdfstring{Extension:
\texttt{intraword\_underscores}}{Extension: intraword\_underscores}}\label{extension-intraword_underscores}

Because \texttt{\_} is sometimes used inside words and identifiers,
pandoc does not interpret a \texttt{\_} surrounded by alphanumeric
characters as an emphasis marker. If you want to emphasize just part of
a word, use \texttt{*}:

\begin{verbatim}
feas*ible*, not feas*able*.
\end{verbatim}

\subsubsection{Highlighting}\label{highlighting}

To highlight text, use the \texttt{mark} class:

\begin{verbatim}
[Mark]{.mark}
\end{verbatim}

Or, without the \texttt{bracketed\_spans} extension (but with
\texttt{native\_spans}):

\begin{verbatim}
<span class="mark">Mark</span>
\end{verbatim}

This will work in html output.

\subsubsection{Strikeout}\label{strikeout}

\paragraph{\texorpdfstring{Extension:
\texttt{strikeout}}{Extension: strikeout}}\label{extension-strikeout}

To strike out a section of text with a horizontal line, begin and end it
with \texttt{\textasciitilde{}\textasciitilde{}}. Thus, for example,

\begin{verbatim}
This ~~is deleted text.~~
\end{verbatim}

\subsubsection{Superscripts and
subscripts}\label{superscripts-and-subscripts}

\paragraph{\texorpdfstring{Extension: \texttt{superscript},
\texttt{subscript}}{Extension: superscript, subscript}}\label{extension-superscript-subscript}

Superscripts may be written by surrounding the superscripted text by
\texttt{\^{}} characters; subscripts may be written by surrounding the
subscripted text by \texttt{\textasciitilde{}} characters. Thus, for
example,

\begin{verbatim}
H~2~O is a liquid.  2^10^ is 1024.
\end{verbatim}

The text between \texttt{\^{}...\^{}} or
\texttt{\textasciitilde{}...\textasciitilde{}} may not contain spaces or
newlines. If the superscripted or subscripted text contains spaces,
these spaces must be escaped with backslashes. (This is to prevent
accidental superscripting and subscripting through the ordinary use of
\texttt{\textasciitilde{}} and \texttt{\^{}}, and also bad interactions
with footnotes.) Thus, if you want the letter P with `a cat' in
subscripts, use
\texttt{P\textasciitilde{}a\textbackslash{}\ cat\textasciitilde{}}, not
\texttt{P\textasciitilde{}a\ cat\textasciitilde{}}.

\subsubsection{Verbatim}\label{verbatim}

To make a short span of text verbatim, put it inside backticks:

\begin{verbatim}
What is the difference between `>>=` and `>>`?
\end{verbatim}

If the verbatim text includes a backtick, use double backticks:

\begin{verbatim}
Here is a literal backtick `` ` ``.
\end{verbatim}

(The spaces after the opening backticks and before the closing backticks
will be ignored.)

The general rule is that a verbatim span starts with a string of
consecutive backticks (optionally followed by a space) and ends with a
string of the same number of backticks (optionally preceded by a space).

Note that backslash-escapes (and other Markdown constructs) do not work
in verbatim contexts:

\begin{verbatim}
This is a backslash followed by an asterisk: `\*`.
\end{verbatim}

\paragraph{\texorpdfstring{Extension:
\texttt{inline\_code\_attributes}}{Extension: inline\_code\_attributes}}\label{extension-inline_code_attributes}

Attributes can be attached to verbatim text, just as with
\hyperref[fenced-code-blocks]{fenced code blocks}:

\begin{verbatim}
`<$>`{.haskell}
\end{verbatim}

\subsubsection{Underline}\label{underline}

To underline text, use the \texttt{underline} class:

\begin{verbatim}
[Underline]{.underline}
\end{verbatim}

Or, without the \texttt{bracketed\_spans} extension (but with
\texttt{native\_spans}):

\begin{verbatim}
<span class="underline">Underline</span>
\end{verbatim}

This will work in all output formats that support underline.

\subsubsection{Small caps}\label{small-caps}

To write small caps, use the \texttt{smallcaps} class:

\begin{verbatim}
[Small caps]{.smallcaps}
\end{verbatim}

Or, without the \texttt{bracketed\_spans} extension:

\begin{verbatim}
<span class="smallcaps">Small caps</span>
\end{verbatim}

For compatibility with other Markdown flavors, CSS is also supported:

\begin{verbatim}
<span style="font-variant:small-caps;">Small caps</span>
\end{verbatim}

This will work in all output formats that support small caps.

\subsection{Math}\label{math}

\paragraph{\texorpdfstring{Extension:
\texttt{tex\_math\_dollars}}{Extension: tex\_math\_dollars}}\label{extension-tex_math_dollars}

Anything between two \texttt{\$} characters will be treated as TeX math.
The opening \texttt{\$} must have a non-space character immediately to
its right, while the closing \texttt{\$} must have a non-space character
immediately to its left, and must not be followed immediately by a
digit. Thus, \texttt{\$20,000\ and\ \$30,000} won't parse as math. If
for some reason you need to enclose text in literal \texttt{\$}
characters, backslash-escape them and they won't be treated as math
delimiters.

For display math, use \texttt{\$\$} delimiters. (In this case, the
delimiters may be separated from the formula by whitespace. However,
there can be no blank lines between the opening and closing
\texttt{\$\$} delimiters.)

TeX math will be printed in all output formats. How it is rendered
depends on the output format:

\begin{description}
\tightlist
\item[LaTeX]
It will appear verbatim surrounded by
\texttt{\textbackslash{}(...\textbackslash{})} (for inline math) or
\texttt{\textbackslash{}{[}...\textbackslash{}{]}} (for display math).
\item[Markdown, Emacs Org mode, ConTeXt, ZimWiki]
It will appear verbatim surrounded by \texttt{\$...\$} (for inline math)
or \texttt{\$\$...\$\$} (for display math).
\item[XWiki]
It will appear verbatim surrounded by
\texttt{\{\{formula\}\}..\{\{/formula\}\}}.
\item[reStructuredText]
It will be rendered using an
\href{https://docutils.sourceforge.io/docs/ref/rst/roles.html\#math}{interpreted
text role \texttt{:math:}}.
\item[AsciiDoc]
For AsciiDoc output format (\texttt{-t\ asciidoc}) it will appear
verbatim surrounded by \texttt{latexmath:{[}\$...\${]}} (for inline
math) or
\texttt{{[}latexmath{]}++++\textbackslash{}{[}...\textbackslash{}{]}+++}
(for display math). For AsciiDoctor output format
(\texttt{-t\ asciidoctor}) the LaTeX delimiters (\texttt{\$..\$} and
\texttt{\textbackslash{}{[}..\textbackslash{}{]}}) are omitted.
\item[Texinfo]
It will be rendered inside a \texttt{@math} command.
\item[roff man, Jira markup]
It will be rendered verbatim without \texttt{\$}'s.
\item[MediaWiki, DokuWiki]
It will be rendered inside \texttt{\textless{}math\textgreater{}} tags.
\item[Textile]
It will be rendered inside
\texttt{\textless{}span\ class="math"\textgreater{}} tags.
\item[RTF, OpenDocument]
It will be rendered, if possible, using Unicode characters, and will
otherwise appear verbatim.
\item[ODT]
It will be rendered, if possible, using MathML.
\item[DocBook]
If the \texttt{-\/-mathml} flag is used, it will be rendered using
MathML in an \texttt{inlineequation} or \texttt{informalequation} tag.
Otherwise it will be rendered, if possible, using Unicode characters.
\item[Docx and PowerPoint]
It will be rendered using OMML math markup.
\item[FictionBook2]
If the \texttt{-\/-webtex} option is used, formulas are rendered as
images using CodeCogs or other compatible web service, downloaded and
embedded in the e-book. Otherwise, they will appear verbatim.
\item[HTML, Slidy, DZSlides, S5, EPUB]
The way math is rendered in HTML will depend on the command-line options
selected. Therefore see \hyperref[math-rendering-in-html]{Math rendering
in HTML} above.
\end{description}

\subsection{Raw HTML}\label{raw-html}

\paragraph{\texorpdfstring{Extension:
\texttt{raw\_html}}{Extension: raw\_html}}\label{extension-raw_html}

Markdown allows you to insert raw HTML (or DocBook) anywhere in a
document (except verbatim contexts, where \texttt{\textless{}},
\texttt{\textgreater{}}, and \texttt{\&} are interpreted literally).
(Technically this is not an extension, since standard Markdown allows
it, but it has been made an extension so that it can be disabled if
desired.)

The raw HTML is passed through unchanged in HTML, S5, Slidy, Slideous,
DZSlides, EPUB, Markdown, CommonMark, Emacs Org mode, and Textile
output, and suppressed in other formats.

For a more explicit way of including raw HTML in a Markdown document,
see the \hyperref[extension-raw_attribute]{\texttt{raw\_attribute}
extension}.

In the CommonMark format, if \texttt{raw\_html} is enabled,
superscripts, subscripts, strikeouts and small capitals will be
represented as HTML. Otherwise, plain-text fallbacks will be used. Note
that even if \texttt{raw\_html} is disabled, tables will be rendered
with HTML syntax if they cannot use pipe syntax.

\paragraph{\texorpdfstring{Extension:
\texttt{markdown\_in\_html\_blocks}}{Extension: markdown\_in\_html\_blocks}}\label{extension-markdown_in_html_blocks}

Original Markdown allows you to include HTML ``blocks'': blocks of HTML
between balanced tags that are separated from the surrounding text with
blank lines, and start and end at the left margin. Within these blocks,
everything is interpreted as HTML, not Markdown; so (for example),
\texttt{*} does not signify emphasis.

Pandoc behaves this way when the \texttt{markdown\_strict} format is
used; but by default, pandoc interprets material between HTML block tags
as Markdown. Thus, for example, pandoc will turn

\begin{verbatim}
<table>
<tr>
<td>*one*</td>
<td>[a link](https://google.com)</td>
</tr>
</table>
\end{verbatim}

into

\begin{verbatim}
<table>
<tr>
<td><em>one</em></td>
<td><a href="https://google.com">a link</a></td>
</tr>
</table>
\end{verbatim}

whereas \texttt{Markdown.pl} will preserve it as is.

There is one exception to this rule: text between
\texttt{\textless{}script\textgreater{}},
\texttt{\textless{}style\textgreater{}}, and
\texttt{\textless{}textarea\textgreater{}} tags is not interpreted as
Markdown.

This departure from original Markdown should make it easier to mix
Markdown with HTML block elements. For example, one can surround a block
of Markdown text with \texttt{\textless{}div\textgreater{}} tags without
preventing it from being interpreted as Markdown.

\paragraph{\texorpdfstring{Extension:
\texttt{native\_divs}}{Extension: native\_divs}}\label{extension-native_divs}

Use native pandoc \texttt{Div} blocks for content inside
\texttt{\textless{}div\textgreater{}} tags. For the most part this
should give the same output as \texttt{markdown\_in\_html\_blocks}, but
it makes it easier to write pandoc filters to manipulate groups of
blocks.

\paragraph{\texorpdfstring{Extension:
\texttt{native\_spans}}{Extension: native\_spans}}\label{extension-native_spans}

Use native pandoc \texttt{Span} blocks for content inside
\texttt{\textless{}span\textgreater{}} tags. For the most part this
should give the same output as \texttt{raw\_html}, but it makes it
easier to write pandoc filters to manipulate groups of inlines.

\paragraph{\texorpdfstring{Extension:
\texttt{raw\_tex}}{Extension: raw\_tex}}\label{extension-raw_tex}

In addition to raw HTML, pandoc allows raw LaTeX, TeX, and ConTeXt to be
included in a document. Inline TeX commands will be preserved and passed
unchanged to the LaTeX and ConTeXt writers. Thus, for example, you can
use LaTeX to include BibTeX citations:

\begin{verbatim}
This result was proved in \cite{jones.1967}.
\end{verbatim}

Note that in LaTeX environments, like

\begin{verbatim}
\begin{tabular}{|l|l|}\hline
Age & Frequency \\ \hline
18--25  & 15 \\
26--35  & 33 \\
36--45  & 22 \\ \hline
\end{tabular}
\end{verbatim}

the material between the begin and end tags will be interpreted as raw
LaTeX, not as Markdown.

For a more explicit and flexible way of including raw TeX in a Markdown
document, see the
\hyperref[extension-raw_attribute]{\texttt{raw\_attribute} extension}.

Inline LaTeX is ignored in output formats other than Markdown, LaTeX,
Emacs Org mode, and ConTeXt.

\subsubsection{Generic raw attribute}\label{generic-raw-attribute}

\paragraph{\texorpdfstring{Extension:
\texttt{raw\_attribute}}{Extension: raw\_attribute}}\label{extension-raw_attribute}

Inline spans and fenced code blocks with a special kind of attribute
will be parsed as raw content with the designated format. For example,
the following produces a raw roff \texttt{ms} block:

\begin{verbatim}
```{=ms}
.MYMACRO
blah blah
```
\end{verbatim}

And the following produces a raw \texttt{html} inline element:

\begin{verbatim}
This is `<a>html</a>`{=html}
\end{verbatim}

This can be useful to insert raw xml into \texttt{docx} documents, e.g.
a pagebreak:

\begin{verbatim}
```{=openxml}
<w:p>
  <w:r>
    <w:br w:type="page"/>
  </w:r>
</w:p>
```
\end{verbatim}

The format name should match the target format name (see
\texttt{-t/-\/-to}, above, for a list, or use
\texttt{pandoc\ -\/-list-output-formats}). Use \texttt{openxml} for
\texttt{docx} output, \texttt{opendocument} for \texttt{odt} output,
\texttt{html5} for \texttt{epub3} output, \texttt{html4} for
\texttt{epub2} output, and \texttt{latex}, \texttt{beamer}, \texttt{ms},
or \texttt{html5} for \texttt{pdf} output (depending on what you use for
\texttt{-\/-pdf-engine}).

This extension presupposes that the relevant kind of inline code or
fenced code block is enabled. Thus, for example, to use a raw attribute
with a backtick code block, \texttt{backtick\_code\_blocks} must be
enabled.

The raw attribute cannot be combined with regular attributes.

\subsection{LaTeX macros}\label{latex-macros}

\paragraph{\texorpdfstring{Extension:
\texttt{latex\_macros}}{Extension: latex\_macros}}\label{extension-latex_macros}

When this extension is enabled, pandoc will parse LaTeX macro
definitions and apply the resulting macros to all LaTeX math and raw
LaTeX. So, for example, the following will work in all output formats,
not just LaTeX:

\begin{verbatim}
\newcommand{\tuple}[1]{\langle #1 \rangle}

$\tuple{a, b, c}$
\end{verbatim}

Note that LaTeX macros will not be applied if they occur inside a raw
span or block marked with the
\hyperref[extension-raw_attribute]{\texttt{raw\_attribute} extension}.

When \texttt{latex\_macros} is disabled, the raw LaTeX and math will not
have macros applied. This is usually a better approach when you are
targeting LaTeX or PDF.

Macro definitions in LaTeX will be passed through as raw LaTeX only if
\texttt{latex\_macros} is not enabled. Macro definitions in Markdown
source (or other formats allowing \texttt{raw\_tex}) will be passed
through regardless of whether \texttt{latex\_macros} is enabled.

\subsection{Links}\label{links-1}

Markdown allows links to be specified in several ways.

\subsubsection{Automatic links}\label{automatic-links}

If you enclose a URL or email address in pointy brackets, it will become
a link:

\begin{verbatim}
<https://google.com>
<sam@green.eggs.ham>
\end{verbatim}

\subsubsection{Inline links}\label{inline-links}

An inline link consists of the link text in square brackets, followed by
the URL in parentheses. (Optionally, the URL can be followed by a link
title, in quotes.)

\begin{verbatim}
This is an [inline link](/url), and here's [one with
a title](https://fsf.org "click here for a good time!").
\end{verbatim}

There can be no space between the bracketed part and the parenthesized
part. The link text can contain formatting (such as emphasis), but the
title cannot.

Email addresses in inline links are not autodetected, so they have to be
prefixed with \texttt{mailto}:

\begin{verbatim}
[Write me!](mailto:sam@green.eggs.ham)
\end{verbatim}

\subsubsection{Reference links}\label{reference-links}

An \emph{explicit} reference link has two parts, the link itself and the
link definition, which may occur elsewhere in the document (either
before or after the link).

The link consists of link text in square brackets, followed by a label
in square brackets. (There cannot be space between the two unless the
\texttt{spaced\_reference\_links} extension is enabled.) The link
definition consists of the bracketed label, followed by a colon and a
space, followed by the URL, and optionally (after a space) a link title
either in quotes or in parentheses. The label must not be parseable as a
citation (assuming the \texttt{citations} extension is enabled):
citations take precedence over link labels.

Here are some examples:

\begin{verbatim}
[my label 1]: /foo/bar.html  "My title, optional"
[my label 2]: /foo
[my label 3]: https://fsf.org (The Free Software Foundation)
[my label 4]: /bar#special  'A title in single quotes'
\end{verbatim}

The URL may optionally be surrounded by angle brackets:

\begin{verbatim}
[my label 5]: <http://foo.bar.baz>
\end{verbatim}

The title may go on the next line:

\begin{verbatim}
[my label 3]: https://fsf.org
  "The Free Software Foundation"
\end{verbatim}

Note that link labels are not case sensitive. So, this will work:

\begin{verbatim}
Here is [my link][FOO]

[Foo]: /bar/baz
\end{verbatim}

In an \emph{implicit} reference link, the second pair of brackets is
empty:

\begin{verbatim}
See [my website][].

[my website]: http://foo.bar.baz
\end{verbatim}

Note: In \texttt{Markdown.pl} and most other Markdown implementations,
reference link definitions cannot occur in nested constructions such as
list items or block quotes. Pandoc lifts this arbitrary-seeming
restriction. So the following is fine in pandoc, though not in most
other implementations:

\begin{verbatim}
> My block [quote].
>
> [quote]: /foo
\end{verbatim}

\paragraph{\texorpdfstring{Extension:
\texttt{shortcut\_reference\_links}}{Extension: shortcut\_reference\_links}}\label{extension-shortcut_reference_links}

In a \emph{shortcut} reference link, the second pair of brackets may be
omitted entirely:

\begin{verbatim}
See [my website].

[my website]: http://foo.bar.baz
\end{verbatim}

\subsubsection{Internal links}\label{internal-links}

To link to another section of the same document, use the automatically
generated identifier (see \hyperref[heading-identifiers]{Heading
identifiers}). For example:

\begin{verbatim}
See the [Introduction](#introduction).
\end{verbatim}

or

\begin{verbatim}
See the [Introduction].

[Introduction]: #introduction
\end{verbatim}

Internal links are currently supported for HTML formats (including HTML
slide shows and EPUB), LaTeX, and ConTeXt.

\subsection{Images}\label{images}

A link immediately preceded by a \texttt{!} will be treated as an image.
The link text will be used as the image's alt text:

\begin{verbatim}
![la lune](lalune.jpg "Voyage to the moon")

![movie reel]

[movie reel]: movie.gif
\end{verbatim}

\paragraph{\texorpdfstring{Extension:
\texttt{implicit\_figures}}{Extension: implicit\_figures}}\label{extension-implicit_figures}

An image with nonempty alt text, occurring by itself in a paragraph,
will be rendered as a figure with a caption. The image's alt text will
be used as the caption.

\begin{verbatim}
![This is the caption](/url/of/image.png)
\end{verbatim}

How this is rendered depends on the output format. Some output formats
(e.g.~RTF) do not yet support figures. In those formats, you'll just get
an image in a paragraph by itself, with no caption.

If you just want a regular inline image, just make sure it is not the
only thing in the paragraph. One way to do this is to insert a
nonbreaking space after the image:

\begin{verbatim}
![This image won't be a figure](/url/of/image.png)\
\end{verbatim}

Note that in reveal.js slide shows, an image in a paragraph by itself
that has the \texttt{r-stretch} class will fill the screen, and the
caption and figure tags will be omitted.

\paragraph{\texorpdfstring{Extension:
\texttt{link\_attributes}}{Extension: link\_attributes}}\label{extension-link_attributes}

Attributes can be set on links and images:

\begin{verbatim}
An inline ![image](foo.jpg){#id .class width=30 height=20px}
and a reference ![image][ref] with attributes.

[ref]: foo.jpg "optional title" {#id .class key=val key2="val 2"}
\end{verbatim}

(This syntax is compatible with
\href{https://michelf.ca/projects/php-markdown/extra/}{PHP Markdown
Extra} when only \texttt{\#id} and \texttt{.class} are used.)

For HTML and EPUB, all known HTML5 attributes except \texttt{width} and
\texttt{height} (but including \texttt{srcset} and \texttt{sizes}) are
passed through as is. Unknown attributes are passed through as custom
attributes, with \texttt{data-} prepended. The other writers ignore
attributes that are not specifically supported by their output format.

The \texttt{width} and \texttt{height} attributes on images are treated
specially. When used without a unit, the unit is assumed to be pixels.
However, any of the following unit identifiers can be used: \texttt{px},
\texttt{cm}, \texttt{mm}, \texttt{in}, \texttt{inch} and \texttt{\%}.
There must not be any spaces between the number and the unit. For
example:

\begin{verbatim}
![](file.jpg){ width=50% }
\end{verbatim}

\begin{itemize}
\tightlist
\item
  Dimensions may be converted to a form that is compatible with the
  output format (for example, dimensions given in pixels will be
  converted to inches when converting HTML to LaTeX). Conversion between
  pixels and physical measurements is affected by the \texttt{-\/-dpi}
  option (by default, 96 dpi is assumed, unless the image itself
  contains dpi information).
\item
  The \texttt{\%} unit is generally relative to some available space.
  For example the above example will render to the following.

  \begin{itemize}
  \tightlist
  \item
    HTML:
    \texttt{\textless{}img\ href="file.jpg"\ style="width:\ 50\%;"\ /\textgreater{}}
  \item
    LaTeX:
    \texttt{\textbackslash{}includegraphics{[}width=0.5\textbackslash{}textwidth,height=\textbackslash{}textheight{]}\{file.jpg\}}
    (If you're using a custom template, you need to configure
    \texttt{graphicx} as in the default template.)
  \item
    ConTeXt:
    \texttt{\textbackslash{}externalfigure{[}file.jpg{]}{[}width=0.5\textbackslash{}textwidth{]}}
  \end{itemize}
\item
  Some output formats have a notion of a class
  (\href{https://wiki.contextgarden.net/Using_Graphics\#Multiple_Image_Settings}{ConTeXt})
  or a unique identifier (LaTeX \texttt{\textbackslash{}caption}), or
  both (HTML).
\item
  When no \texttt{width} or \texttt{height} attributes are specified,
  the fallback is to look at the image resolution and the dpi metadata
  embedded in the image file.
\end{itemize}

\subsection{Divs and Spans}\label{divs-and-spans}

Using the \texttt{native\_divs} and \texttt{native\_spans} extensions
(see \hyperref[extension-native_divs]{above}), HTML syntax can be used
as part of markdown to create native \texttt{Div} and \texttt{Span}
elements in the pandoc AST (as opposed to raw HTML). However, there is
also nicer syntax available:

\paragraph{\texorpdfstring{Extension:
\texttt{fenced\_divs}}{Extension: fenced\_divs}}\label{extension-fenced_divs}

Allow special fenced syntax for native \texttt{Div} blocks. A Div starts
with a fence containing at least three consecutive colons plus some
attributes. The attributes may optionally be followed by another string
of consecutive colons. The attribute syntax is exactly as in fenced code
blocks (see \hyperref[extension-fenced_code_attributes]{Extension:
\texttt{fenced\_code\_attributes}}). As with fenced code blocks, one can
use either attributes in curly braces or a single unbraced word, which
will be treated as a class name. The Div ends with another line
containing a string of at least three consecutive colons. The fenced Div
should be separated by blank lines from preceding and following blocks.

Example:

\begin{verbatim}
::::: {#special .sidebar}
Here is a paragraph.

And another.
:::::
\end{verbatim}

Fenced divs can be nested. Opening fences are distinguished because they
\emph{must} have attributes:

\begin{verbatim}
::: Warning ::::::
This is a warning.

::: Danger
This is a warning within a warning.
:::
::::::::::::::::::
\end{verbatim}

Fences without attributes are always closing fences. Unlike with fenced
code blocks, the number of colons in the closing fence need not match
the number in the opening fence. However, it can be helpful for visual
clarity to use fences of different lengths to distinguish nested divs
from their parents.

\paragraph{\texorpdfstring{Extension:
\texttt{bracketed\_spans}}{Extension: bracketed\_spans}}\label{extension-bracketed_spans}

A bracketed sequence of inlines, as one would use to begin a link, will
be treated as a \texttt{Span} with attributes if it is followed
immediately by attributes:

\begin{verbatim}
[This is *some text*]{.class key="val"}
\end{verbatim}

\subsection{Footnotes}\label{footnotes}

\paragraph{\texorpdfstring{Extension:
\texttt{footnotes}}{Extension: footnotes}}\label{extension-footnotes}

Pandoc's Markdown allows footnotes, using the following syntax:

\begin{verbatim}
Here is a footnote reference,[^1] and another.[^longnote]

[^1]: Here is the footnote.

[^longnote]: Here's one with multiple blocks.

    Subsequent paragraphs are indented to show that they
belong to the previous footnote.

        { some.code }

    The whole paragraph can be indented, or just the first
    line.  In this way, multi-paragraph footnotes work like
    multi-paragraph list items.

This paragraph won't be part of the note, because it
isn't indented.
\end{verbatim}

The identifiers in footnote references may not contain spaces, tabs, or
newlines. These identifiers are used only to correlate the footnote
reference with the note itself; in the output, footnotes will be
numbered sequentially.

The footnotes themselves need not be placed at the end of the document.
They may appear anywhere except inside other block elements (lists,
block quotes, tables, etc.). Each footnote should be separated from
surrounding content (including other footnotes) by blank lines.

\paragraph{\texorpdfstring{Extension:
\texttt{inline\_notes}}{Extension: inline\_notes}}\label{extension-inline_notes}

Inline footnotes are also allowed (though, unlike regular notes, they
cannot contain multiple paragraphs). The syntax is as follows:

\begin{verbatim}
Here is an inline note.^[Inline notes are easier to write, since
you don't have to pick an identifier and move down to type the
note.]
\end{verbatim}

Inline and regular footnotes may be mixed freely.

\subsection{Citation syntax}\label{citation-syntax}

\paragraph{\texorpdfstring{Extension:
\texttt{citations}}{Extension: citations}}\label{extension-citations}

To cite a bibliographic item with an identifier foo, use the syntax
\texttt{@foo}. Normal citations should be included in square brackets,
with semicolons separating distinct items:

\begin{verbatim}
Blah blah [@doe99; @smith2000; @smith2004].
\end{verbatim}

How this is rendered depends on the citation style. In an author-date
style, it might render as

\begin{verbatim}
Blah blah (Doe 1999, Smith 2000, 2004).
\end{verbatim}

In a footnote style, it might render as

\begin{verbatim}
Blah blah.[^1]

[^1]:  John Doe, "Frogs," *Journal of Amphibians* 44 (1999);
Susan Smith, "Flies," *Journal of Insects* (2000);
Susan Smith, "Bees," *Journal of Insects* (2004).
\end{verbatim}

See the \href{https://citationstyles.org/authors/}{CSL user
documentation} for more information about CSL styles and how they affect
rendering.

Unless a citation key starts with a letter, digit, or \texttt{\_}, and
contains only alphanumerics and single internal punctuation characters
(\texttt{:.\#\$\%\&-+?\textless{}\textgreater{}\textasciitilde{}/}), it
must be surrounded by curly braces, which are not considered part of the
key. In \texttt{@Foo\_bar.baz.}, the key is \texttt{Foo\_bar.baz}
because the final period is not \emph{internal} punctuation, so it is
not included in the key. In \texttt{@\{Foo\_bar.baz.\}}, the key is
\texttt{Foo\_bar.baz.}, including the final period. In
\texttt{@Foo\_bar-\/-baz}, the key is \texttt{Foo\_bar} because the
repeated internal punctuation characters terminate the key. The curly
braces are recommended if you use URLs as keys:
\texttt{{[}@\{https://example.com/bib?name=foobar\&date=2000\},\ p.\ \ 33{]}}.

Citation items may optionally include a prefix, a locator, and a suffix.
In

\begin{verbatim}
Blah blah [see @doe99, pp. 33-35 and *passim*; @smith04, chap. 1].
\end{verbatim}

the first item (\texttt{doe99}) has prefix \texttt{see}, locator
\texttt{pp.\ \ 33-35}, and suffix \texttt{and\ *passim*}. The second
item (\texttt{smith04}) has locator \texttt{chap.\ 1} and no prefix or
suffix.

Pandoc uses some heuristics to separate the locator from the rest of the
subject. It is sensitive to the locator terms defined in the
\href{https://github.com/citation-style-language/locales}{CSL locale
files}. Either abbreviated or unabbreviated forms are accepted. In the
\texttt{en-US} locale, locator terms can be written in either singular
or plural forms, as \texttt{book}, \texttt{bk.}/\texttt{bks.};
\texttt{chapter}, \texttt{chap.}/\texttt{chaps.}; \texttt{column},
\texttt{col.}/\texttt{cols.}; \texttt{figure},
\texttt{fig.}/\texttt{figs.}; \texttt{folio},
\texttt{fol.}/\texttt{fols.}; \texttt{number},
\texttt{no.}/\texttt{nos.}; \texttt{line}, \texttt{l.}/\texttt{ll.};
\texttt{note}, \texttt{n.}/\texttt{nn.}; \texttt{opus},
\texttt{op.}/\texttt{opp.}; \texttt{page}, \texttt{p.}/\texttt{pp.};
\texttt{paragraph}, \texttt{para.}/\texttt{paras.}; \texttt{part},
\texttt{pt.}/\texttt{pts.}; \texttt{section},
\texttt{sec.}/\texttt{secs.}; \texttt{sub\ verbo},
\texttt{s.v.}/\texttt{s.vv.}; \texttt{verse}, \texttt{v.}/\texttt{vv.};
\texttt{volume}, \texttt{vol.}/\texttt{vols.}; \texttt{¶}/\texttt{¶¶};
\texttt{§}/\texttt{§§}. If no locator term is used, ``page'' is assumed.

In complex cases, you can force something to be treated as a locator by
enclosing it in curly braces or prevent parsing the suffix as locator by
prepending curly braces:

\begin{verbatim}
[@smith{ii, A, D-Z}, with a suffix]
[@smith, {pp. iv, vi-xi, (xv)-(xvii)} with suffix here]
[@smith{}, 99 years later]
\end{verbatim}

A minus sign (\texttt{-}) before the \texttt{@} will suppress mention of
the author in the citation. This can be useful when the author is
already mentioned in the text:

\begin{verbatim}
Smith says blah [-@smith04].
\end{verbatim}

You can also write an author-in-text citation, by omitting the square
brackets:

\begin{verbatim}
@smith04 says blah.

@smith04 [p. 33] says blah.
\end{verbatim}

This will cause the author's name to be rendered, followed by the
bibliographical details. Use this form when you want to make the
citation the subject of a sentence.

When you are using a note style, it is usually better to let citeproc
create the footnotes from citations rather than writing an explicit
note. If you do write an explicit note that contains a citation, note
that normal citations will be put in parentheses, while author-in-text
citations will not. For this reason, it is sometimes preferable to use
the author-in-text style inside notes when using a note style.

\subsection{Non-default extensions}\label{non-default-extensions}

The following Markdown syntax extensions are not enabled by default in
pandoc, but may be enabled by adding \texttt{+EXTENSION} to the format
name, where \texttt{EXTENSION} is the name of the extension. Thus, for
example, \texttt{markdown+hard\_line\_breaks} is Markdown with hard line
breaks.

\paragraph{\texorpdfstring{Extension:
\texttt{rebase\_relative\_paths}}{Extension: rebase\_relative\_paths}}\label{extension-rebase_relative_paths}

Rewrite relative paths for Markdown links and images, depending on the
path of the file containing the link or image link. For each link or
image, pandoc will compute the directory of the containing file,
relative to the working directory, and prepend the resulting path to the
link or image path.

The use of this extension is best understood by example. Suppose you
have a subdirectory for each chapter of a book, \texttt{chap1},
\texttt{chap2}, \texttt{chap3}. Each contains a file \texttt{text.md}
and a number of images used in the chapter. You would like to have
\texttt{!{[}image{]}(spider.jpg)} in \texttt{chap1/text.md} refer to
\texttt{chap1/spider.jpg} and \texttt{!{[}image{]}(spider.jpg)} in
\texttt{chap2/text.md} refer to \texttt{chap2/spider.jpg}. To do this,
use

\begin{verbatim}
pandoc chap*/*.md -f markdown+rebase_relative_paths
\end{verbatim}

Without this extension, you would have to use
\texttt{!{[}image{]}(chap1/spider.jpg)} in \texttt{chap1/text.md} and
\texttt{!{[}image{]}(chap2/spider.jpg)} in \texttt{chap2/text.md}. Links
with relative paths will be rewritten in the same way as images.

Absolute paths and URLs are not changed. Neither are empty paths or
paths consisting entirely of a fragment, e.g., \texttt{\#foo}.

Note that relative paths in reference links and images will be rewritten
relative to the file containing the link reference definition, not the
file containing the reference link or image itself, if these differ.

\paragraph{\texorpdfstring{Extension:
\texttt{attributes}}{Extension: attributes}}\label{extension-attributes}

Allows attributes to be attached to any inline or block-level element
when parsing \texttt{commonmark}. The syntax for the attributes is the
same as that used in
\hyperref[extension-header_attributes]{\texttt{header\_attributes}}.

\begin{itemize}
\tightlist
\item
  Attributes that occur immediately after an inline element affect that
  element. If they follow a space, then they belong to the space.
  (Hence, this option subsumes \texttt{inline\_code\_attributes} and
  \texttt{link\_attributes}.)
\item
  Attributes that occur immediately before a block element, on a line by
  themselves, affect that element.
\item
  Consecutive attribute specifiers may be used, either for blocks or for
  inlines. Their attributes will be combined.
\item
  Attributes that occur at the end of the text of a Setext or ATX
  heading (separated by whitespace from the text) affect the heading
  element. (Hence, this option subsumes \texttt{header\_attributes}.)
\item
  Attributes that occur after the opening fence in a fenced code block
  affect the code block element. (Hence, this option subsumes
  \texttt{fenced\_code\_attributes}.)
\item
  Attributes that occur at the end of a reference link definition affect
  links that refer to that definition.
\end{itemize}

Note that pandoc's AST does not currently allow attributes to be
attached to arbitrary elements. Hence a Span or Div container will be
added if needed.

\paragraph{\texorpdfstring{Extension:
\texttt{old\_dashes}}{Extension: old\_dashes}}\label{extension-old_dashes}

Selects the pandoc \textless= 1.8.2.1 behavior for parsing smart dashes:
\texttt{-} before a numeral is an en-dash, and \texttt{-\/-} is an
em-dash. This option only has an effect if \texttt{smart} is enabled. It
is selected automatically for \texttt{textile} input.

\paragraph{\texorpdfstring{Extension:
\texttt{angle\_brackets\_escapable}}{Extension: angle\_brackets\_escapable}}\label{extension-angle_brackets_escapable}

Allow \texttt{\textless{}} and \texttt{\textgreater{}} to be
backslash-escaped, as they can be in GitHub flavored Markdown but not
original Markdown. This is implied by pandoc's default
\texttt{all\_symbols\_escapable}.

\paragraph{\texorpdfstring{Extension:
\texttt{lists\_without\_preceding\_blankline}}{Extension: lists\_without\_preceding\_blankline}}\label{extension-lists_without_preceding_blankline}

Allow a list to occur right after a paragraph, with no intervening blank
space.

\paragraph{\texorpdfstring{Extension:
\texttt{four\_space\_rule}}{Extension: four\_space\_rule}}\label{extension-four_space_rule}

Selects the pandoc \textless= 2.0 behavior for parsing lists, so that
four spaces indent are needed for list item continuation paragraphs.

\paragraph{\texorpdfstring{Extension:
\texttt{spaced\_reference\_links}}{Extension: spaced\_reference\_links}}\label{extension-spaced_reference_links}

Allow whitespace between the two components of a reference link, for
example,

\begin{verbatim}
[foo] [bar].
\end{verbatim}

\paragraph{\texorpdfstring{Extension:
\texttt{hard\_line\_breaks}}{Extension: hard\_line\_breaks}}\label{extension-hard_line_breaks}

Causes all newlines within a paragraph to be interpreted as hard line
breaks instead of spaces.

\paragraph{\texorpdfstring{Extension:
\texttt{ignore\_line\_breaks}}{Extension: ignore\_line\_breaks}}\label{extension-ignore_line_breaks}

Causes newlines within a paragraph to be ignored, rather than being
treated as spaces or as hard line breaks. This option is intended for
use with East Asian languages where spaces are not used between words,
but text is divided into lines for readability.

\paragraph{\texorpdfstring{Extension:
\texttt{east\_asian\_line\_breaks}}{Extension: east\_asian\_line\_breaks}}\label{extension-east_asian_line_breaks}

Causes newlines within a paragraph to be ignored, rather than being
treated as spaces or as hard line breaks, when they occur between two
East Asian wide characters. This is a better choice than
\texttt{ignore\_line\_breaks} for texts that include a mix of East Asian
wide characters and other characters.

\paragraph{\texorpdfstring{Extension:
\texttt{emoji}}{Extension: emoji}}\label{extension-emoji}

Parses textual emojis like \texttt{:smile:} as Unicode emoticons.

\paragraph{\texorpdfstring{Extension:
\texttt{tex\_math\_single\_backslash}}{Extension: tex\_math\_single\_backslash}}\label{extension-tex_math_single_backslash}

Causes anything between \texttt{\textbackslash{}(} and
\texttt{\textbackslash{})} to be interpreted as inline TeX math, and
anything between \texttt{\textbackslash{}{[}} and
\texttt{\textbackslash{}{]}} to be interpreted as display TeX math.
Note: a drawback of this extension is that it precludes escaping
\texttt{(} and \texttt{{[}}.

\paragraph{\texorpdfstring{Extension:
\texttt{tex\_math\_double\_backslash}}{Extension: tex\_math\_double\_backslash}}\label{extension-tex_math_double_backslash}

Causes anything between \texttt{\textbackslash{}\textbackslash{}(} and
\texttt{\textbackslash{}\textbackslash{})} to be interpreted as inline
TeX math, and anything between
\texttt{\textbackslash{}\textbackslash{}{[}} and
\texttt{\textbackslash{}\textbackslash{}{]}} to be interpreted as
display TeX math.

\paragraph{\texorpdfstring{Extension:
\texttt{markdown\_attribute}}{Extension: markdown\_attribute}}\label{extension-markdown_attribute}

By default, pandoc interprets material inside block-level tags as
Markdown. This extension changes the behavior so that Markdown is only
parsed inside block-level tags if the tags have the attribute
\texttt{markdown=1}.

\paragraph{\texorpdfstring{Extension:
\texttt{mmd\_title\_block}}{Extension: mmd\_title\_block}}\label{extension-mmd_title_block}

Enables a
\href{https://fletcherpenney.net/multimarkdown/}{MultiMarkdown} style
title block at the top of the document, for example:

\begin{verbatim}
Title:   My title
Author:  John Doe
Date:    September 1, 2008
Comment: This is a sample mmd title block, with
         a field spanning multiple lines.
\end{verbatim}

See the MultiMarkdown documentation for details. If
\texttt{pandoc\_title\_block} or \texttt{yaml\_metadata\_block} is
enabled, it will take precedence over \texttt{mmd\_title\_block}.

\paragraph{\texorpdfstring{Extension:
\texttt{abbreviations}}{Extension: abbreviations}}\label{extension-abbreviations}

Parses PHP Markdown Extra abbreviation keys, like

\begin{verbatim}
*[HTML]: Hypertext Markup Language
\end{verbatim}

Note that the pandoc document model does not support abbreviations, so
if this extension is enabled, abbreviation keys are simply skipped (as
opposed to being parsed as paragraphs).

\paragraph{\texorpdfstring{Extension:
\texttt{autolink\_bare\_uris}}{Extension: autolink\_bare\_uris}}\label{extension-autolink_bare_uris}

Makes all absolute URIs into links, even when not surrounded by pointy
braces \texttt{\textless{}...\textgreater{}}.

\paragraph{\texorpdfstring{Extension:
\texttt{mmd\_link\_attributes}}{Extension: mmd\_link\_attributes}}\label{extension-mmd_link_attributes}

Parses multimarkdown style key-value attributes on link and image
references. This extension should not be confused with the
\hyperref[extension-link_attributes]{\texttt{link\_attributes}}
extension.

\begin{verbatim}
This is a reference ![image][ref] with multimarkdown attributes.

[ref]: https://path.to/image "Image title" width=20px height=30px
       id=myId class="myClass1 myClass2"
\end{verbatim}

\paragraph{\texorpdfstring{Extension:
\texttt{mmd\_header\_identifiers}}{Extension: mmd\_header\_identifiers}}\label{extension-mmd_header_identifiers}

Parses multimarkdown style heading identifiers (in square brackets,
after the heading but before any trailing \texttt{\#}s in an ATX
heading).

\paragraph{\texorpdfstring{Extension:
\texttt{compact\_definition\_lists}}{Extension: compact\_definition\_lists}}\label{extension-compact_definition_lists}

Activates the definition list syntax of pandoc 1.12.x and earlier. This
syntax differs from the one described above under
\hyperref[definition-lists]{Definition lists} in several respects:

\begin{itemize}
\tightlist
\item
  No blank line is required between consecutive items of the definition
  list.
\item
  To get a ``tight'' or ``compact'' list, omit space between consecutive
  items; the space between a term and its definition does not affect
  anything.
\item
  Lazy wrapping of paragraphs is not allowed: the entire definition must
  be indented four spaces.\footnote{To see why laziness is incompatible
    with relaxing the requirement of a blank line between items,
    consider the following example:

\begin{Verbatim}
bar
:    definition
foo
:    definition
\end{Verbatim}

    Is this a single list item with two definitions of ``bar,'' the
    first of which is lazily wrapped, or two list items? To remove the
    ambiguity we must either disallow lazy wrapping or require a blank
    line between list items.}
\end{itemize}

\paragraph{\texorpdfstring{Extension:
\texttt{gutenberg}}{Extension: gutenberg}}\label{extension-gutenberg}

Use \href{https://www.gutenberg.org}{Project Gutenberg} conventions for
\texttt{plain} output: all-caps for strong emphasis, surround by
underscores for regular emphasis, add extra blank space around headings.

\paragraph{\texorpdfstring{Extension:
\texttt{sourcepos}}{Extension: sourcepos}}\label{extension-sourcepos}

Include source position attributes when parsing \texttt{commonmark}. For
elements that accept attributes, a \texttt{data-pos} attribute is added;
other elements are placed in a surrounding Div or Span element with a
\texttt{data-pos} attribute.

\paragraph{\texorpdfstring{Extension:
\texttt{short\_subsuperscripts}}{Extension: short\_subsuperscripts}}\label{extension-short_subsuperscripts}

Parse multimarkdown style subscripts and superscripts, which start with
a `\textasciitilde{}' or `\^{}' character, respectively, and include the
alphanumeric sequence that follows. For example:

\begin{verbatim}
x^2 = 4
\end{verbatim}

or

\begin{verbatim}
Oxygen is O~2.
\end{verbatim}

\subsection{Markdown variants}\label{markdown-variants}

In addition to pandoc's extended Markdown, the following Markdown
variants are supported:

\begin{itemize}
\tightlist
\item
  \texttt{markdown\_phpextra} (PHP Markdown Extra)
\item
  \texttt{markdown\_github} (deprecated GitHub-Flavored Markdown)
\item
  \texttt{markdown\_mmd} (MultiMarkdown)
\item
  \texttt{markdown\_strict} (Markdown.pl)
\item
  \texttt{commonmark} (CommonMark)
\item
  \texttt{gfm} (Github-Flavored Markdown)
\item
  \texttt{commonmark\_x} (CommonMark with many pandoc extensions)
\end{itemize}

To see which extensions are supported for a given format, and which are
enabled by default, you can use the command

\begin{verbatim}
pandoc --list-extensions=FORMAT
\end{verbatim}

where \texttt{FORMAT} is replaced with the name of the format.

Note that the list of extensions for \texttt{commonmark}, \texttt{gfm},
and \texttt{commonmark\_x} are defined relative to default commonmark.
So, for example, \texttt{backtick\_code\_blocks} does not appear as an
extension, since it is enabled by default and cannot be disabled.

\section{Citations}\label{citations}

When the \texttt{-\/-citeproc} option is used, pandoc can automatically
generate citations and a bibliography in a number of styles. Basic usage
is

\begin{verbatim}
pandoc --citeproc myinput.txt
\end{verbatim}

To use this feature, you will need to have

\begin{itemize}
\tightlist
\item
  a document containing citations (see
  \hyperref[org-citations]{Extension: \texttt{citations}});
\item
  a source of bibliographic data: either an external bibliography file
  or a list of \texttt{references} in the document's YAML metadata;
\item
  optionally, a
  \href{https://docs.citationstyles.org/en/stable/specification.html}{CSL}
  citation style.
\end{itemize}

\subsection{Specifying bibliographic
data}\label{specifying-bibliographic-data}

You can specify an external bibliography using the \texttt{bibliography}
metadata field in a YAML metadata section or the
\texttt{-\/-bibliography} command line argument. If you want to use
multiple bibliography files, you can supply multiple
\texttt{-\/-bibliography} arguments or set \texttt{bibliography}
metadata field to YAML array. A bibliography may have any of these
formats:

\begin{longtable}[]{@{}ll@{}}
\toprule\noalign{}
Format & File extension \\
\midrule\noalign{}
\endhead
\bottomrule\noalign{}
\endlastfoot
BibLaTeX & .bib \\
BibTeX & .bibtex \\
CSL JSON & .json \\
CSL YAML & .yaml \\
RIS & .ris \\
\end{longtable}

Note that \texttt{.bib} can be used with both BibTeX and BibLaTeX files;
use the extension \texttt{.bibtex} to force interpretation as BibTeX.

In BibTeX and BibLaTeX databases, pandoc parses LaTeX markup inside
fields such as \texttt{title}; in CSL YAML databases, pandoc Markdown;
and in CSL JSON databases, an
\href{https://docs.citationstyles.org/en/1.0/release-notes.html\#rich-text-markup-within-fields}{HTML-like
markup}:

\begin{description}
\tightlist
\item[\texttt{\textless{}i\textgreater{}...\textless{}/i\textgreater{}}]
italics
\item[\texttt{\textless{}b\textgreater{}...\textless{}/b\textgreater{}}]
bold
\item[\texttt{\textless{}span\ style="font-variant:small-caps;"\textgreater{}...\textless{}/span\textgreater{}}
or \texttt{\textless{}sc\textgreater{}...\textless{}/sc\textgreater{}}]
small capitals
\item[\texttt{\textless{}sub\textgreater{}...\textless{}/sub\textgreater{}}]
subscript
\item[\texttt{\textless{}sup\textgreater{}...\textless{}/sup\textgreater{}}]
superscript
\item[\texttt{\textless{}span\ class="nocase"\textgreater{}...\textless{}/span\textgreater{}}]
prevent a phrase from being capitalized as title case
\end{description}

As an alternative to specifying a bibliography file using
\texttt{-\/-bibliography} or the YAML metadata field
\texttt{bibliography}, you can include the citation data directly in the
\texttt{references} field of the document's YAML metadata. The field
should contain an array of YAML-encoded references, for example:

\begin{verbatim}
---
references:
- type: article-journal
  id: WatsonCrick1953
  author:
  - family: Watson
    given: J. D.
  - family: Crick
    given: F. H. C.
  issued:
    date-parts:
    - - 1953
      - 4
      - 25
  title: 'Molecular structure of nucleic acids: a structure for
    deoxyribose nucleic acid'
  title-short: Molecular structure of nucleic acids
  container-title: Nature
  volume: 171
  issue: 4356
  page: 737-738
  DOI: 10.1038/171737a0
  URL: https://www.nature.com/articles/171737a0
  language: en-GB
...
\end{verbatim}

If both an external bibliography and inline (YAML metadata) references
are provided, both will be used. In case of conflicting \texttt{id}s,
the inline references will take precedence.

Note that pandoc can be used to produce such a YAML metadata section
from a BibTeX, BibLaTeX, or CSL JSON bibliography:

\begin{verbatim}
pandoc chem.bib -s -f biblatex -t markdown
pandoc chem.json -s -f csljson -t markdown
\end{verbatim}

Indeed, pandoc can convert between any of these citation formats:

\begin{verbatim}
pandoc chem.bib -s -f biblatex -t csljson
pandoc chem.yaml -s -f markdown -t biblatex
\end{verbatim}

Running pandoc on a bibliography file with the \texttt{-\/-citeproc}
option will create a formatted bibliography in the format of your
choice:

\begin{verbatim}
pandoc chem.bib -s --citeproc -o chem.html
pandoc chem.bib -s --citeproc -o chem.pdf
\end{verbatim}

\subsubsection{Capitalization in titles}\label{capitalization-in-titles}

If you are using a bibtex or biblatex bibliography, then observe the
following rules:

\begin{itemize}
\item
  English titles should be in title case. Non-English titles should be
  in sentence case, and the \texttt{langid} field in biblatex should be
  set to the relevant language. (The following values are treated as
  English: \texttt{american}, \texttt{british}, \texttt{canadian},
  \texttt{english}, \texttt{australian}, \texttt{newzealand},
  \texttt{USenglish}, or \texttt{UKenglish}.)
\item
  As is standard with bibtex/biblatex, proper names should be protected
  with curly braces so that they won't be lowercased in styles that call
  for sentence case. For example:

\begin{verbatim}
title = {My Dinner with {Andre}}
\end{verbatim}
\item
  In addition, words that should remain lowercase (or camelCase) should
  be protected:

\begin{verbatim}
title = {Spin Wave Dispersion on the {nm} Scale}
\end{verbatim}

  Though this is not necessary in bibtex/biblatex, it is necessary with
  citeproc, which stores titles internally in sentence case, and
  converts to title case in styles that require it. Here we protect
  ``nm'' so that it doesn't get converted to ``Nm'' at this stage.
\end{itemize}

If you are using a CSL bibliography (either JSON or YAML), then observe
the following rules:

\begin{itemize}
\item
  All titles should be in sentence case.
\item
  Use the \texttt{language} field for non-English titles to prevent
  their conversion to title case in styles that call for this.
  (Conversion happens only if \texttt{language} begins with \texttt{en}
  or is left empty.)
\item
  Protect words that should not be converted to title case using this
  syntax:

\begin{verbatim}
Spin wave dispersion on the <span class="nocase">nm</span> scale
\end{verbatim}
\end{itemize}

\subsubsection{Conference Papers, Published
vs.~Unpublished}\label{conference-papers-published-vs.-unpublished}

For a formally published conference paper, use the biblatex entry type
\texttt{inproceedings} (which will be mapped to CSL
\texttt{paper-conference}).

For an unpublished manuscript, use the biblatex entry type
\texttt{unpublished} without an \texttt{eventtitle} field (this entry
type will be mapped to CSL \texttt{manuscript}).

For a talk, an unpublished conference paper, or a poster presentation,
use the biblatex entry type \texttt{unpublished} with an
\texttt{eventtitle} field (this entry type will be mapped to CSL
\texttt{speech}). Use the biblatex \texttt{type} field to indicate the
type, e.g.~``Paper'', or ``Poster''. \texttt{venue} and
\texttt{eventdate} may be useful too, though \texttt{eventdate} will not
be rendered by most CSL styles. Note that \texttt{venue} is for the
event's venue, unlike \texttt{location} which describes the publisher's
location; do not use the latter for an unpublished conference paper.

\subsection{Specifying a citation
style}\label{specifying-a-citation-style}

Citations and references can be formatted using any style supported by
the \href{https://citationstyles.org}{Citation Style Language}, listed
in the \href{https://www.zotero.org/styles}{Zotero Style Repository}.
These files are specified using the \texttt{-\/-csl} option or the
\texttt{csl} (or \texttt{citation-style}) metadata field. By default,
pandoc will use the \href{https://chicagomanualofstyle.org}{Chicago
Manual of Style} author-date format. (You can override this default by
copying a CSL style of your choice to \texttt{default.csl} in your user
data directory.) The CSL project provides further information on
\href{https://citationstyles.org/authors/}{finding and editing styles}.

The \texttt{-\/-citation-abbreviations} option (or the
\texttt{citation-abbreviations} metadata field) may be used to specify a
JSON file containing abbreviations of journals that should be used in
formatted bibliographies when \texttt{form="short"} is specified. The
format of the file can be illustrated with an example:

\begin{verbatim}
{ "default": {
    "container-title": {
            "Lloyd's Law Reports": "Lloyd's Rep",
            "Estates Gazette": "EG",
            "Scots Law Times": "SLT"
    }
  }
}
\end{verbatim}

\subsection{Citations in note styles}\label{citations-in-note-styles}

Pandoc's citation processing is designed to allow you to move between
author-date, numerical, and note styles without modifying the markdown
source. When you're using a note style, avoid inserting footnotes
manually. Instead, insert citations just as you would in an author-date
style---for example,

\begin{verbatim}
Blah blah [@foo, p. 33].
\end{verbatim}

The footnote will be created automatically. Pandoc will take care of
removing the space and moving the note before or after the period,
depending on the setting of \texttt{notes-after-punctuation}, as
described below in \hyperref[other-relevant-metadata-fields]{Other
relevant metadata fields}.

In some cases you may need to put a citation inside a regular footnote.
Normal citations in footnotes (such as \texttt{{[}@foo,\ p.\ 33{]}})
will be rendered in parentheses. In-text citations (such as
\texttt{@foo\ {[}p.\ 33{]}}) will be rendered without parentheses. (A
comma will be added if appropriate.) Thus:

\begin{verbatim}
[^1]:  Some studies [@foo; @bar, p. 33] show that
frubulicious zoosnaps are quantical.  For a survey
of the literature, see @baz [chap. 1].
\end{verbatim}

\subsection{Raw content in a style}\label{raw-content-in-a-style}

To include raw content in a prefix, suffix, delimiter, or term, surround
it with these tags indicating the format:

\begin{verbatim}
{{jats}}&lt;ref&gt;{{/jats}}
\end{verbatim}

Without the tags, the string will be interpreted as a string and escaped
in the output, rather than being passed through raw.

This feature allows stylesheets to be customized to give different
output for different output formats. However, stylesheets customized in
this way will not be usable by other CSL implementations.

\subsection{Placement of the
bibliography}\label{placement-of-the-bibliography}

If the style calls for a list of works cited, it will be placed in a div
with id \texttt{refs}, if one exists:

\begin{verbatim}
::: {#refs}
:::
\end{verbatim}

Otherwise, it will be placed at the end of the document. Generation of
the bibliography can be suppressed by setting
\texttt{suppress-bibliography:\ true} in the YAML metadata.

If you wish the bibliography to have a section heading, you can set
\texttt{reference-section-title} in the metadata, or put the heading at
the beginning of the div with id \texttt{refs} (if you are using it) or
at the end of your document:

\begin{verbatim}
last paragraph...

# References
\end{verbatim}

The bibliography will be inserted after this heading. Note that the
\texttt{unnumbered} class will be added to this heading, so that the
section will not be numbered.

If you want to put the bibliography into a variable in your template,
one way to do that is to put the div with id \texttt{refs} into a
metadata field, e.g.

\begin{verbatim}
---
refs: |
   ::: {#refs}
   :::
...
\end{verbatim}

You can then put the variable \texttt{\$refs\$} into your template where
you want the bibliography to be placed.

\subsection{Including uncited items in the
bibliography}\label{including-uncited-items-in-the-bibliography}

If you want to include items in the bibliography without actually citing
them in the body text, you can define a dummy \texttt{nocite} metadata
field and put the citations there:

\begin{verbatim}
---
nocite: |
  @item1, @item2
...

@item3
\end{verbatim}

In this example, the document will contain a citation for \texttt{item3}
only, but the bibliography will contain entries for \texttt{item1},
\texttt{item2}, and \texttt{item3}.

It is possible to create a bibliography with all the citations, whether
or not they appear in the document, by using a wildcard:

\begin{verbatim}
---
nocite: |
  @*
...
\end{verbatim}

For LaTeX output, you can also use
\href{https://ctan.org/pkg/natbib}{\texttt{natbib}} or
\href{https://ctan.org/pkg/biblatex}{\texttt{biblatex}} to render the
bibliography. In order to do so, specify bibliography files as outlined
above, and add \texttt{-\/-natbib} or \texttt{-\/-biblatex} argument to
pandoc invocation. Bear in mind that bibliography files have to be in
either BibTeX (for \texttt{-\/-natbib}) or BibLaTeX (for
\texttt{-\/-biblatex}) format.

\subsection{Other relevant metadata
fields}\label{other-relevant-metadata-fields}

A few other metadata fields affect bibliography formatting:

\begin{description}
\item[\texttt{link-citations}]
If true, citations will be hyperlinked to the corresponding bibliography
entries (for author-date and numerical styles only). Defaults to false.
\item[\texttt{link-bibliography}]
If true, DOIs, PMCIDs, PMID, and URLs in bibliographies will be rendered
as hyperlinks. (If an entry contains a DOI, PMCID, PMID, or URL, but
none of these fields are rendered by the style, then the title, or in
the absence of a title the whole entry, will be hyperlinked.) Defaults
to true.
\item[\texttt{lang}]
The \texttt{lang} field will affect how the style is localized, for
example in the translation of labels, the use of quotation marks, and
the way items are sorted. (For backwards compatibility, \texttt{locale}
may be used instead of \texttt{lang}, but this use is deprecated.)

A BCP 47 language tag is expected: for example, \texttt{en},
\texttt{de}, \texttt{en-US}, \texttt{fr-CA}, \texttt{ug-Cyrl}. The
unicode extension syntax (after \texttt{-u-}) may be used to specify
options for collation (sorting) more precisely. Here are some examples:

\begin{itemize}
\tightlist
\item
  \texttt{zh-u-co-pinyin} -- Chinese with the Pinyin collation.
\item
  \texttt{es-u-co-trad} -- Spanish with the traditional collation (with
  \texttt{Ch} sorting after \texttt{C}).
\item
  \texttt{fr-u-kb} -- French with ``backwards'' accent sorting (with
  \texttt{coté} sorting after \texttt{côte}).
\item
  \texttt{en-US-u-kf-upper} -- English with uppercase letters sorting
  before lower (default is lower before upper).
\end{itemize}
\item[\texttt{notes-after-punctuation}]
If true (the default for note styles), pandoc will put footnote
references or superscripted numerical citations after following
punctuation. For example, if the source contains
\texttt{blah\ blah\ {[}@jones99{]}.}, the result will look like
\texttt{blah\ blah.{[}\^{}1{]}}, with the note moved after the period
and the space collapsed. If false, the space will still be collapsed,
but the footnote will not be moved after the punctuation. The option may
also be used in numerical styles that use superscripts for citation
numbers (but for these styles the default is not to move the citation).
\end{description}

\section{Slide shows}\label{slide-shows}

You can use pandoc to produce an HTML + JavaScript slide presentation
that can be viewed via a web browser. There are five ways to do this,
using \href{https://meyerweb.com/eric/tools/s5/}{S5},
\href{https://paulrouget.com/dzslides/}{DZSlides},
\href{https://www.w3.org/Talks/Tools/Slidy2/}{Slidy},
\href{https://goessner.net/articles/slideous/}{Slideous}, or
\href{https://revealjs.com/}{reveal.js}. You can also produce a PDF
slide show using LaTeX
\href{https://ctan.org/pkg/beamer}{\texttt{beamer}}, or slide shows in
Microsoft
\href{https://en.wikipedia.org/wiki/Microsoft_PowerPoint}{PowerPoint}
format.

Here's the Markdown source for a simple slide show, \texttt{habits.txt}:

\begin{verbatim}
% Habits
% John Doe
% March 22, 2005

# In the morning

## Getting up

- Turn off alarm
- Get out of bed

## Breakfast

- Eat eggs
- Drink coffee

# In the evening

## Dinner

- Eat spaghetti
- Drink wine

------------------

![picture of spaghetti](images/spaghetti.jpg)

## Going to sleep

- Get in bed
- Count sheep
\end{verbatim}

To produce an HTML/JavaScript slide show, simply type

\begin{verbatim}
pandoc -t FORMAT -s habits.txt -o habits.html
\end{verbatim}

where \texttt{FORMAT} is either \texttt{s5}, \texttt{slidy},
\texttt{slideous}, \texttt{dzslides}, or \texttt{revealjs}.

For Slidy, Slideous, reveal.js, and S5, the file produced by pandoc with
the \texttt{-s/-\/-standalone} option embeds a link to JavaScript and
CSS files, which are assumed to be available at the relative path
\texttt{s5/default} (for S5), \texttt{slideous} (for Slideous),
\texttt{reveal.js} (for reveal.js), or at the Slidy website at
\texttt{w3.org} (for Slidy). (These paths can be changed by setting the
\texttt{slidy-url}, \texttt{slideous-url}, \texttt{revealjs-url}, or
\texttt{s5-url} variables; see
\hyperref[variables-for-html-slides]{Variables for HTML slides}, above.)
For DZSlides, the (relatively short) JavaScript and CSS are included in
the file by default.

With all HTML slide formats, the \texttt{-\/-self-contained} option can
be used to produce a single file that contains all of the data necessary
to display the slide show, including linked scripts, stylesheets,
images, and videos.

To produce a PDF slide show using beamer, type

\begin{verbatim}
pandoc -t beamer habits.txt -o habits.pdf
\end{verbatim}

Note that a reveal.js slide show can also be converted to a PDF by
printing it to a file from the browser.

To produce a PowerPoint slide show, type

\begin{verbatim}
pandoc habits.txt -o habits.pptx
\end{verbatim}

\subsection{Structuring the slide
show}\label{structuring-the-slide-show}

By default, the \emph{slide level} is the highest heading level in the
hierarchy that is followed immediately by content, and not another
heading, somewhere in the document. In the example above, level-1
headings are always followed by level-2 headings, which are followed by
content, so the slide level is 2. This default can be overridden using
the \texttt{-\/-slide-level} option.

The document is carved up into slides according to the following rules:

\begin{itemize}
\item
  A horizontal rule always starts a new slide.
\item
  A heading at the slide level always starts a new slide.
\item
  Headings \emph{below} the slide level in the hierarchy create headings
  \emph{within} a slide. (In beamer, a ``block'' will be created. If the
  heading has the class \texttt{example}, an \texttt{exampleblock}
  environment will be used; if it has the class \texttt{alert}, an
  \texttt{alertblock} will be used; otherwise a regular \texttt{block}
  will be used.)
\item
  Headings \emph{above} the slide level in the hierarchy create ``title
  slides,'' which just contain the section title and help to break the
  slide show into sections. Non-slide content under these headings will
  be included on the title slide (for HTML slide shows) or in a
  subsequent slide with the same title (for beamer).
\item
  A title page is constructed automatically from the document's title
  block, if present. (In the case of beamer, this can be disabled by
  commenting out some lines in the default template.)
\end{itemize}

These rules are designed to support many different styles of slide show.
If you don't care about structuring your slides into sections and
subsections, you can either just use level-1 headings for all slides (in
that case, level 1 will be the slide level) or you can set
\texttt{-\/-slide-level=0}.

Note: in reveal.js slide shows, if slide level is 2, a two-dimensional
layout will be produced, with level-1 headings building horizontally and
level-2 headings building vertically. It is not recommended that you use
deeper nesting of section levels with reveal.js unless you set
\texttt{-\/-slide-level=0} (which lets reveal.js produce a
one-dimensional layout and only interprets horizontal rules as slide
boundaries).

\subsubsection{PowerPoint layout choice}\label{powerpoint-layout-choice}

When creating slides, the pptx writer chooses from a number of
pre-defined layouts, based on the content of the slide:

\begin{description}
\tightlist
\item[Title Slide]
This layout is used for the initial slide, which is generated and filled
from the metadata fields \texttt{date}, \texttt{author}, and
\texttt{title}, if they are present.
\item[Section Header]
This layout is used for what pandoc calls ``title slides'', i.e. slides
which start with a header which is above the slide level in the
hierarchy.
\item[Two Content]
This layout is used for two-column slides, i.e.~slides containing a div
with class \texttt{columns} which contains at least two divs with class
\texttt{column}.
\item[Comparison]
This layout is used instead of ``Two Content'' for any two-column slides
in which at least one column contains text followed by non-text (e.g.~an
image or a table).
\item[Content with Caption]
This layout is used for any non-two-column slides which contain text
followed by non-text (e.g.~an image or a table).
\item[Blank]
This layout is used for any slides which only contain blank content,
e.g.~a slide containing only speaker notes, or a slide containing only a
non-breaking space.
\item[Title and Content]
This layout is used for all slides which do not match the criteria for
another layout.
\end{description}

These layouts are chosen from the default pptx reference doc included
with pandoc, unless an alternative reference doc is specified using
\texttt{-\/-reference-doc}.

\subsection{Incremental lists}\label{incremental-lists}

By default, these writers produce lists that display ``all at once.'' If
you want your lists to display incrementally (one item at a time), use
the \texttt{-i} option. If you want a particular list to depart from the
default, put it in a \texttt{div} block with class \texttt{incremental}
or \texttt{nonincremental}. So, for example, using the
\texttt{fenced\ div} syntax, the following would be incremental
regardless of the document default:

\begin{verbatim}
::: incremental

- Eat spaghetti
- Drink wine

:::
\end{verbatim}

or

\begin{verbatim}
::: nonincremental

- Eat spaghetti
- Drink wine

:::
\end{verbatim}

While using \texttt{incremental} and \texttt{nonincremental} divs is the
recommended method of setting incremental lists on a per-case basis, an
older method is also supported: putting lists inside a blockquote will
depart from the document default (that is, it will display incrementally
without the \texttt{-i} option and all at once with the \texttt{-i}
option):

\begin{verbatim}
> - Eat spaghetti
> - Drink wine
\end{verbatim}

Both methods allow incremental and nonincremental lists to be mixed in a
single document.

If you want to include a block-quoted list, you can work around this
behavior by putting the list inside a fenced div, so that it is not the
direct child of the block quote:

\begin{verbatim}
> ::: wrapper
> - a
> - list in a quote
> :::
\end{verbatim}

\subsection{Inserting pauses}\label{inserting-pauses}

You can add ``pauses'' within a slide by including a paragraph
containing three dots, separated by spaces:

\begin{verbatim}
# Slide with a pause

content before the pause

. . .

content after the pause
\end{verbatim}

Note: this feature is not yet implemented for PowerPoint output.

\subsection{Styling the slides}\label{styling-the-slides}

You can change the style of HTML slides by putting customized CSS files
in \texttt{\$DATADIR/s5/default} (for S5), \texttt{\$DATADIR/slidy} (for
Slidy), or \texttt{\$DATADIR/slideous} (for Slideous), where
\texttt{\$DATADIR} is the user data directory (see
\texttt{-\/-data-dir}, above). The originals may be found in pandoc's
system data directory (generally
\texttt{\$CABALDIR/pandoc-VERSION/s5/default}). Pandoc will look there
for any files it does not find in the user data directory.

For dzslides, the CSS is included in the HTML file itself, and may be
modified there.

All \href{https://revealjs.com/config/}{reveal.js configuration options}
can be set through variables. For example, themes can be used by setting
the \texttt{theme} variable:

\begin{verbatim}
-V theme=moon
\end{verbatim}

Or you can specify a custom stylesheet using the \texttt{-\/-css}
option.

To style beamer slides, you can specify a \texttt{theme},
\texttt{colortheme}, \texttt{fonttheme}, \texttt{innertheme}, and
\texttt{outertheme}, using the \texttt{-V} option:

\begin{verbatim}
pandoc -t beamer habits.txt -V theme:Warsaw -o habits.pdf
\end{verbatim}

Note that heading attributes will turn into slide attributes (on a
\texttt{\textless{}div\textgreater{}} or
\texttt{\textless{}section\textgreater{}}) in HTML slide formats,
allowing you to style individual slides. In beamer, a number of heading
classes and attributes are recognized as frame options and will be
passed through as options to the frame: see
\hyperref[frame-attributes-in-beamer]{Frame attributes in beamer},
below.

\subsection{Speaker notes}\label{speaker-notes}

Speaker notes are supported in reveal.js, PowerPoint (pptx), and beamer
output. You can add notes to your Markdown document thus:

\begin{verbatim}
::: notes

This is my note.

- It can contain Markdown
- like this list

:::
\end{verbatim}

To show the notes window in reveal.js, press \texttt{s} while viewing
the presentation. Speaker notes in PowerPoint will be available, as
usual, in handouts and presenter view.

Notes are not yet supported for other slide formats, but the notes will
not appear on the slides themselves.

\subsection{Columns}\label{columns}

To put material in side by side columns, you can use a native div
container with class \texttt{columns}, containing two or more div
containers with class \texttt{column} and a \texttt{width} attribute:

\begin{verbatim}
:::::::::::::: {.columns}
::: {.column width="40%"}
contents...
:::
::: {.column width="60%"}
contents...
:::
::::::::::::::
\end{verbatim}

\subsubsection{Additional columns attributes in
beamer}\label{additional-columns-attributes-in-beamer}

The div containers with classes \texttt{columns} and \texttt{column} can
optionally have an \texttt{align} attribute. The class \texttt{columns}
can optionally have a \texttt{totalwidth} attribute or an
\texttt{onlytextwidth} class.

\begin{verbatim}
:::::::::::::: {.columns align=center totalwidth=8em}
::: {.column width="40%"}
contents...
:::
::: {.column width="60%" align=bottom}
contents...
:::
::::::::::::::
\end{verbatim}

The \texttt{align} attributes on \texttt{columns} and \texttt{column}
can be used with the values \texttt{top}, \texttt{top-baseline},
\texttt{center} and \texttt{bottom} to vertically align the columns. It
defaults to \texttt{top} in \texttt{columns}.

The \texttt{totalwidth} attribute limits the width of the columns to the
given value.

\begin{verbatim}
:::::::::::::: {.columns align=top .onlytextwidth}
::: {.column width="40%" align=center}
contents...
:::
::: {.column width="60%"}
contents...
:::
::::::::::::::
\end{verbatim}

The class \texttt{onlytextwidth} sets the \texttt{totalwidth} to
\texttt{\textbackslash{}textwidth}.

See Section 12.7 of the
\href{http://mirrors.ctan.org/macros/latex/contrib/beamer/doc/beameruserguide.pdf}{Beamer
User's Guide} for more details.

\subsection{Frame attributes in
beamer}\label{frame-attributes-in-beamer}

Sometimes it is necessary to add the LaTeX \texttt{{[}fragile{]}} option
to a frame in beamer (for example, when using the \texttt{minted}
environment). This can be forced by adding the \texttt{fragile} class to
the heading introducing the slide:

\begin{verbatim}
# Fragile slide {.fragile}
\end{verbatim}

All of the other frame attributes described in Section 8.1 of the
\href{http://mirrors.ctan.org/macros/latex/contrib/beamer/doc/beameruserguide.pdf}{Beamer
User's Guide} may also be used: \texttt{allowdisplaybreaks},
\texttt{allowframebreaks}, \texttt{b}, \texttt{c}, \texttt{s},
\texttt{t}, \texttt{environment}, \texttt{label}, \texttt{plain},
\texttt{shrink}, \texttt{standout}, \texttt{noframenumbering},
\texttt{squeeze}. \texttt{allowframebreaks} is recommended especially
for bibliographies, as it allows multiple slides to be created if the
content overfills the frame:

\begin{verbatim}
# References {.allowframebreaks}
\end{verbatim}

In addition, the \texttt{frameoptions} attribute may be used to pass
arbitrary frame options to a beamer slide:

\begin{verbatim}
# Heading {frameoptions="squeeze,shrink,customoption=foobar"}
\end{verbatim}

\subsection{Background in reveal.js, beamer, and
pptx}\label{background-in-reveal.js-beamer-and-pptx}

Background images can be added to self-contained reveal.js slide shows,
beamer slide shows, and pptx slide shows.

\subsubsection{On all slides (beamer, reveal.js,
pptx)}\label{on-all-slides-beamer-reveal.js-pptx}

With beamer and reveal.js, the configuration option
\texttt{background-image} can be used either in the YAML metadata block
or as a command-line variable to get the same image on every slide.

For pptx, you can use a \hyperref[option--reference-doc]{reference doc}
in which background images have been set on the
\hyperref[powerpoint-layout-choice]{relevant layouts}.

\paragraph{\texorpdfstring{\texttt{parallaxBackgroundImage}
(reveal.js)}{parallaxBackgroundImage (reveal.js)}}\label{parallaxbackgroundimage-reveal.js}

For reveal.js, there is also the reveal.js-native option
\texttt{parallaxBackgroundImage}, which can be used instead of
\texttt{background-image} to produce a parallax scrolling background.
You must also set \texttt{parallaxBackgroundSize}, and can optionally
set \texttt{parallaxBackgroundHorizontal} and
\texttt{parallaxBackgroundVertical} to configure the scrolling
behaviour. See the
\href{https://revealjs.com/backgrounds/\#parallax-background}{reveal.js
documentation} for more details about the meaning of these options.

In reveal.js's overview mode, the parallaxBackgroundImage will show up
only on the first slide.

\subsubsection{On individual slides (reveal.js,
pptx)}\label{on-individual-slides-reveal.js-pptx}

To set an image for a particular reveal.js or pptx slide, add
\texttt{\{background-image="/path/to/image"\}} to the first slide-level
heading on the slide (which may even be empty).

As the \hyperref[extension-link_attributes]{HTML writers pass unknown
attributes through}, other reveal.js background settings also work on
individual slides, including \texttt{background-size},
\texttt{background-repeat}, \texttt{background-color},
\texttt{transition}, and \texttt{transition-speed}. (The \texttt{data-}
prefix will automatically be added.)

Note: \texttt{data-background-image} is also supported in pptx for
consistency with reveal.js -- if \texttt{background-image} isn't found,
\texttt{data-background-image} will be checked.

\subsubsection{On the title slide (reveal.js,
pptx)}\label{on-the-title-slide-reveal.js-pptx}

To add a background image to the automatically generated title slide for
reveal.js, use the \texttt{title-slide-attributes} variable in the YAML
metadata block. It must contain a map of attribute names and values.
(Note that the \texttt{data-} prefix is required here, as it isn't added
automatically.)

For pptx, pass a \hyperref[option--reference-doc]{reference doc} with
the background image set on the ``Title Slide'' layout.

\subsubsection{Example (reveal.js)}\label{example-reveal.js}

\begin{verbatim}
---
title: My Slide Show
parallaxBackgroundImage: /path/to/my/background_image.png
title-slide-attributes:
    data-background-image: /path/to/title_image.png
    data-background-size: contain
---

## Slide One

Slide 1 has background_image.png as its background.

## {background-image="/path/to/special_image.jpg"}

Slide 2 has a special image for its background, even though the heading has no content.
\end{verbatim}

\section{EPUBs}\label{epubs}

\subsection{EPUB Metadata}\label{epub-metadata}

EPUB metadata may be specified using the \texttt{-\/-epub-metadata}
option, but if the source document is Markdown, it is better to use a
\hyperref[extension-yaml_metadata_block]{YAML metadata block}. Here is
an example:

\begin{verbatim}
---
title:
- type: main
  text: My Book
- type: subtitle
  text: An investigation of metadata
creator:
- role: author
  text: John Smith
- role: editor
  text: Sarah Jones
identifier:
- scheme: DOI
  text: doi:10.234234.234/33
publisher:  My Press
rights: © 2007 John Smith, CC BY-NC
ibooks:
  version: 1.3.4
...
\end{verbatim}

The following fields are recognized:

\begin{description}
\item[\texttt{identifier}]
Either a string value or an object with fields \texttt{text} and
\texttt{scheme}. Valid values for \texttt{scheme} are \texttt{ISBN-10},
\texttt{GTIN-13}, \texttt{UPC}, \texttt{ISMN-10}, \texttt{DOI},
\texttt{LCCN}, \texttt{GTIN-14}, \texttt{ISBN-13},
\texttt{Legal\ deposit\ number}, \texttt{URN}, \texttt{OCLC},
\texttt{ISMN-13}, \texttt{ISBN-A}, \texttt{JP}, \texttt{OLCC}.
\item[\texttt{title}]
Either a string value, or an object with fields \texttt{file-as} and
\texttt{type}, or a list of such objects. Valid values for \texttt{type}
are \texttt{main}, \texttt{subtitle}, \texttt{short},
\texttt{collection}, \texttt{edition}, \texttt{extended}.
\item[\texttt{creator}]
Either a string value, or an object with fields \texttt{role},
\texttt{file-as}, and \texttt{text}, or a list of such objects. Valid
values for \texttt{role} are
\href{https://loc.gov/marc/relators/relaterm.html}{MARC relators}, but
pandoc will attempt to translate the human-readable versions (like
``author'' and ``editor'') to the appropriate marc relators.
\item[\texttt{contributor}]
Same format as \texttt{creator}.
\item[\texttt{date}]
A string value in \texttt{YYYY-MM-DD} format. (Only the year is
necessary.) Pandoc will attempt to convert other common date formats.
\item[\texttt{lang} (or legacy: \texttt{language})]
A string value in \href{https://tools.ietf.org/html/bcp47}{BCP 47}
format. Pandoc will default to the local language if nothing is
specified.
\item[\texttt{subject}]
Either a string value, or an object with fields \texttt{text},
\texttt{authority}, and \texttt{term}, or a list of such objects. Valid
values for \texttt{authority} are either a
\href{https://idpf.github.io/epub-registries/authorities/}{reserved
authority value} (currently \texttt{AAT}, \texttt{BIC}, \texttt{BISAC},
\texttt{CLC}, \texttt{DDC}, \texttt{CLIL}, \texttt{EuroVoc},
\texttt{MEDTOP}, \texttt{LCSH}, \texttt{NDC}, \texttt{Thema},
\texttt{UDC}, and \texttt{WGS}) or an absolute IRI identifying a custom
scheme. Valid values for \texttt{term} are defined by the scheme.
\item[\texttt{description}]
A string value.
\item[\texttt{type}]
A string value.
\item[\texttt{format}]
A string value.
\item[\texttt{relation}]
A string value.
\item[\texttt{coverage}]
A string value.
\item[\texttt{rights}]
A string value.
\item[\texttt{belongs-to-collection}]
A string value. Identifies the name of a collection to which the EPUB
Publication belongs.
\item[\texttt{group-position}]
The \texttt{group-position} field indicates the numeric position in
which the EPUB Publication belongs relative to other works belonging to
the same \texttt{belongs-to-collection} field.
\item[\texttt{cover-image}]
A string value (path to cover image).
\item[\texttt{css} (or legacy: \texttt{stylesheet})]
A string value (path to CSS stylesheet).
\item[\texttt{page-progression-direction}]
Either \texttt{ltr} or \texttt{rtl}. Specifies the
\texttt{page-progression-direction} attribute for the
\href{http://idpf.org/epub/301/spec/epub-publications.html\#sec-spine-elem}{\texttt{spine}
element}.
\item[\texttt{ibooks}]
iBooks-specific metadata, with the following fields:

\begin{itemize}
\tightlist
\item
  \texttt{version}: (string)
\item
  \texttt{specified-fonts}: \texttt{true}\textbar{}\texttt{false}
  (default \texttt{false})
\item
  \texttt{ipad-orientation-lock}:
  \texttt{portrait-only}\textbar{}\texttt{landscape-only}
\item
  \texttt{iphone-orientation-lock}:
  \texttt{portrait-only}\textbar{}\texttt{landscape-only}
\item
  \texttt{binding}: \texttt{true}\textbar{}\texttt{false} (default
  \texttt{true})
\item
  \texttt{scroll-axis}:
  \texttt{vertical}\textbar{}\texttt{horizontal}\textbar{}\texttt{default}
\end{itemize}
\end{description}

\subsection{\texorpdfstring{The \texttt{epub:type}
attribute}{The epub:type attribute}}\label{the-epubtype-attribute}

For \texttt{epub3} output, you can mark up the heading that corresponds
to an EPUB chapter using the
\href{http://www.idpf.org/epub/31/spec/epub-contentdocs.html\#sec-epub-type-attribute}{\texttt{epub:type}
attribute}. For example, to set the attribute to the value
\texttt{prologue}, use this markdown:

\begin{verbatim}
# My chapter {epub:type=prologue}
\end{verbatim}

Which will result in:

\begin{verbatim}
<body epub:type="frontmatter">
  <section epub:type="prologue">
    <h1>My chapter</h1>
\end{verbatim}

Pandoc will output
\texttt{\textless{}body\ epub:type="bodymatter"\textgreater{}}, unless
you use one of the following values, in which case either
\texttt{frontmatter} or \texttt{backmatter} will be output.

\begin{longtable}[]{@{}ll@{}}
\toprule\noalign{}
\texttt{epub:type} of first section & \texttt{epub:type} of body \\
\midrule\noalign{}
\endhead
\bottomrule\noalign{}
\endlastfoot
prologue & frontmatter \\
abstract & frontmatter \\
acknowledgments & frontmatter \\
copyright-page & frontmatter \\
dedication & frontmatter \\
credits & frontmatter \\
keywords & frontmatter \\
imprint & frontmatter \\
contributors & frontmatter \\
other-credits & frontmatter \\
errata & frontmatter \\
revision-history & frontmatter \\
titlepage & frontmatter \\
halftitlepage & frontmatter \\
seriespage & frontmatter \\
foreword & frontmatter \\
preface & frontmatter \\
frontispiece & frontmatter \\
appendix & backmatter \\
colophon & backmatter \\
bibliography & backmatter \\
index & backmatter \\
\end{longtable}

\subsection{Linked media}\label{linked-media}

By default, pandoc will download media referenced from any
\texttt{\textless{}img\textgreater{}},
\texttt{\textless{}audio\textgreater{}},
\texttt{\textless{}video\textgreater{}} or
\texttt{\textless{}source\textgreater{}} element present in the
generated EPUB, and include it in the EPUB container, yielding a
completely self-contained EPUB. If you want to link to external media
resources instead, use raw HTML in your source and add
\texttt{data-external="1"} to the tag with the \texttt{src} attribute.
For example:

\begin{verbatim}
<audio controls="1">
  <source src="https://example.com/music/toccata.mp3"
          data-external="1" type="audio/mpeg">
  </source>
</audio>
\end{verbatim}

If the input format already is HTML then \texttt{data-external="1"} will
work as expected for \texttt{\textless{}img\textgreater{}} elements.
Similarly, for Markdown, external images can be declared with
\texttt{!{[}img{]}(url)\{external=1\}}. Note that this only works for
images; the other media elements have no native representation in
pandoc's AST and require the use of raw HTML.

\subsection{EPUB styling}\label{epub-styling}

By default, pandoc will include some basic styling contained in its
\texttt{epub.css} data file. (To see this, use
\texttt{pandoc\ -\/-print-default-data-file\ epub.css}.) To use a
different CSS file, just use the \texttt{-\/-css} command line option. A
few inline styles are defined in addition; these are essential for
correct formatting of pandoc's HTML output.

The \texttt{document-css} variable may be set if the more opinionated
styling of pandoc's default HTML templates is desired (and in that case
the variables defined in \hyperref[variables-for-html]{Variables for
HTML} may be used to fine-tune the style).

\section{Jupyter notebooks}\label{jupyter-notebooks}

When creating a
\href{https://nbformat.readthedocs.io/en/latest/}{Jupyter notebook},
pandoc will try to infer the notebook structure. Code blocks with the
class \texttt{code} will be taken as code cells, and intervening content
will be taken as Markdown cells. Attachments will automatically be
created for images in Markdown cells. Metadata will be taken from the
\texttt{jupyter} metadata field. For example:

\begin{verbatim}
---
title: My notebook
jupyter:
  nbformat: 4
  nbformat_minor: 5
  kernelspec:
     display_name: Python 2
     language: python
     name: python2
  language_info:
     codemirror_mode:
       name: ipython
       version: 2
     file_extension: ".py"
     mimetype: "text/x-python"
     name: "python"
     nbconvert_exporter: "python"
     pygments_lexer: "ipython2"
     version: "2.7.15"
---

# Lorem ipsum

**Lorem ipsum** dolor sit amet, consectetur adipiscing elit. Nunc luctus
bibendum felis dictum sodales.

``` code
print("hello")
```

## Pyout

``` code
from IPython.display import HTML
HTML("""
<script>
console.log("hello");
</script>
<b>HTML</b>
""")
```

## Image

This image ![image](myimage.png) will be
included as a cell attachment.
\end{verbatim}

If you want to add cell attributes, group cells differently, or add
output to code cells, then you need to include divs to indicate the
structure. You can use either \hyperref[extension-fenced_divs]{fenced
divs} or \hyperref[extension-native_divs]{native divs} for this. Here is
an example:

\begin{verbatim}
:::::: {.cell .markdown}
# Lorem

**Lorem ipsum** dolor sit amet, consectetur adipiscing elit. Nunc luctus
bibendum felis dictum sodales.
::::::

:::::: {.cell .code execution_count=1}
``` {.python}
print("hello")
```

::: {.output .stream .stdout}
```
hello
```
:::
::::::

:::::: {.cell .code execution_count=2}
``` {.python}
from IPython.display import HTML
HTML("""
<script>
console.log("hello");
</script>
<b>HTML</b>
""")
```

::: {.output .execute_result execution_count=2}
```{=html}
<script>
console.log("hello");
</script>
<b>HTML</b>
hello
```
:::
::::::
\end{verbatim}

If you include raw HTML or TeX in an output cell, use the {[}raw
attribute{]}{[}Extension: \texttt{fenced\_attribute}{]}, as shown in the
last cell of the example above. Although pandoc can process ``bare'' raw
HTML and TeX, the result is often interspersed raw elements and normal
textual elements, and in an output cell pandoc expects a single,
connected raw block. To avoid using raw HTML or TeX except when marked
explicitly using raw attributes, we recommend specifying the extensions
\texttt{-raw\_html-raw\_tex+raw\_attribute} when translating between
Markdown and ipynb notebooks.

Note that options and extensions that affect reading and writing of
Markdown will also affect Markdown cells in ipynb notebooks. For
example, \texttt{-\/-wrap=preserve} will preserve soft line breaks in
Markdown cells; \texttt{-\/-atx-headers} will cause ATX-style headings
to be used; and \texttt{-\/-preserve-tabs} will prevent tabs from being
turned to spaces.

\section{Syntax highlighting}\label{syntax-highlighting}

Pandoc will automatically highlight syntax in
\hyperref[fenced-code-blocks]{fenced code blocks} that are marked with a
language name. The Haskell library
\href{https://github.com/jgm/skylighting}{skylighting} is used for
highlighting. Currently highlighting is supported only for HTML, EPUB,
Docx, Ms, and LaTeX/PDF output. To see a list of language names that
pandoc will recognize, type
\texttt{pandoc\ -\/-list-highlight-languages}.

The color scheme can be selected using the \texttt{-\/-highlight-style}
option. The default color scheme is \texttt{pygments}, which imitates
the default color scheme used by the Python library pygments (though
pygments is not actually used to do the highlighting). To see a list of
highlight styles, type \texttt{pandoc\ -\/-list-highlight-styles}.

If you are not satisfied with the predefined styles, you can use
\texttt{-\/-print-highlight-style} to generate a JSON \texttt{.theme}
file which can be modified and used as the argument to
\texttt{-\/-highlight-style}. To get a JSON version of the
\texttt{pygments} style, for example:

\begin{verbatim}
pandoc --print-highlight-style pygments > my.theme
\end{verbatim}

Then edit \texttt{my.theme} and use it like this:

\begin{verbatim}
pandoc --highlight-style my.theme
\end{verbatim}

If you are not satisfied with the built-in highlighting, or you want to
highlight a language that isn't supported, you can use the
\texttt{-\/-syntax-definition} option to load a
\href{https://docs.kde.org/stable5/en/kate/katepart/highlight.html}{KDE-style
XML syntax definition file}. Before writing your own, have a look at
KDE's
\href{https://github.com/KDE/syntax-highlighting/tree/master/data/syntax}{repository
of syntax definitions}.

To disable highlighting, use the \texttt{-\/-no-highlight} option.

\section{Custom Styles}\label{custom-styles}

Custom styles can be used in the docx and ICML formats.

\subsection{Output}\label{output}

By default, pandoc's docx and ICML output applies a predefined set of
styles for blocks such as paragraphs and block quotes, and uses largely
default formatting (italics, bold) for inlines. This will work for most
purposes, especially alongside a \texttt{reference.docx} file. However,
if you need to apply your own styles to blocks, or match a preexisting
set of styles, pandoc allows you to define custom styles for blocks and
text using \texttt{div}s and \texttt{span}s, respectively.

If you define a \texttt{div} or \texttt{span} with the attribute
\texttt{custom-style}, pandoc will apply your specified style to the
contained elements (with the exception of elements whose function
depends on a style, like headings, code blocks, block quotes, or links).
So, for example, using the \texttt{bracketed\_spans} syntax,

\begin{verbatim}
[Get out]{custom-style="Emphatically"}, he said.
\end{verbatim}

would produce a docx file with ``Get out'' styled with character style
\texttt{Emphatically}. Similarly, using the \texttt{fenced\_divs}
syntax,

\begin{verbatim}
Dickinson starts the poem simply:

::: {custom-style="Poetry"}
| A Bird came down the Walk---
| He did not know I saw---
:::
\end{verbatim}

would style the two contained lines with the \texttt{Poetry} paragraph
style.

For docx output, styles will be defined in the output file as inheriting
from normal text, if the styles are not yet in your reference.docx. If
they are already defined, pandoc will not alter the definition.

This feature allows for greatest customization in conjunction with
\href{https://pandoc.org/filters.html}{pandoc filters}. If you want all
paragraphs after block quotes to be indented, you can write a filter to
apply the styles necessary. If you want all italics to be transformed to
the \texttt{Emphasis} character style (perhaps to change their color),
you can write a filter which will transform all italicized inlines to
inlines within an \texttt{Emphasis} custom-style \texttt{span}.

For docx output, you don't need to enable any extensions for custom
styles to work.

\subsection{Input}\label{input}

The docx reader, by default, only reads those styles that it can convert
into pandoc elements, either by direct conversion or interpreting the
derivation of the input document's styles.

By enabling the \hyperref[ext-styles]{\texttt{styles} extension} in the
docx reader (\texttt{-f\ docx+styles}), you can produce output that
maintains the styles of the input document, using the
\texttt{custom-style} class. Paragraph styles are interpreted as divs,
while character styles are interpreted as spans.

For example, using the \texttt{custom-style-reference.docx} file in the
test directory, we have the following different outputs:

Without the \texttt{+styles} extension:

\begin{verbatim}
$ pandoc test/docx/custom-style-reference.docx -f docx -t markdown
This is some text.

This is text with an *emphasized* text style. And this is text with a
**strengthened** text style.

> Here is a styled paragraph that inherits from Block Text.
\end{verbatim}

And with the extension:

\begin{verbatim}
$ pandoc test/docx/custom-style-reference.docx -f docx+styles -t markdown

::: {custom-style="First Paragraph"}
This is some text.
:::

::: {custom-style="Body Text"}
This is text with an [emphasized]{custom-style="Emphatic"} text style.
And this is text with a [strengthened]{custom-style="Strengthened"}
text style.
:::

::: {custom-style="My Block Style"}
> Here is a styled paragraph that inherits from Block Text.
:::
\end{verbatim}

With these custom styles, you can use your input document as a
reference-doc while creating docx output (see below), and maintain the
same styles in your input and output files.

\section{Custom readers and writers}\label{custom-readers-and-writers}

Pandoc can be extended with custom readers and writers written in
\href{https://www.lua.org}{Lua}. (Pandoc includes a Lua interpreter, so
Lua need not be installed separately.)

To use a custom reader or writer, simply specify the path to the Lua
script in place of the input or output format. For example:

\begin{verbatim}
pandoc -t data/sample.lua
pandoc -f my_custom_markup_language.lua -t latex -s
\end{verbatim}

If the script is not found relative to the working directory, it will be
sought in the \texttt{readers} or \texttt{writers} subdirectory of the
user data directory (see \texttt{-\/-data-dir}).

A custom reader is a Lua script that defines one function, Reader, which
takes a string as input and returns a Pandoc AST. See the
\href{https://pandoc.org/lua-filters.html}{Lua filters documentation}
for documentation of the functions that are available for creating
pandoc AST elements. For parsing, the
\href{http://www.inf.puc-rio.br/~roberto/lpeg/}{lpeg} parsing library is
available by default. To see a sample custom reader:

\begin{verbatim}
pandoc --print-default-data-file creole.lua
\end{verbatim}

If you want your custom reader to have access to reader options
(e.g.~the tab stop setting), you give your Reader function a second
\texttt{options} parameter.

A custom writer is a Lua script that defines a function that specifies
how to render each element in a Pandoc AST. To see a documented example
which you can modify according to your needs:

\begin{verbatim}
pandoc --print-default-data-file sample.lua
\end{verbatim}

Note that custom writers have no default template. If you want to use
\texttt{-\/-standalone} with a custom writer, you will need to specify a
template manually using \texttt{-\/-template} or add a new default
template with the name \texttt{default.NAME\_OF\_CUSTOM\_WRITER.lua} to
the \texttt{templates} subdirectory of your user data directory (see
\hyperref[templates]{Templates}).

\section{Reproducible builds}\label{reproducible-builds}

Some of the document formats pandoc targets (such as EPUB, docx, and
ODT) include build timestamps in the generated document. That means that
the files generated on successive builds will differ, even if the source
does not. To avoid this, set the \texttt{SOURCE\_DATE\_EPOCH}
environment variable, and the timestamp will be taken from it instead of
the current time. \texttt{SOURCE\_DATE\_EPOCH} should contain an integer
unix timestamp (specifying the number of seconds since midnight UTC
January 1, 1970).

Some document formats also include a unique identifier. For EPUB, this
can be set explicitly by setting the \texttt{identifier} metadata field
(see \hyperref[epub-metadata]{EPUB Metadata}, above).

\section{Running pandoc as a web
server}\label{running-pandoc-as-a-web-server}

If you rename (or symlink) the pandoc executable to
\texttt{pandoc-server}, it will start up a web server with a JSON API.
This server exposes most of the conversion functionality of pandoc. For
full documentation, see the
\href{https://github.com/jgm/pandoc/blob/master/doc/pandoc-server.md}{pandoc-server}
man page.

If you rename (or symlink) the pandoc executable to
\texttt{pandoc-server.cgi}, it will function as a CGI program exposing
the same API as \texttt{pandoc-server}.

\texttt{pandoc-server} is designed to be maximally secure; it uses
Haskell's type system to provide strong guarantees that no I/O will be
performed on the server during pandoc conversions.

\section{A note on security}\label{a-note-on-security}

\begin{enumerate}
\def\labelenumi{\arabic{enumi}.}
\item
  Although pandoc itself will not create or modify any files other than
  those you explicitly ask it create (with the exception of temporary
  files used in producing PDFs), a filter or custom writer could in
  principle do anything on your file system. Please audit filters and
  custom writers very carefully before using them.
\item
  Several input formats (including HTML, Org, and RST) support
  \texttt{include} directives that allow the contents of a file to be
  included in the output. An untrusted attacker could use these to view
  the contents of files on the file system. (Using the
  \texttt{-\/-sandbox} option can protect against this threat.)
\item
  Several output formats (including RTF, FB2, HTML with
  \texttt{-\/-self-contained}, EPUB, Docx, and ODT) will embed encoded
  or raw images into the output file. An untrusted attacker could
  exploit this to view the contents of non-image files on the file
  system. (Using the \texttt{-\/-sandbox} option can protect against
  this threat, but will also prevent including images in these formats.)
\item
  If your application uses pandoc as a Haskell library (rather than
  shelling out to the executable), it is possible to use it in a mode
  that fully isolates pandoc from your file system, by running the
  pandoc operations in the \texttt{PandocPure} monad. See the document
  \href{https://pandoc.org/using-the-pandoc-api.html}{Using the pandoc
  API} for more details. (This corresponds to the use of the
  \texttt{-\/-sandbox} option on the command line.)
\item
  Pandoc's parsers can exhibit pathological performance on some corner
  cases. It is wise to put any pandoc operations under a timeout, to
  avoid DOS attacks that exploit these issues. If you are using the
  pandoc executable, you can add the command line options
  \texttt{+RTS\ -M512M\ -RTS} (for example) to limit the heap size to
  512MB. Note that the \texttt{commonmark} parser (including
  \texttt{commonmark\_x} and \texttt{gfm}) is much less vulnerable to
  pathological performance than the \texttt{markdown} parser, so it is a
  better choice when processing untrusted input.
\item
  The HTML generated by pandoc is not guaranteed to be safe. If
  \texttt{raw\_html} is enabled for the Markdown input, users can inject
  arbitrary HTML. Even if \texttt{raw\_html} is disabled, users can
  include dangerous content in URLs and attributes. To be safe, you
  should run all HTML generated from untrusted user input through an
  HTML sanitizer.
\end{enumerate}

\section{Authors}\label{authors}

Copyright 2006--2022 John MacFarlane (jgm@berkeley.edu). Released under
the \href{https://www.gnu.org/copyleft/gpl.html}{GPL}, version 2 or
greater. This software carries no warranty of any kind. (See COPYRIGHT
for full copyright and warranty notices.) For a full list of
contributors, see the file AUTHORS.md in the pandoc source code.

\end{document}
